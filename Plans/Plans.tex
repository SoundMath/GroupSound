\documentclass[11pt]{article}
\usepackage[margin=1in]{geometry}
\usepackage{enumerate,amsmath,amssymb,fancyhdr,mathrsfs,amscd,stmaryrd,amsthm}
\usepackage{graphicx}
\usepackage{color}
\usepackage{tikz}
\usepackage{url}
%\usepackage[colorlinks=true,urlcolor=black,linkcolor=black,citecolor=black]{hyperref}
\usepackage[colorlinks=true,urlcolor=blue,linkcolor=blue,citecolor=blue]{hyperref}
\usepackage[printonlyused,smaller]{acronym}

\newcommand\C{\ensuremath{\mathbb{C}}}
\newcommand\Z{\ensuremath{\mathbb{Z}}}
\newcommand\N{\ensuremath{\mathbb{N}}}
\newcommand\vf{\ensuremath{\mathbf{f}}}
\newcommand\sL{\ensuremath{\mathcal{L}}}
%linear transformations \lt should be sans-serif
\newcommand{\lt}[1]{\ensuremath{\mathsf{#1}}}
\newcommand{\ltb}[1]{\ensuremath{[\mathsf{#1}]}}
\newcommand{\linmap}[3]{\ensuremath{\lt{#1}: \vs{#2} \rightarrow \vs{#3}}}
\renewcommand{\S}{\lt{S}}       % a linear operator (\eg translation)
\newcommand{\T}{\lt{T}}       % a linear operator (\eg translation)
%\DeclareMathOperator\T{T}
%\DeclareMathOperator\conv{C}
\newcommand\conv{\lt{C}}

\begin{document}
\begin{center}
{\bf What does a nonabelian group sound like?}\\
Reginald Bain, Matthew Corley, William DeMeo
\end{center}

\section{Short term goals}

\noindent {\bf Choice of programming language and other software.}
\\[5pt] (wjd: December 15, 2013)
This will be an ongoing discussion, but to get the ball rolling it seems a
functional programming language like Scala would be ideal for expressing the
math that we want to implement, so I suggest we start with that as our primary
language and use Java as a fallback.  

\vskip5mm

\noindent {\bf Incorporating finite groups into Scala or Java programs.}  
\\[5pt] (wjd: December 15, 2013)
One of our primary goals is to write a program to do convolution of signals
defined on finite nonabelian groups, so we need to incorporate finite
groups into our programs.  
The main thing we need is the multiplication table---that is, the table that
tells us the result of multiplying any two group elements---as well as
a function that tells us the inverse of a given element.
We could write programs to do all this from scratch, especially for small groups
that are easy to represent (like semidirect products of cyclic groups).
However, there is plenty of code already written for
finite groups, and we should avoid reinventing this wheel.
I suggest the following strategy instead:
\begin{enumerate}
\item Use the \acs{GAP} software for exploring various finite groups to
  identifying some groups we want to consider.
\item Use my \acs{GAP} program
  \href{http://universalalgebra.wordpress.com/documentation/gap/gap-and-uacalc/}{gap2uacalc}
  which writes \acs{GAP} groups to \acs{XML} files and represents them as general (universal)
  algebras---i.e., simply a set with operations defined on the set. 
  (More details about this and the \acs{XML} format used by 
\href{http://universalalgebra.wordpress.com/documentation/gap/gap-and-uacalc/}{gap2uacalc}
 will appear below.)
\item To read the resulting \acs{XML} files into our Scala or Java programs,
  we can use a UACalc Java class called AlgebraIO 
  (see \href{http://uacalc.org/doc/}{the UACalc javadoc}). (The
  \href{http://universalalgebra.wordpress.com/documentation/gap/gap-and-uacalc/}{gap2uacalc}
  program writes groups to a file in UACalc's \acs{XML} format.)
\end{enumerate}

\noindent {\bf Convolution background.}
\\[5pt]  (wjd: December 15, 2013)
The main operation/function we will considering is convolution, and we should
think about how best to view this mathematically, as well as how best to
represent it in the computer.  In this section is some background on the
mathematical aspects.  In the next section are some initial thoughts on how to
code this up in Scala.

Let $\C^G$ denote the set of complex valued functions defined on the group
$G$.  That is 
\[
\C^G = \{f : G\rightarrow \C \}.
\]
(In the Tolimieri-An books, \cite{Tolimieri:1998} \cite{Tolimieri:2003}, this set is also denoted by
$\sL(G)$.)  If the group has $|G| = n$ elements, say, 
$G = \{x_0, x_1, \dots, x_{n-1}\}$, 
then each function $f\in  \C^G$ can be represented as
a length-$n$ vector in $\C^n$---namely, the vector of its values on $G$:
\[
\vf = [f(x_0), f(x_1), \dots, f(x_{n-1})].
\]

Given two functions $f$ and $g$ in $\C^G$, the \emph{convolution of
$f$ and $g$}, denoted, $f*g$, is also function in $\C^G$ and is
defined by the values it takes at each $x\in G$ as follows:
\begin{equation}
\label{eq:1}  
(f*g)(x) = \sum_{y \in G} f(y) g(y^{-1}x).
\end{equation}
Note that this is a weighted sum of translations of $g$.
Indeed, let $\T_y: \C^G\rightarrow \C^G$ denote
the \emph{translation by $y$} operator---that is, $\T_y$ maps 
a function $g\in \C^G$ to a translated version of itself, $\T_y(g)$, which is defined at each 
$x \in G$ by $\T_y(g)(x) = g(y^{-1}x)$.
Then~(\ref{eq:1}) can be written as
\begin{equation}
\label{eq:4}  
(f*g)(x) = \sum_{y \in G} f(y) \T_y(g)(x),
\end{equation}
a sum of weighted translations of $g$ where 
the coefficients $f(y)$ are the weights, and 
$\T_y(g)$ is the function $g$ ``shifted'' by $y$. (When $G$ is the abelian
group $\Z/n\Z$ with addition modulo $n$, we have
$\T_y(g)(x) = g(y^{-1}x) = g(x-y)$, so in this case
$\T_y(g)$ is literally $g$ shifted by $y$ units to the right.)

Equation~(\ref{eq:4}) defines the convolution, $f * g$, by giving its
value at each $x\in G$. Using the 
translation operator, however, we can define convolution
 ``functionally,'' instead of element-wise, as follows:
\begin{equation}
\label{eq:2}  
f*g = \sum_{y \in G} f(y) \T_y(g)
\end{equation}
(Pause to look at the right hand side of~(\ref{eq:2}), and let it sink in that
this is a function that takes arguments $x\in G$; compare with the right
hand side of~(\ref{eq:4}).)

 This is fine, but it is also useful to think of~(\ref{eq:2}) as $f$ acting
on $g$.  Indeed, on the right hand side of~(\ref{eq:2}) we have the operator 
$\sum_{y \in G} f(y) \T_y$ that maps the function $g$ to the function $f*g$.  But
on the left hand side we have a binary operation $f*g$, written in infix
notation, which doesn't jibe very well with this functional interpretation.  So,
instead of saying ``the convolution of $f$ and $g$'', and writing $f*g$, we will
say
%% \emph{convolution by $f$ of $g$} 
``the convolution {\bf by} $f$ {\bf of} $g$,'' and write $\conv(f)(g)$. In this
way, we have the \emph{convolution by $f$} operator:
\begin{equation}
\label{eq:3}  
\conv(f) = \sum_{y \in G} f(y) \T_y,
\end{equation}
which is a weighted sum of translation operators.  The function $\conv(f)$ takes other
functions, like $g$, as its argument.

So, the functional types we have here are the following:
\[
\conv : \C^G \rightarrow (\C^G)^{\C^G}
\]
Given $f\in \C^G$,
\[
\conv(f) : \C^G \rightarrow \C^G
\]
Given $f\in \C^G$ and $g\in \C^G$,
\[
\conv(f)(g) : G \rightarrow \C
\]
Or, in the notational style of a functional programming language like Scala:
\[
\conv : (G \Rightarrow \C) \Rightarrow ( (G \Rightarrow \C) \Rightarrow (G \Rightarrow \C) )
\]
Given $f\in \C^G$,
\[
\conv(f) : (G \Rightarrow \C) \Rightarrow (G \Rightarrow \C)
\]
Given $f\in \C^G$ and $g\in \C^G$,
\[
\conv(f)(g) : (G \Rightarrow \C)
\]

\newpage

\noindent {\bf Initial thoughts on implementing convolution in Scala.}
\\[5pt]  (wjd: December 15, 2013)
Functions in Scala can have multiple argument lists. If we have, say, 4
argument lists, then we can write
{\small
\begin{verbatim}
    def f(args1)(args2)(args3)(args4) = E
\end{verbatim}}
\noindent where {\tt E} is some expression involving the arguments in the lists.
This is equivalent to
{\small
\begin{verbatim}
    def f(args1)(args2)(args3) = (args4 => E)
\end{verbatim}}
\noindent because {\tt f(args1)(args2)(args3)} can be thought of as a function that takes
as input the last argument list, {\tt args4}, and returns {\tt E}.
Another way to write this is
{\small
\begin{verbatim}
    def f = (args1 => (args2 => (args3 => (args4 => E) ) ) )
\end{verbatim}}
\noindent (This is called \emph{Currying}, named after one of its proponents, Haskell Curry.)
\\[5pt]  
Here's what the interface to a convolution function might look like in Scala:
{\small
\begin{verbatim}
    def convolution(f: Int => Double)(g: Int => Double)(x: Int): Double = {

        // insert convolution algorithm

    }
\end{verbatim}}
\noindent This expresses precisely the functional view of convolution
described mathematically in the previous.  Given this interface, for each 
function $f \in \C^G$ we could define \emph{convolution by $f$} as
{\small
\begin{verbatim}
    def Cf = convolution(f)_
\end{verbatim}}
\noindent (The underscore is required in Scala when you leave off argument
lists.) 
\\[5pt]  
Then, given another function $g \in \C^G$, we could define
{\small
\begin{verbatim}
    def fg = Cf(g)
\end{verbatim}
}  
\noindent Thus, we have 
\emph{convolution by $f$ of $g$} defined functionally, rather than pointwise.
Of course, we can evaluate this function at various points: {\tt fg(x)}.

Alternatively, it might make more sense to take an object oriented approach (let's
discuss).  For example, it probably makes sense to define a CG class to
represent complex valued functions defined on groups, and define the binary
operation * for this class to be convolution of two CG objects.  This can also
be done very simply and elegantly in Scala. 

Perhaps we should implement convolution both ways---I don't think it will be
very difficult---and then use whichever seems more natural in a given context.
In Tolimieri-An, there is a nice discussion of the mathematics of these two
alternative views---i.e., convolution as a binary operation defined on $\sL(G)$, versus convolution
as a functional defined on $\C G$. (Note: 
$\sL(G)$ and $\C G$ are simply two mathematical representations of the
same object, that is, the set of complex valued functions defined on the group
$G$.)

We should also discuss whether we will need to use complex valued functions, or if
real valued functions will suffice.


\newpage

\appendix

\begin{center}
  \textsc{Appendix}
\end{center}
\section{Excerpts from Original Proposal}

\noindent {\bf Abstract.}
Underlying many \ac{DSP} algorithms, in particular those
used for digital audio filters, is the convolution operation.  This operation
acts on a signal, $f(x)$, and can be viewed as a weighted sum of translations,
$f(x-y)$. Most classical results of \ac{DSP} are easily and elegantly derived if we
define our functions on $\mathbb{Z}/n$, the abelian (or commutative) group of
integers modulo $n$ (see~\cite{Tolimieri:1998}).
  The term \emph{abelian} here refers to the fact that the basic
group operation is addition (modulo $n$) which is a commutative operation (i.e.,
$x+y = y+x$).

If we replace this ``index set'' (the set on which functions are defined) with a
\emph{nonabelian} group---where the group operation is now multiplication,
$xy$---then instead of the usual translation, $f(x-y)$, we have a
\emph{generalized translation}, $f(y^{-1}x)$. 
If we carry out convolution using this generalized translation, the
resulting audio filters will naturally produce different effects than those
obtained with ordinary (abelian group) convolution.  


Dr.~DeMeo, initiated research based on these ideas in
2004 and presented some preliminary findings at the International Symposium on Musical
Acoustics (see~\cite{nonabeliandsp}, which received a ``best paper'' award).
Similar ideas have been successfully applied to two dimensional image data
as well as to other areas of engineering (see~\cite{Chirikjian:2002}
and~\cite{Tolimieri:2003}).  However, to date the application of 
nonabelian groups to audio signal processing seems relatively unexplored, and
there are a number of fundamental open questions in this area that we hope to
answer.  
 
\vskip5mm

\noindent {\bf Research question.}
If the underlying index set of a digital audio filtering algorithm is modified
to use various nonabelian groups (instead of the commonly used abelian group),
how does this change the behavior of the filter and the resulting audio output?

\vskip5mm

\noindent {\bf Project time-line.}\\
%% 6. Project timeline – Provide an overview of the timing for specific steps of
%% your project. This does not need to be a day to day list but depending on the
%% length of your project, it may give an overview biweekly or monthly. Be sure to
%% include time to review/synthesize your data or to reflect on the experience and
%% time to write the final report. This section can include a pre and post grant
%% period, if you have already started your project and/or plan to continue working
%% on this after the grant period ends. Review Section IV of this document for
%% additional assistance with this section. 
{\it October 2013--December 2013:} Become more familiar with music
analysis/synthesis and \ac{DSP} algorithms, and gain further knowledge of group
theory and its role in classical \ac{DSP} implementations. 
\\[5pt]
{\it January 2014--April 2014:} Write code to implement algorithms for
general nonabelian group \ac{DSP}.  
Identify specific characteristics of groups that make them more (or less)
useful as an index set on which to define \ac{DSP} operations like convolution.
\\[5pt]
{\it May 2014--October 2014:} Gather and analyze results, and write up reports.  
Submit manuscript to an academic journal.  Prepare
for and attend conferences.

\vskip5mm


\noindent {\bf Project goals and objectives.}
We propose to explore the idea of using the underlying finite group (i.e., the
index set) as an adjustable parameter of a digital audio filter.  By listening to
samples produced using various nonabelian groups, we hope to get a sense of the
``acoustical characters'' of finite groups.  We will attempt to associate these
acoustical features with various mathematical properties of the groups, and
develop a classification scheme that might be useful to
practitioners in audio signal processing and computer music composition.

%% There are some basic classes of nonabelian groups that have been studied by
%% mathematicians for more than 100 years.  For example, the symmetric and 
%% alternating groups, dihedral groups, semidirect products, wreath products,
%% etc.  These classes are now well understood and cataloged 
%% (see, e.g.,~\cite{ATLAS:1986}), and there are billions of examples at our
%% fingertips in the SmallGroups library of the GAP Computer Algebra System~\cite{GAP4}.   
\begin{itemize}
\item {\it Goals:}
Develop the mathematical theory necessary to provide sonic characterizations of
nonabelian groups. Discover which mathematical features of a group can be used to
describe how a given \ac{DSP} algorithm based on that group will behave. 
Produce computer software that allows users to process and manipulate musical
signals using nonabelian group filters.
\item 
{\it Objective 1:} Develop an understanding of the basic math underlying signal
processing algorithms in general and convolution in particular and show
mathematically what effects the use of a nonabelian group index set will have
on the convolution operation.
\item
{\it Objective 2:} Find a short list of nonabelian groups that are useful for
nonabelian group audio filters and effects processors, prove their effectiveness
both mathematically and experimentally, and document these discoveries.
\item
{\it Objective 3:} Implement a software program that takes an audio signal as input and
allows the user to apply filters corresponding to specific nonabelian groups to
achieve different effects. 
\end{itemize}

 
\vskip5mm

\noindent {\bf Methodology.}
We will conduct controlled experiments with very simple sound signals at first
(sine waves and linear chirps), and filter these signals using
standard convolution.  Then we will filter the original signals using a
generalized (nonabelian) convolution, substituting the underlying index set with
various groups from the wide variety of nonabelian groups available in 
the SmallGroups library of \acs{GAP}~\cite{GAP4}.   

When we replace the index set $\mathbb{Z}/n$ with various finite nonabelian
groups, in the beginning, the simplest examples of nonabelian groups (such as
semidirect product groups), will be constructed ``by hand''using \acs{GAP}'s {\tt
  SemidirectProduct} function.  Groups with a more complicated structure will be
selected from \acs{GAP}'s vast SmallGroups library using various selection criteria.
For each of the groups tested, we will implement the
convolution function using the generalized (nonabelian) translation $f(y^{-1}x)$
in place of ordinary translation $f(x-y)$ used in classical convolution. 

After completing these controlled experiments, we will analyze the results to
compare the effects of the choice of group on the resulting convolution filter.
Finally, we will attempt to make a connection between the mathematical
properties of the group and the acoustical properties of the resulting
convolution.  

Both \acs{GAP} and Matlab will be used for much of the initial prototyping and
testing. Matlab provides easy methods for constructing wav files ``from 
scratch'' with its {\tt wavwrite()} function. Additionally, Myoung An (a colleague of
Dr.~DeMeo) has provided us with the Matlab code that she and Richard Tolimieri
developed for their work in image processing, where they applied nonabelian group
filters to the processing of 2D digital images. This code will be a valuable
resource as we seek to apply similar ideas to audio signal processing.  

As the project progresses, we will likely use the JavaSound library and
implement our generalized \ac{DSP} algorithms in Java.  JavaSound provides methods
for reading and altering wav files frequencies and sound intensity levels, which
will prove useful when we apply our generalized \ac{DSP} algorithms to more complex
sounds.

\vskip5mm


\noindent {\bf Anticipated results, final products, and dissemination.}
By the end of the Spring 2014 semester, I expect to have written Matlab programs
to test the results of the modified \ac{DSP} implementations described above.  I also
expect to have developed a Java software program which allows easy application
of nonabelian group filters through a graphical user interface.  I hope that the
results will prove interesting and have practical applications for computer
music composition.  

The abstract for this project has already been accepted for presentation at the
Joint Mathematics Meetings in Baltimore in 2014.  In addition, I will submit the
work to the \ac{ICMC}, the \ac{ISMA}, and the 14th International
Conference on \ac{NIME}. Previous work on
topics related to this proposal by my faculty mentors have been accepted at both
ICMC and ISMA,
% (see~\cite{nonabeliandsp} and~\cite{dissonance})
 so we have high expectations for this project. I will write up a formal article 
describing the research and results and submit the manuscript to at least
one scholarly journal in mathematics or music. Finally, if my project proposal
is accepted and I become a Magellan Scholar, I will be honored to present the
work at Discovery Day 2014.   

\vskip5mm



\section*{List of Acronyms}
\begin{acronym}
\acro{GAP}{Group, Algorithms, Programming\protect\footnote{\url{http://gap-system.org/}}}
\acro{XML}{Extensible Markup Language}
\acro{ICMC}{International Computer Music Conference\protect\footnote{\url{http://www.computermusic.org/page/23/}}}
\acro{ISMA}{International Symposium on Musical Acoustics\protect\footnote{\url{http://isma.univ-lemans.fr/en/index.html}}}
\acro{NIME}{New Interfaces for Musical Expression\protect\footnote{\url{http://www.nime.org/nime2014/}}}
\acro{DSP}{digital signal processing}
\end{acronym}



%% \bibliographystyle{plainurl}
%% \bibliography{wjd}
\def\cprime{$'$} \def\cprime{$'$}
  \def\ocirc#1{\ifmmode\setbox0=\hbox{$#1$}\dimen0=\ht0 \advance\dimen0
  by1pt\rlap{\hbox to\wd0{\hss\raise\dimen0
  \hbox{\hskip.2em$\scriptscriptstyle\circ$}\hss}}#1\else {\accent"17 #1}\fi}
\begin{thebibliography}{1}

\bibitem{Chirikjian:2002}
Gregory~S. Chirikjian and Alexander~B. Kyatkin.
\newblock {\em Engineering Applications of Noncommutative Harmonic Analysis:
  With Emphasis on Rotation and Motion Groups}.
\newblock CRC Press, 2002.

\bibitem{ATLAS:1986}
J.~H. Conway, R.~T. Curtis, R.~A. Wilson, S.~P. Norton, and R.~A. Parker.
\newblock {\em ATLAS of Finite Groups}.
\newblock Oxford University Press, 1986.

\bibitem{dissonance}
William DeMeo.
\newblock Characterizing musical signals with {W}igner-{V}ille interferences.
\newblock In {\em Proceedings of the International Computer Music Conference}.
  ICMC, 2002.
\newblock Available from:
  \url{http://math.hawaii.edu/~williamdemeo/ICMC2002.pdf}.

\bibitem{nonabeliandsp}
William DeMeo.
\newblock Topics in nonabelian harmonic analysis and {DSP} applications.
\newblock In {\em Proceedings of the International Symposium on Musical
  Acoustics}. ISMA, 2004.
\newblock Available from:
  \url{http://math.hawaii.edu/~williamdemeo/ISMA2004.pdf}.

\bibitem{GAP4}
The GAP Group.
\newblock {\em {GAP -- Groups, Algorithms, and Programming, Ver.~4.4.12}},
  2008.
\newblock Available from: \url{http://www.gap-system.org}.

\bibitem{Tolimieri:1998}
Richard Tolimieri and Myoung An.
\newblock {\em Time-Frequency Representations}.
\newblock Birkh\"{a}user, Boston, 1998.

\bibitem{Tolimieri:2003}
Richard Tolimieri and Myoung An.
\newblock {\em Group Filters and Image Processing}.
\newblock Kluwer Acad., 2004.
\newblock Available from:
  \url{http://prometheus-us.com/asi/algebra2003/papers/tolimieri.pdf}.

\end{thebibliography}


\end{document}
