\documentclass[10pt]{article}
\usepackage[margin=1in]{geometry}
\usepackage{enumerate,amsmath,amssymb,fancyhdr,mathrsfs,amscd,stmaryrd,amsthm}
\usepackage{graphicx}
\usepackage{color}
\usepackage{tikz}
\usepackage{url}
\usepackage[colorlinks=true,urlcolor=black,linkcolor=black,citecolor=black]{hyperref}

\begin{document}
\begin{center}
{\bf What does a nonabelian group sound like?}\\
Matthew Corley (Computer Science); William DeMeo (Math; mentor), Reg
Bain (Music; co-mentor)
\end{center}

\noindent {\bf Background and prior work.}
%% Your proposal should consist of the following: 
%% 1. Background/Knowledge in the field/Literature review - Be succinct. This
%% section should provide the information that the reviewer needs to know to
%% understand what and why you are doing this project. Include a discussion of  the
%% present understanding and/or state of knowledge concerning the question/problem
%% or a discussion of the context of the scholarly or creative work. This section
%% presents and summarizes the problem you intend to solve.  This section should
%% include documentation, references, and a review of the literature that supports
%% the need for your research or creative endeavor. **For questions regarding works
%% cited, references, or bibliography, please see  8 below.
%% If your project is part of a larger project, how should this information be
%% included? It depends on what type of information you are including and how it
%% applies to your project. Information can be included in the background to
%% provide context and justification or as an introduction within the Project
%% design/methods section (or in both sections, if appropriate). 
In digital signal processing, the convolution function can be described as a sum
of weighted translations.  For normal DSP algorithms, this underlying function
is defined on the ``index set'' of the integers modulo $n$, with the group operation
addition. 

My faculty mentor, Dr.~William DeMeo (Mathematics), initiated research in this
direction in 2004 and presented his initial findings at the International
Symposium on Musical Acoustics. (His paper~\cite{nonabeliandsp} appeared in the
proceedings and won the conference's ``best paper award.'')

Similarly, the idea of indexing signals with nonabelian groups has been applied
to two dimensional image data with successful results (see~\cite{Tolimieri:2003}).\\
\\
See also \cite{Tolimieri:1998}.\\
\\
TODO: add more stuff here!
 
\vskip5mm

\noindent {\bf Research question.}
%% 2. Research question or statement – Very clearly state what you will be studying
%% in 1-2 sentences. The information on why, how, importance, etc will be in other
%% sections. Be sure that this is understandable to someone who does not know much
%% about your field of study. To test your explanation – give this to a friend not
%% in your major. If he/she doesn’t understand, try again! 
If the underlying index set of a digital signal processing algorithm is modified
to various nonabelian groups instead of the ordinary abelian groups, what
effects will this have on the audio output? 

\vskip5mm

\noindent {\bf Project goals and objectives.}
%% 3. Project Goal and Objectives - Goals and Objectives are often confused with
%% each other. They both describe things that a person may want to achieve or
%% attain but in relative terms may mean different things. Both are desired
%% outcomes of work done by a person but what sets them apart is the time frame,
%% attributes they are set for and the effect they inflict. Both terms imply the
%% target that one's efforts is desired to accomplish. (Review Section II of this
%% document for additional assistance with this section.)  
%%
%% Example: 
%% -- Goal: Our after-school program will help children read better. 
%% -- Objective: Our after-school remedial education program will assist 50
%%    children in improving their reading scores by one grade level as demonstrated on
%%    standardized reading tests administered after participating in the program for
%%    six months.  
%%
%% NOTE: Sections 2 and 3 are very important. They don’t need to be long – one
%% short paragraph should be enough – but they are critical. The rest of your
%% proposal supports these statements and explains why you want to explore this
%% question, how you will do it, and what it means to you. 
\begin{itemize}
\item {\it Goal:}
Develop mathematical theory to give sonic characterizations to nonabelian groups.
\item
{\it Objective:} Find a short list of nonabelian groups that are useful for
nonabelian group audio filters and effects processors, prove their effectiveness
both mathematically and experimentally, and document the results.
\item
{\it Objective:} Implement a software program that takes audio as input and
allows user to apply filters corresponding to specific nonabelian groups to
acheive different effects. 
\end{itemize}
 
\vskip5mm

\noindent {\bf Project significance.}
%% 4. Project impact, significance, or purpose - Keep the statement of significance
%% brief (2-4 sentences) - be succinct!  Some things to consider for this section:
%% what can your research be used for in the big picture; how is your research
%% innovative, unique or different; how will your project increase knowledge in the
%% field (is there a void that  your project will fill); what is the bigger
%% question that your question might help answer or how can it be used by others;
%% is there a direct impact to the community, environment, or USC. In thinking
%% about the significance, try to take the position of an educated newspaper
%% reader. If she or he were to see an article about your project in the paper, how
%% would you explain the importance or purpose of your project?  
The proposed research introduces the novel concept of giving sonic
characteristics to mathematical groups.  There is also a possibility of
applications in digital audio engineering, especially if certain nonabelian
groups are shown to produce interesting audio effects. 
 
\vskip5mm

\noindent {\bf Methodology.}
%% 5. Project Design or Methods - Design and describe a work plan consistent with
%% your academic discipline. This may include musical, creative, lab, or
%% field-based explorations; use of population samples; experimental and control
%% groups; statistical analysis; surveys or interviews; archival research;
%% translating; ethnographic fieldwork; case studies; or other forms of analysis
%% and synthesis of ideas and concepts. This section of the proposal should explain
%% the details of the proposed plan. How will you go about exploring your research
%% question? What are your methods? Who else will be involved and how? You can also
%% include a brief overview of what you (or others) have already done on the
%% project and/or what you will be doing after the project period in over, if your
%% project is of longer duration. 
%% 
%% a) Remember that literature reviews and understanding the current state of
%% knowledge in your field or topic of interest is an extremely important part of
%% ALL research projects. 
%% b) If your project involves people or animals, be sure to state in this section
%% that you are submitting for Animal Use/Human Subject approval and will comply
%% with all rules, regulations, and training requirements.   
%% 
%% Be specific on what you will be doing. The reasoning behind the Magellan program
%% is to make sure that you have a meaningful experience. If the reviewer can’t
%% tell what part of a project you will be doing, he/she can’t evaluate your
%% experience.  
%% 
%% Review Section III of this document for additional assistance with this section. 
%%
We will replace the index set $\mathbb{Z}/n$ with various finite nonabelian
groups, using the GAP library's SmallGroups and IsAbelian functions to find
appropriate groups to test. For each of the groups tested, we implement the
convolution function using the generalized (nonabelian) translation $f(xy^{-1})$
in place of oridinary translation $f(x-y)$ used in classical convolution. (Here,
the group multiplication is replacing the traditional addition modulo $n$ as the
group operation.  We are simply generalizing the underlying translation
operation, and the result is a generalized convolution.

Matlab will be used for much of the initial testing and prototyping, as it
provides easy methods for constructing WAV files "from scratch" with its
wavwrite() function. Additionally, Myoung An (a colleague of Dr.~DeMeo)
has made available to us the Matlab code she has developed for her work
in image processing, in which she and Richard Tolimieri applied nonabelian group
filters to the processing of digital images. This code will be a valuable
resource as we seek to apply similar ideas to audio signal processing.  

As the project progresses, we will likely use the JavaSound library and
implement our generalized DSP algorithms in Java.  JavaSound provides methods
for reading and altering wav files frequencies and sound intensity levels, which
will prove useful when we apply our generalized DSP algorithms to more complex
sounds. \\
\\
TODO: need more specifics here!


\vskip5mm

\noindent {\bf Project timeline.}\\
%% 6. Project timeline – Provide an overview of the timing for specific steps of
%% your project. This does not need to be a day to day list but depending on the
%% length of your project, it may give an overview biweekly or monthly. Be sure to
%% include time to review/synthesize your data or to reflect on the experience and
%% time to write the final report. This section can include a pre and post grant
%% period, if you have already started your project and/or plan to continue working
%% on this after the grant period ends. Review Section IV of this document for
%% additional assistance with this section. 
{\it October 2013--December 2013:} Become more familiar with current DSP algorithms
and gain further knowledge of group theory, and its role in classical DSP
implementations. 
\\[5pt]
{\it December 2013--March 2014:} Write code to implement DSP algorithms with a
selection of nonabelian groups and group operations.  
\\[5pt]
{\it January 2014--May 2014:} 
Identify specific characteristics of groups that make them more (or less)
useful as an index set on which to define DSP operations like convolution.
Gather and analyze results, and write up reports.  
\\[5pt]
{\it June 2014--October 2014:} Submit manuscript to an academic journal.  Prepare
for and attend conferences.

\vskip5mm

\noindent {\bf Anticipated results/Final Products and Dissemination.}
%% 7. Anticipated results/Final Products and Dissemination - Describe possible
%% forms of the final product, e.g., publishable manuscript, conference paper,
%% invention, software, exhibit, performance, etc. Be specific about how you intend
%% to share your results or project with others including names of possible
%% conferences or journals. This section may also include an interpretation and
%% explanation of results as related to your question; an analysis of the expected
%% impact of the scholarly or creative work on the audience; or a discussion on any
%% problems that could hinder your creative endeavor. Be sure to include your
%% Discovery Day presentation.  
By the end of the Spring 2014 semester, I expect to have written Matlab programs
to test the results of the modified DSP implementations described above.  I also
expect to have developed a Java software program which allows easy application
of nonabelian group filters through a graphical user interface.  I hope that the
results will prove interesting and have practical applications for computer
music composition.  

The abstract for this project has already been accepted for presentation at the
Joint Mathematics Meetings in Baltimore in 2014.  In addition, I will submit our
work to the International Computer Music Conference (ICMC) and International
Symposium on Musical Accoustics (ISMA), as well as the 14th International
Conference on New Interfaces for Musical Expression (NIME). Previous, similar
works by my faculty mentor Dr.~DeMeo have been accepted at both ICMC and ISMA
(see~\cite{nonabeliandsp} and~\cite{dissonance}), so we have high expectations
for this project. Furthermore, I will send in the formal paper to several
Mathematical and Music scholarly journals. Of course, I will also present the
results of our work at Discovery Day 2014.    \\
 \\
TODO: add something about Bain's interest/collaboration!

\vskip5mm

\noindent {\bf Personal statement.}
%% 8. Personal statement – This section is read carefully by the reviewers and does
%% impact their decision. Consider addressing: why you want to do this project,
%% what got you interested in it, your career goals, and how this project would
%% further those goals. While it is important, please remember that it should not
%% overpower the rest of the proposal. One-quarter of the page should be
%% sufficient.  
I have never had considerable ability with music, but I have always been
fascinated by its intersection with my favorite subject, mathematics.  This
project piqued my interest because it allows me the rare opportunity to
contribute to both fields.  I believe that I have developed the necessary skills
to succeed in this project through my past mathematical research projects and my
current internship developing Java applications at a local software company.  A
Magellan Grant would allow me the fiscal freedom to dedicate time to engaging in
research and possibly traveling to share my findings with international
audiences.  



%% 9. **Bibliography/References/Works cited – Use the standard convention of your
%% discipline including the author, title of article, journal title, volume, pages,
%% and date. References are not included in (are in addition to) the 2 page max.


%% \bibliographystyle{plainurl}
%% \bibliography{wjd}
\def\cprime{$'$} \def\cprime{$'$}
  \def\ocirc#1{\ifmmode\setbox0=\hbox{$#1$}\dimen0=\ht0 \advance\dimen0
  by1pt\rlap{\hbox to\wd0{\hss\raise\dimen0
  \hbox{\hskip.2em$\scriptscriptstyle\circ$}\hss}}#1\else {\accent"17 #1}\fi}
\begin{thebibliography}{1}

\bibitem{dissonance}
William DeMeo.
\newblock Characterizing musical signals with {W}igner-{V}ille interferences.
\newblock In {\em Proceedings of the International Computer Music Conference}.
  ICMC, 2002.
\newblock Available from:
  \url{http://math.hawaii.edu/~williamdemeo/ICMC2002.pdf}.

\bibitem{nonabeliandsp}
William DeMeo.
\newblock Topics in nonabelian harmonic analysis and {DSP} applications.
\newblock In {\em Proceedings of the International Symposium on Musical
  Acoustics}. ISMA, 2004.
\newblock Available from:
  \url{http://math.hawaii.edu/~williamdemeo/ISMA2004.pdf}.

\bibitem{Tolimieri:1998}
Richard Tolimieri and Myoung An.
\newblock {\em Time-Frequency Representations}.
\newblock Birkh\"{a}user, Boston, 1998.

\bibitem{Tolimieri:2003}
Richard Tolimieri and Myoung An.
\newblock {\em Group Filters and Image Processing}.
\newblock Kluwer Acad., 2004.
\newblock Available from:
  \url{http://prometheus-us.com/asi/algebra2003/papers/tolimieri.pdf}.

\end{thebibliography}


\end{document}
