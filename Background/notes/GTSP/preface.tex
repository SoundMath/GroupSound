%%%%%%%%%%%%%%%%%%%%%% pref.tex %%%%%%%%%%%%%%%%%%%%%%%%%%%%%%%%%%%%%
%
% sample preface
%
% Use this file as a template for your own input.
%
%%%%%%%%%%%%%%%%%%%%%%%% Springer-Verlag %%%%%%%%%%%%%%%%%%%%%%%%%%

\preface

%\begin{quote}
%``The [math] literature would have you believe that essence
%  precedes existence. In my experience, the reverse holds.''
%\end{quote}

%If ``existence precedes essence'' is the mark of existentialism, then an
%algebraist does not exist.
%one cannot be the pure mathematician cannot be 

This document is divided into two parts: Part \ref{part:theory} is a compendium 
of mathematics that I find interesting and useful for working in the field know
as \ac{DSP}.  
This serves to introduce a powerful, if somewhat abstract, cast
of characters from the fields of group theory and linear algebra.
%, but only insofar as it reveals the essence of each character.
Part~\ref{part:applications} then brings these characters down to earth by
giving them roles to play in the ``real world.''  
%The leading role is that of signal processor.  

At various places in the book -- especially during the more abstract lines of
reasoning -- the term \emph{digital signal processing} may seem out-of-place,
but, as the reader will see, an underlying theme of this work is the
importance of interpreting each word in this phrase at whatever level of
abstraction is most appropriate to the given context.  Typically,
%and in what follows
a ``signal'' is a function or process of interest, or data
which represent this function or process.  The term ``processing'' refers to
understanding, analysis, synthesis, manipulation, \etc of the signal.  In this
work, the term ``digital'' merely indicates that the focus in on math tools and
methods that are particularly well suited to functions that have discrete representations.    

Each application chapter of Part~\ref{part:applications} describes a research area or
problem of practical interest from within the field of \ac{DSP}, and
then demonstrates how the mathematical concepts from Part~\ref{part:theory} 
manifest in elegant and revealing mathematical expressions of the problem.
The approach provides powerful means of understanding and analysis 
which are difficult, if not impossible, with other approaches.

For the reader's part, becoming well acquainted with the theory in
Part~\ref{part:theory} can demand significant investments of time, patience, and
diligence.  However, as noted above, the rewards are substantial.
 %% and, as implied above, they include greater capacity for analytical
 %%clarity and deep insight. 
%To get an idea of the processing power and utility of these mathematical
%methods, the reader may wish to look first at the application chapters.  
%Of course, we present the theory in Part~\ref{part:theory} before 
%applying it in Part~\ref{part:applications}. Whether this ordering has some deep,
%philosophical significance is left as an exercise.%\footnote{{\it Hint:} consider
%%%  the quote above.} 

%% Please "sign" your preface
\vspace{1cm}
\begin{flushright}\noindent
Kihei, Hawaii \hfill {\it William  DeMeo}\\
August 2003\hfill \\
\end{flushright}


