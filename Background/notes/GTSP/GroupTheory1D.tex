%% LaTeX file 'GroupTheory1D.tex'

%% latexfile{
%% author = {William DeMeo},
%% filename = {GroupTheory1D.tex},
%% date = {2002.08.07},
%% text = {Main latex file for Group Theory for One-dimensional Signals chapter of dspmath book.}
%% }

%%% BEGIN FILE INSERT: TFR.tex

\title{Abelian Groups and Univariate DSP}
\titlerunning{Abelian Groups \& Univariate DSP}
\author{Richard Tolimieri \and Myoung An\inst{1}\and William DeMeo\inst{2}}
\institute{{\tt psypher@tiac.net} \and {\tt williamdemeo@yahoo.com}}
\authorrunning{Tolimieri, An, DeMeo}
\maketitle

\newcommand\sectype{chapter}
\newcommand\chznz{\ensuremath{(\mathbb{Z} / N )^*}}

%The main source of most material in this \sectype\ is
%Tolimieri and An, {\it Time Frequency Representations},\cite{Tolimieri:1998}.

Although this \sectype\ assumes some basic group theory,
such as that reviewed in Chapter~\ref{sec:abstr-algebr}, the
following section reviews those prerequisites that arise
most frequently in our application of the theory. Further
details can be found in \cite{Fraleigh:1994}, and Chapters~1
and~2 of~\cite{Tolimieri:1998}. 

\section{Preliminaries}

Throughout, unless otherwise stated, $A$ denotes an abelian group and
$\LA$ is the space of all complex valued functions on $A$.

\ifthenelse{\boolean{nospecialfootnotes}}
{\subsection{Translation and Convolution}}
{\subsection{Translation and Convolution\protect\footnotemark}
\footnotetext{op.~cit.~\cite{Tolimieri:1998}, \S\ 3.7 and \cite{An:2003},\S\ 1.1.}}
\label{sec:trans-conv}
%%% left off here %%%
\begin{definition}[Unitary Transform, Isometry]\\
A linear mapping $\T: \LA \to \LA$ is called \emph{unitary} if, for all
$f, g \in \LA$,
\[
(\T f,\T g) = (f,g)
\]
More generally, a linear mapping $\tau: \vs{L}(A_1) \to \vs{L}(A_2)$ is called
an \emph{isometry} if, for all $f, g \in \vs{L}(A_1)$, 
\[
(\tau f,\tau g) = (f,g)
\]
where the inner product $(f,g)$ is taken in $\vs{L}(A_1)$ and the inner product
$(\tau f,\tau g)$ is taken in $\vs{L}(A_2)$.
\end{definition}
Many important linear mappings are isometries.
\begin{theorem}\label{thm:unitary}
A linear mapping $\T$ is unitary if and only if $\T$ maps an orthonormal basis
onto an orthonormal basis.
\end{theorem}

\begin{definition}[Translation]\label{def:translation}\\
For $y\in A$, the mapping $\T(y)$ of $\LA$ defined by
%For $f\in \LA$ and $y\in A$, the mapping $\T_y: \LA \to \LA$ defined by 
\[
(\T(y)f)(x) = f(y^{-1}x), \qquad f \in \LA, x \in A
\]
is a linear operator of $\LA$ called \emph{translation by} $y$.  
\end{definition}
In the interest of generality, Definition~\ref{def:translation} denotes the 
inverse of $y\in A$ by $y^{-1}$.  Often translation defined for the special,
additive inverse case, $y^{-1}x = x-y$.

Denote the collection of all translations of $\LA$ by 
$\T(A) = \{\T(y) : y\in A\}$.  The mapping $\T :A \to \T(A)$
satisfies 
\[
\T(x+y) = \T(x)\T(y), \qquad x,y \in A,
\]
\[
\T(y)^{-1} = \T(y^{-1}), \qquad y \in A
\]
%where $y^{-1}$ is interpreted here as additive inverse.  
The collection $\T(A)$ is closed under operator composition and operator inverse.  
%Since addition modulo $N$ is commutative, 
Since $A$ is abelian, and addition commutative, 
$\T(A)$ is a {\it commuting family of linear operators}.
%; that is,\[\T(x)\T(y) = \T(y)\T(x)\]
Since $A$ is finite, $y\in A$ implies $Ny = 0$, and
\[
\T(y)^N =  \T(Ny) =  \T(0) = \eye
\]
where $\eye$ is the identity operator of $\LA$.

\begin{definition}[Convolution]\\
For $g \in \LA$, the mapping $\lt{C}(g)$ of $\LA$ defined by 
\begin{equation}\label{eq:convolution}
\lt{C}(g) = \sum_{y\in A} g(y)\T(y)
\end{equation}
is a linear operator of $\LA$ called \emph{convolution by} $g$.  
\end{definition}
The summation~(\ref{eq:convolution}) is a linear combination of translations of
$\LA$ using operator scalar multiplication and operator addition. 

Often the convolution $\lt{C}(g)f$ is written as a binary operation, $f*g$, called
the \emph{convolution product} on \LA.  By definition, 
the convolution product of $f$ and $g$, evaluated at $x\in A$, is 
\begin{align*}
(\lt{C}(f)g)(x) = (f * g)(x) &= \sum_{y\in A} f(y)g(y^{-1}x)\\
&=\sum_{y\in A} g(y)f(xy^{-1})\qquad (\text{change of variables})\\
&=(\lt{C}(g)f)(x) = (g * f)(x)\qquad (\because \; A\text{ is abelian})\\
\end{align*}
\begin{theorem} Convolution is a commutative algebra product on \LA.
\end{theorem}
\begin{definition}[Convolution Algebra over $A$]\\
The algebra formed by \LA paired with the convolution product is called the
\emph{convolution algebra over $A$}.
\end{definition}

\subsection{Factor Groups}
Many of our examples will focus on the group
\[
N\Z = \{\ldots,-2N, -N,0,N,2N,\ldots\}, \qquad N \in \Z
\]
as well as the {\it factor group},
\begin{equation}\label{eq:5}
\Z/N\Z = \{0+N\Z, \, 1+N\Z, \ldots,\,(N-1)+N\Z \}
\end{equation}
The notation $k+N\Z$ in~(\ref{eq:5}) denotes the group
\[
k+N\Z = \{\ldots,k-2N,\, k-N,\, k,\, k+N,\, k+2N \ldots\}
\]
Thus a factor group, such as $\Z/N\Z$, is really a ``group of groups'', or
a group of {\it equivalence classes}. 
By choosing one element from each equivalence class as a {\it representative}
of that class, we arrive at an isomorphism between the factor group and the
group of all equivalence class representatives:
\begin{equation}\label{eq:T8}
\Z/N\Z \simeq \{0, 1, 2, \ldots, (N-1)\}
\end{equation}
The right hand side of (\ref{eq:T8}) is often denoted by $\Z_N$, and the left
hand side by $\Z/N$. In the sequel, we abuse this notation and simply refer
to the equivalence as $\Z/N = \{0, 1, 2, \ldots, (N-1)\}$.

We also make heavy use of the group $L\Z/N\Z$, where $L$ is a divisor of $N$,
say $LM=N, \; M \in \Z$.  
%%%% CHECK THIS FACT:  %%%%
%This implies that $N\Z$ is a subgroup of $L\Z$, sometimes denoted $N\Z < L\Z$.  
Recall that, for such an $L$, the factor group $L\Z/N\Z$ is
\begin{align}\label{eqn:LZNZ}
L\Z/N\Z &= \{0+N\Z,\,L+N\Z,\,2L+N\Z,\ldots,\,(M-1)L+N\Z\}\nonumber \\
        &\simeq \{0,\,L,\,2L,\ldots,(M-1)L\}, \\
        &\simeq \{0, 1, 2,\ldots,(M-1)\} = \Z_M \nonumber 
\end{align}
Again, abusing notation, we will write equivalence (\ref{eqn:LZNZ}) as 
$L\Z/N = \{0,\,L,\,2L,\ldots,(M-1)L\}$.

Finally, recall that
\begin{equation}\label{eqn:ZL}
(\Z/N)/(L\Z/N) \simeq  \Z/L \simeq \{0,1, \ldots, (L-1)\}
\end{equation}
which we also write $\Z/L = \{0,1, \ldots, (L-1)\}$, as remarked above.

\begin{example} \label{ex:factor-groups}
Consider the group
\[
L\Z/N \simeq \{0,L,2L,\ldots,(M-1)L\}, \quad \text{ for } LM = N,
\]
and suppose $f\in \vs{L}(\Z/N)$ is $L\Z/N$-periodic.
Then a vector representation of $f$ is
\[
\vec{f} = 
\begin{pmatrix}%\left(\begin{array}{c}
\vec{f}^{(0)}\\
\vec{f}^{(0)}\\
\vdots\\
\vec{f}^{(0)}
\end{pmatrix},%\end{array}\right),
 \quad \text{ where } \quad
\vec{f}^{(0)} =
\begin{pmatrix}%\left(\begin{array}{c}
f(0)\\
f(1)\\
\vdots\\
f(L-1)
\end{pmatrix}%\end{array}\right)
\]
That is $\vec{f}\in \C^N$ is a vector containing $M = N/L$ sub-vectors,
$\vec{f}^{(0)}\in \C^L$.
\end{example}

\begin{remark}
The group $\Z/N_1 \times \Z/N_2$ is useful for image processing
applications.  Note that the foregoing definitions of translation and
convolution are general enough to apply to functions in 
$\vs{L}(\Z/N_1 \times \Z/N_2)$, so long as we have a suitable definition for
addition and additive inverse on $\Z/N_1 \times \Z/N_2$.  
\end{remark}
%\begin{example}
%For $y\in \Z/N$, and $g,f \in \vs{L}(\Z/N)$,
%\[(T(y)f)(x) = f(x-y), \qquad x \in \Z/N\]\[\lt{C}(g) = \sum_{y\in \Z/N} g(y)T(y)\]
%and, evaluated at any particular $x \in \Z/N$, the convolution by $g$ of $f$ is 
%\[(\lt{C}(g)f)(x) = \sum_{y\in \Z/N} g(y)f(x-y), \qquad \]
%\end{example}

\begin{definition}[Canonical Basis]\\
For $y\in \Z/N$, define $e_y \in \vs{L}(\Z/N)$ as the function whose value is 1 at
$y$ and 0 otherwise.  The set $\{ e_y : y\in \Z/N\}$ is called the
\emph{canonical basis}.
\end{definition}
Translations permute the elements of the canonical basis by the equation
$\T(y)e_x = e_{y+x}$, where $y+x$ is taken modulo $N$.  
In particular, $\T(y)$ maps the evaluation basis onto the
evaluation basis.  As such, by Theorem~\ref{thm:unitary}, translations
are unitary operators.

\section{Fourier Analysis on Finite Abelian Groups}
\label{sec:four-analys-over}
\ifthenelse{\boolean{nospecialfootnotes}}
{\subsection{Character Groups}}
{\subsection{Character Groups\protect\footnotemark}
\footnotetext{op.~cit.~\cite{Tolimieri:1998}, \S\ 3.2.}}
Suppose $A$ is a finite abelian group of order $N$.  Denote by $U$ the
multiplicative group of all complex numbers of absolute value 1.
\begin{definition}[Character]
A mapping $a^*: A \to U$ is called a \emph{character} of $A$ if it is
a homomorphism; that is, if the mapping satisfies, 
\[
a^*(a+b) = a^*(a)a^*(b), \qquad a, b \in A.
\]
\end{definition}
We often denote $a^*(a)$ by $\langle a, a^* \rangle$.
Every character of $A$ maps $A$ into the subgroup $U_N$ of all
$N$-complex roots of unity.
Denote by $A^*$ the set of all characters of $A$.  The set $A^*$ is an abelian
group, called the \emph{character group} of $A$, under the composition law,
\[
\langle a, a^* + b^* \rangle = \langle a, a^* \rangle \langle a, b^* \rangle,
\qquad a \in A, \; a^*, b^* \in A^*
\]
The mapping $0^*$ of $A^*$ defined by
\[ 
0^*(a) = \langle a, 0^* \rangle = 1, \qquad a \in A,
\]
is the \emph{zero} element in the group $A^*$.  
Since, for any $a^*\in A^*$,
\[
\langle a, a^* \rangle \overline{\langle a, a^* \rangle}= 1, \qquad a\in A,
\]
the additive inverse of $a^*$ in $A^*$ is given by 
\[
\langle a, -a^* \rangle  = \overline{\langle a, a^* \rangle}, \qquad a\in A.
\]


\begin{example}\label{ex:std-pres}
Consider the pairing
\[ 
\langle a, c \rangle = \exp(\I 2\pi \frac{ca}{N}), \qquad a,c \in \Z /N
\]
For $c\in \Z /N$, we define the mapping $\chi(c): \Z /N \to U_N$
by the following equivalent expressions
\begin{equation}\label{eq:2}
\chi(c)(a) = \langle a, \chi(c)\rangle = \langle a, c \rangle 
= \exp(\I 2\pi \frac{ca}{N})
\end{equation}
Thus, for all $c\in \Z/N$, the mapping $\chi(c)$ is a character of $\Z /N$,
and the mapping $\chi: \Z /N \to (\Z /N )^*$ is an isomorphism
of $\Z /N$ onto its character group $(\Z /N )^*$.
\end{example}

Any isomorphism from a finite abelian group $A$ onto its character group $A^*$
is called a \emph{presentation} of $A$.  The presentation defined in
example~\ref{ex:std-pres} is called the \emph{standard presentation} of
$\Z/N$. 

\ifthenelse{\boolean{nospecialfootnotes}}
{\subsection{Character Formulas}}
{\subsection{Character Formulas\protect\footnotemark}
\footnotetext{op.~cit.~\cite{Tolimieri:1998}, \S\ 3.3.}}
\label{sec:char-form}
\begin{example}
Suppose $A = \Z/N$ and $N$ is even.  Then the $N$-roots of unity contain $-1$
and are written as
\[
1, v, \ldots, v^{M-1}, -1, -v,\ldots, -v^{M-1}, \qquad N=2M, \; v=\exp(\frac{\I 2\pi}{N})
\]
implying that the sum of the $N$-roots of unity vanishes for $N$ even.  For
$N$ odd, this fact is still true but harder to show.
\end{example}

\begin{theorem}\label{thm:SumOfCharacters}
For $a^*\in A^*$,
\[
\sum_{a\in A} \langle a,a^*\rangle 
= \left\{ \begin{array}{ll}
N, & \quad a^* = 0^*,\\
0, & \quad \text{otherwise}.
\end{array} \right.
\]
\end{theorem}
\begin{proof}
If $a^* = 0^*$, then $\langle a, a^* \rangle=1$ for all $a\in A$, so the
result is obvious.  Suppose $a^* \neq 0^*$.  Then there exists an $a_0 \in A$
such that $\langle a_0, a^* \rangle \neq 1$.  Also, since $A$ is a group,
summing over all $a\in A$ is the same as summing over all $a+a_0 \in A$.
Therefore,
\begin{align*} 
\sum_{a \in A}\langle a, a^* \rangle &= \sum_{a \in A}\langle a + a_0, a^* \rangle \\
&=\langle a_0, a^* \rangle \sum_{a \in A}\langle a, a^* \rangle
\end{align*}
Since $\langle a_0, a^* \rangle \neq 1$, the sum must be 0. \qed
\end{proof}
Theorem~\ref{thm:SumOfCharacters} implies, for example, that the
$N$-complex roots of unity sum to 0.
\begin{theorem}\label{thm:SumOfCharacters-dual}
For $a\in A$,
\[
\sum_{a^*\in A^*} \langle a,a^*\rangle 
= \left\{ \begin{array}{ll}
N, & \quad a = 0,\\
0, & \quad \text{otherwise}.
\end{array} \right.
\]
\end{theorem}
The proof of Theorem~\ref{thm:SumOfCharacters-dual} is the same as that for
Theorem~\ref{thm:SumOfCharacters}, {\it mutatis mutandis}.  A simple but
important corollary is 
\begin{corollary}\label{cor:zeroElement}
For $a\in A$, if for all $a^*\in A^*$,
\[
\langle a, a^* \rangle = 1
\]
then $a=0$.
\end{corollary}

\ifthenelse{\boolean{nospecialfootnotes}}
{\subsection{Duality Theory}}
{\subsection{Duality Theory\protect\footnotemark}
\footnotetext{op.~cit.~\cite{Tolimieri:1998}, \S\ 3.4.}}
Denote by $A^{**}$ the character group of $A^*$.  The groups $A$ and
$A^{**}$ are canonically isomorphic, and this permits $A^{**}$ to be
replaced by $A$ in all discussions and results.  We call this replacement
\emph{duality}. 
%\begin{definition}[Character]
For $a\in A$, the mapping 
\[
\varTheta(a): A^* \to U
\]
defined by 
\[
\varTheta(a)(a^*) = \overline{\langle a, a^* \rangle}, \qquad a^* \in A^*,
\]
is a character of $A^*$.
%\end{definition}
\begin{theorem}
The mapping 
\[
\varTheta: A \to A^{**}
\] 
is an isomorphism of $A$ onto $A^{**}$.
\end{theorem}
\begin{proof} If $\varTheta(a)$ is the trivial character of $A^*$, then
\[
\langle a, a^* \rangle = 1, \text{  for all } a^* \in A^*,
\]
So, by Corollary \ref{cor:zeroElement}, $a = 0$. \qed
\end{proof}

We identify $A$ with $A^{**}$ by the canonical isomorphism $\varTheta$ and denote
by $a$ the character $\varTheta(a)$ of $A^*$, for $a\in A$.  It will be clear
from context whether $a$ is being viewed as an element of $A$ or an element of
$A^{**}$.

\begin{definition}[Dual]\\
For a subgroup $B$ of $A$, the set
\[
B_* = \{ a^* \in A^* : \langle b, a^* \rangle =1, \text{ for all } b \in B\}
\]
is a subgroup of $A^*$ called the \emph{dual} of $B$.  
\end{definition}
The subgroup $B_*$ of $A^*$ is the set of all characters of $A$ that act
trivially on $B$.  For $y^* \in B_*, a\in A$, and $b\in B$,
\[
y^*(a+b) = \langle a+b, y^* \rangle = 
\langle a, y^* \rangle  \langle b, y^* \rangle =
\langle a, y^* \rangle  
\]
Therefore, $y^*$ defines a character of $ A/B = \{a + B: a\in A\}$.
We denote this character by $\utheta(y^*)\in (A/B)^*$.  It is defined by the
formula,
\[
\utheta(y^*)(a+B) = \langle a+B, \utheta(y^*) \rangle = 
\langle a, y^* \rangle, \qquad a \in A
\]
\begin{exercise}
Consider the subgroup $L\Z/N$, of $\Z/N$, for $N = LM$.
Identifying $\Z/N$ with $(\Z/N)^*$ by the standard
presentation, show 
that\footnote{cf.~typo on page 32 of \cite{Tolimieri:1998}.} 
\[
(L\Z/N)_* = (M\Z/N)^*
\]
That is, show that the \emph{dual} of $L\Z/N$ is 
$(M\Z/N)^*$.
Or, to put it another way, show that (the standard presentation of) the set of
characters that act trivially on $L\Z/N$ is 
\[
M\Z/N = \{0,\,M,\,2M,\ldots,(L-1)M\}
\]
\end{exercise}
\begin{solution}
Recall from (\ref{eqn:LZNZ}),
\begin{equation}\label{eq:1}
L\Z/N  %=\left\{\{0+N\},\,\{L+N\},\,\{2L+N\},\ldots,\{(M-1)L+N\}\right\}
= \{0,\,L,\,2L,\ldots,(M-1)L\}%\\&\simeq \{0,\,1,\,2,\ldots,(M-1)\}\nonumber
\end{equation}
Also, recall the definition of the dual; \ie the set of characters that act
trivially on $L\Z/N$:
\[
(L\Z/N)_* = \{a^*\in (\Z/N)^* : \langle b, a^* \rangle =1, \text{ for all }
b \in L\Z/N \}
\]
Expression~(\ref{eq:1}) makes clear that any element $b\in L\Z/N$ can be
written $yL$ for some  
$y \in \{0, 1, 2, \ldots, (M-1)\}$.  Thus,
\[
\langle b, a^* \rangle = \exp(\I 2\pi\frac{ayL}{N})
= \exp(\I 2\pi\frac{ay}{M}), \qquad b \in L\Z/N
\]
Now suppose $a^* \in (M\Z/N)^*$, so that $a^* = xM$, for some 
$x \in \{0,1,2,\ldots,(L-1)\}$.  Then 
\[
\langle b, a^* \rangle = \exp(\I 2\pi xy) = 1, \qquad b \in L\Z/N
\]
Therefore, the elements of $(M\Z/N)^*$ act trivially on $L\Z/N$.  

Next suppose $a^* \notin (M\Z/N)^*$, so that $a^*$ is not a multiple of $M$, and let
$b = L$.  Then, 
\[
\langle b, a^* \rangle 
= \exp(\I 2\pi\frac{a}{M}) \neq 1
\]
We have thus shown that $a^* \in (L\Z/N)_*$ if and only if $a^* \in
(M\Z/N)^*$, {\it q.e.d.}
\end{solution}

%% [2002.09.02] more new stuff (p.32 of Tolimieri) %%
\begin{theorem}
The mapping 
\[
\utheta: B_* \to (A/B)^*
\]
is a canonical isomorphism of $B_*$ onto $(A/B)^*$.
\end{theorem}
For proof, see \cite{Tolimieri:1998}, page 32.

The identification of $B_*$ with $(A/B)^*$, and $B$ with $(A^*/B_*)^*$ results
in the following refinements of the character formulas.
\begin{corollary}\label{cor:char1}
For $x\in A$,
\[
\sum_{x^*\in B_*} \langle x, x^* \rangle = 
\left\{\begin{array}{ll}
o(B_*),& \quad x \in B,\\
0,& \quad \text{otherwise}
\end{array}
\right.
\]
\end{corollary}
\begin{corollary}\label{cor:char2}
For $x^*\in A^*$,
\[
\sum_{x\in B} \langle x, x^* \rangle = 
\left\{\begin{array}{ll}
o(B),& \quad x^* \in B_*,\\
0,& \quad \text{otherwise}
\end{array}
\right.
\]
\end{corollary}

\ifthenelse{\boolean{nospecialfootnotes}}
{\subsection{Character Group Basis}}
{\subsection{Character Group Basis\protect\footnotemark}
\footnotetext{op.~cit.~\cite{Tolimieri:1998}, \S\ 3.5.}}
Consider the character group $A^*$ as a subset of $\LA$, the space of all
complex valued functions on $A$.  In this section we show that $A^*$ is an
orthogonal basis of $\LA$.
\begin{example}
Consider the group $\Z/N$ and the standard presentation $\chi$.  Recalling the
defining equation~(\ref{eq:2}), the inner product  in $\vs{L}(\Z/N)$ of two
characters $\chi(a)$ and $\chi(b)$ is given by  
%\begin{align*}
\[
%(\chi(b),\chi(c)) &= \sum_{a\in A} \chi(b)(a) \overline{\chi(c)(a)} \\
(\chi(b),\chi(c)) = \sum_{a\in A} \langle a,\chi(b) \rangle \overline{\langle a,\chi(c) \rangle}
= \sum_{a\in A} \exp (2\pi \I \frac{a(b-c)}{N}),
\qquad b,\, c \in \Z/N
%\end{align*}
\]
The sum vanishes unless $b=c$, in which case the sum is equal to $N$.  
\end{example}
In general, we have the following result:
\begin{theorem}
$A^*$ is an orthogonal basis of $\LA$.
\end{theorem}
\begin{proof}
Orthogonality follows from Theorem~\ref{thm:SumOfCharacters} and 
\[
(a^*,c^*) = \sum_{a \in A} \langle a, a^* \rangle \overline{\langle a, c^*
  \rangle}
= \sum_{a \in A} \langle a, a^* - c^* \rangle, \qquad a^*,\, c^* \in A^*
\]
Arguing by dimension completes the proof. \qed
\end{proof}
\begin{corollary}
\[
\frac{1}{\sqrt{N}} A^*
\]
is an orthonormal basis of $\LA$. 
\end{corollary}
Technically, $A^*$ is a basis only if some ordering is placed on $A^*$.
In Fourier theory the distinction is significant since there is no canonical
ordering on $A^*$.  For the development of the theory, we use summation
notation and order is no problem. However, in order to represent algorithms in
terms of matrix factorization, some order must be taken, usually by
realization.  The form of the realization can affect the size and dimension of
the algorithm. 

\ifthenelse{\boolean{nospecialfootnotes}}
{\subsection{Fourier Transform}}
{\subsection{Fourier Transform\protect\footnotemark}
\footnotetext{op.~cit.~\cite{Tolimieri:1998}, \S\ 3.6.}}
\begin{definition}[Fourier Expansion, Fourier Coefficient Set]\\
Suppose $f\in \LA$. The expansion of $f$ over the character group basis
$A^*$ given by
\begin{equation}\label{eq:FE}
f = \sum_{a^*\in A^*} \alpha(a^*)a^*
\end{equation}
is called the \emph{Fourier expansion} of $f$ and the coefficient set 
\[
\alpha \in \vs{L}(A^*)
\]
is called the \emph{Fourier coefficient set}.
\end{definition}
Note that expression~(\ref{eq:FE}), evaluated at a point $a\in A$, is
\begin{equation}\label{eq:FEpt}
f(a) = \sum_{a^*\in A^*} \alpha(a^*)a^*(a)
= \sum_{a^*\in A^*} \alpha(a^*)\langle a,a^*\rangle 
\end{equation}
\begin{theorem}\label{thm:Fcoef}
If $f\in \LA$ has Fourier coefficient set $\alpha \in \vs{L}(A^*)$, then for all
$a^*\in A^*$,
\begin{equation}\label{eq:Fcoef}
\alpha(a^*) = \frac{1}{N} (f,a^*)
\end{equation}
\end{theorem}
\begin{proof}
For $c^* \in A^*$,
\begin{align*}
(f,c^*)&= \sum_{a\in A}f(a)\overline{\langle a,c^*\rangle} \\
&= \sum_{a\in A}\sum_{a^*\in A^*} \alpha(a^*)\langle a,a^*\rangle
\overline{\langle a,c^*\rangle}, \quad
\text{ by equation~(\ref{eq:FEpt})}\\
&= \sum_{a^*\in A^*} \alpha(a^*)\sum_{a\in A} \langle a,a^*-c^*\rangle\\
&= \alpha(c^*)N
\end{align*}
The final equality holds because, recall,
\[
\sum_{a\in A} \langle a,a^*-c^*\rangle = 
\left\{\begin{array}{ll}
N,&\quad \text{ for } a^* - c^* = 0^*\\
0,&\quad \text{otherwise}
\end{array}\right .
\]
\qed
\end{proof}
Note that expression~(\ref{eq:Fcoef}) is equivalent to 
%, evaluated at a point $a^*\in A^*$, is
\[
\alpha(a^*) = \frac{1}{N}\sum_{a\in A} f(a)\overline{\langle a,a^*\rangle},
\]
%\def\caoN{{\scriptstyle \frac{ca}{N}}}
\begin{example}\label{ex:FT}
Consider the group $\Z/N$ identified with its character group by the standard
presentation.  If $f\in \vs{L}(\Z/N)$ has Fourier coefficient set $\alpha \in
\vs{L}(\Z/N)$, then 
\[
\alpha(c) = \frac{1}{N}\sum_{a=0}^{N-1} f(a)
\exp(-2\pi \I \frac{ca}{N}),\qquad c\in \Z/N
\]
Denoting by $\vec{f} \in \C^N$ the vector representation of $f$ and by
$F(N)$ the $N$-point Fourier transform matrix,
\[
%F(N) = \left\{ \exp(-2\pi \I \frac{ca}{N}),\qquad c\in \Z/N
F(N) = \left\{ \exp(-2\pi \I \frac{ca}{N}) \right\}_{0\leq c,a < N}
\]
the Fourier coefficient set $\alpha(c), c\in \Z/N$, is given by the matrix
product
\[
\alpha = \frac{1}{N} F(N) \vec{f}
\]
\end{example}

\begin{example}\label{ex:FT2}
Consider the group $\Z/N_1 \times \Z/N_2$ identified with its character group
by the standard presentation.  If 
\[
f\in \vs{L}(\Z/N_1 \times \Z/N_2), \qquad N = N_1 N_2,
\]
has Fourier coefficient set $\alpha \in \vs{L}(\Z/N_1 \times \Z/N_2)$, then for all
$c = (c_1,c_2) \in \Z/N_1 \times \Z/N_2$, 
\[
\alpha(c) = \frac{1}{N}\sum_{a_1=0}^{N_1-1} 
\sum_{a_2=0}^{N_2-1} f(a_1,a_2)
\exp(-2\pi \I \frac{c_1a_1}{N_1})
\exp(-2\pi \I \frac{c_2a_2}{N_2})
\]
Denote the matrices
\[
\left[f(a_1,a_2) \right]_{\stackrel{0\leq a_1 < N_1}{_{0\leq a_2 < N_2}}} 
\quad \text{ and }\quad
\left[\alpha(c_1,c_2) \right]_{\stackrel{0\leq c_1 < N_1}{_{0\leq c_2 < N_2}}}
\]
by $Mat(f)$ and $Mat(\alpha)$, respectively.
Then the relationship between $f$ and $\alpha$ can be written as 
\[
Mat(\alpha) = \frac{1}{N} F(N_1) Mat(f) F(N_2)
\]
\end{example}

\begin{definition}[Fourier Transform]\\
The \emph{Fourier transform} $F_A$ over $A$ is the linear mapping
$F_A : \LA \to \vs{L}(A^*)$
defined by
\[
F_A f(a^*) = (f,a^*) = \sum_{a\in A}f(a)\overline{\langle a, a^* \rangle},
\qquad f \in \LA, \;a^*\in A^*
\]
\end{definition}
The mapping $F_A$, up to a scalar factor $1/N$, maps $f$ onto the coefficient
set of its Fourier expansion over $A$.  

By duality,  the Fourier transform
$F_{A^*}$ over $A^*$ is the linear mapping $F_{A^*} :  \vs{L}(A^*) \to \LA$
defined by
\[
F_{A^*}\alpha(a) = \sum_{a^* \in A^*}\alpha(a^*)\langle a, a^*\rangle,
\qquad \alpha \in \vs{L}(A^*), \; a\in A
\]
The composition $F_{A^*}F_A$ is a linear mapping from $\LA$ to $\LA$.
\begin{theorem}
\[
F_{A^*}F_A = N I_A
\]
where $I_A$ is the identity mapping on $\LA$.
\end{theorem}
\begin{proof}
The line of argument is the same as that of Theorem~\ref{thm:Fcoef}, {\it mutatis mutandis}.
For $a \in A$,
\begin{align*}
F_{A^*}F_A f(a)&= \sum_{a^*\in A^*}F_A f(a^*)\langle a,a^*\rangle \\
&= \sum_{a^*\in A^*}\sum_{c\in A} f(c)
\overline{\langle c,a^*\rangle}\langle a,a^*\rangle\\
&= \sum_{c\in A} f(c)\sum_{a^*\in A^*}
\overline{\langle c-a,a^*\rangle}\\
&= f(a)N
\end{align*}
\qed
\end{proof}


\ifthenelse{\boolean{nospecialfootnotes}}
{\section{Poisson Summation Formula}}
{\section{Poisson Summation Formula\protect\footnotemark}
\footnotetext{op.~cit.~\cite{Tolimieri:1998}, Chapter 4.}}
The Poisson summation (PS) formula describes the fundamental duality between
periodization and decimation operators under the Fourier transform.  In this
section, the finite abelian group version of the PS formula is derived as a
simple application of the character formulas of Section~\ref{sec:char-form}.

In sampling applications, the main use for the PS formula is to provide a
procedure for preprocessing a signal for the Fourier transform and other
computations by establishing a Nyquist sampling rate.  In FFT algorithm
design, it is the first step in a divide-and-conquer strategy.

The text~\cite{Tolimieri:1998} -- the source for these notes -- uses
the PS formula to unify basic algorithmic procedures in time-frequency
processing and to derive closed form formulas that play a significant role in
\emph{Weyl-Heisenberg duality theory}, which is the subject of Chapter 13
of~\cite{Tolimieri:1998}. 

\ifthenelse{\boolean{nospecialfootnotes}}
{\subsection{Statement and proof}}
{\subsection{Statement and proof\protect\footnotemark}
\footnotetext{op.~cit.~\cite{Tolimieri:1998}, \S\ 4.2.}}
Assume $B$ is a subgroup of a finite abelian group $A$ and $f\in \LA$.
Denote the orders of $A$ and $B$ by $N$ and $M$, respectively, with $N = LM$.
Consider the Fourier expansion of $f$,
\[
f = \sum_{a^*\in A^*} \alpha(a^*)a^*
\]
with coefficient set
\[
\alpha \in \vs{L}(A^*)
\]
\begin{definition}[Periodic]\\
The function $f\in \LA$ is called \emph{B-periodic} if, for all $a\in A$ and
$b\in B$, 
\[
f(a+b) = f(a)
\]
and $\alpha \in \vs{L}(A^*)$ is called $B_*$\emph{-decimated} if $\alpha$ vanishes
off of $B_*$.
\end{definition}

\begin{example}
Suppose $f\in \vs{L}(\Z/N)$ is $L\Z/N$-periodic. Recall,
\[
L\Z/N \simeq \{0,L,2L,\ldots,(M-1)L\}, \qquad \text{ for } LM = N
\]
and the vector representation of $f$ can be written
$\vec{f} = (\vec{f}^{(0)},\,\vec{f}^{(0)},\ldots,\vec{f}^{(0)})^t$,
%\[\vec{f} = 
%\left(\begin{array}{c}\vec{f}^{(0)}\\\vec{f}^{(0)}\\\vdots\\\vec{f}^{(0)}\end{array}\right), 
%\quad \text{ where } \quad \vec{f}^{(0)} =
%\left(\begin{array}{c}f(0)\\f(1)\\\vdots\\f(L-1)\end{array}\right)\]
That is $\vec{f}\in \C^N$ is a vector containing $M = N/L$ sub-vectors,
$\vec{f}^{(0)}\in \C^L$.
Identifying $\Z/N$ with its character group by the standard presentation, the
Fourier coefficient set $\alpha \in \vs{L}(\Z/N)$ of $f$ is given by
\[
\alpha = \frac{1}{N} F(N) \vec{f}
\]

Consider the value of $\alpha$ at $a^*\in (\Z/N)^*$.
\begin{align}\label{eq:decimation}
\alpha(a^*) &= \frac{1}{N} \sum_{n=0}^{N-1}f(n)\overline{\langle n,a^*\rangle}\nonumber\\
&= \frac{1}{N} \sum_{a\in \Z/L}\sum_{b\in L\Z/N}f(a+b)\overline{\langle a+b,a^*\rangle}\nonumber\\
&= \frac{1}{N} \sum_{a\in \Z/L}f(a)\overline{\langle a,a^*\rangle}\sum_{b\in L\Z/N}\overline{\langle b,a^*\rangle}
\end{align}
Corollary~\ref{cor:char2} implies, for $a^*\in (\Z/N)^*$,
\[
\sum_{b\in L\Z/N} \overline{\langle b, a^* \rangle} = 
\left\{\begin{array}{ll}
o(L\Z/N),& \quad a^* \in (L\Z/N)_*,\\
0,& \quad \text{otherwise}
\end{array}
\right.
\]
Recall that $M\Z/N = (L\Z/N)_*$, the dual of $L\Z/N$.  
Therefore, since $o(L\Z/N) = M$, (\ref{eq:decimation}) becomes
\[
\alpha(a^*) =
\left\{\begin{array}{ll}
\frac{1}{L} \sum_{a\in \Z/L}f(a)\overline{\langle a,a^*\rangle},& \quad a^* \in M\Z/N,\\
0,& \quad \text{otherwise}
\end{array}
\right.
\]
Thus we have shown that, for any $a^* \notin M\Z/N$, the value of $\alpha(a^*)$ vanishes.
So we say that $\alpha$ is $(L\Z/N)_*$-decimated. 

Before stating this result more generally as
Theorem~\ref{thm:periodic-decimated}, let's consider a more concrete example. 
%, let $v = \exp(2\pi i/N)$, and consider the
The value of $\alpha$ at 1 is
\begin{align*}
\alpha(1) &= \frac{1}{N} \sum_{n=0}^{N-1}f(n)\overline{\langle n,1\rangle}\\
&= \frac{1}{N} \sum_{a=0}^{L-1}\sum_{y = 0}^{M-1}f(a+yL)\overline{\langle a+yL,1\rangle}\\
&= \frac{1}{N} \sum_{a=0}^{L-1}f(a)\overline{\langle a,1\rangle}\sum_{y = 0}^{M-1}\exp(-\I2\pi\frac{yL}{N})
\end{align*}
which vanishes since
\[
\sum_{y = 0}^{M-1}\exp(-\I2\pi\frac{yL}{N})=\sum_{y = 0}^{M-1}\exp(-\I2\pi\frac{y}{M}) = 0
\]
%&= \frac{1}{N} \left(\sum_{n=0}^{M-1}v^{-nL}\right)\left(\sum_{l=0}^{L-1}v^{l}f_l\right),
%\sum_{n=0}^{M-1}v^{-nL} = 0
\end{example}
The foregoing is a special case of the following result:
\begin{theorem}\label{thm:periodic-decimated} 
If $f\in \LA$ has Fourier coefficient set $\alpha$, then $f$
  is $B$-periodic if and only if $\alpha$ is $B_*$-decimated.
\end{theorem}
\begin{proof}
Recall that
\begin{equation}\label{eq:3}
f(a) = \sum_{x^*\in A^*} \alpha(x^*) \langle a, x^* \rangle
\end{equation}
and
\begin{equation}\label{eq:4}
f(a+b) = \sum_{x^*\in A^*} \alpha(x^*) \langle b, x^* \rangle 
                                       \langle a, x^* \rangle
\end{equation}
So $f$ is $B$-periodic if and only if equations~(\ref{eq:3}) and~(\ref{eq:4})
are equal; that is, if and only if,
\[
\alpha(x^*) = \alpha(x^*) \langle b, x^* \rangle
\]
This occurs if and only if $\alpha(x^*)=0 $ whenever 
$\langle b, x^* \rangle \neq 1$ for all $b\in B$; that is, whenever
$x^*\notin B_*$. \qed
\end{proof}

\begin{definition}[Evaluation Function]\\
For any subset $X\subset A$, the \emph{evaluation function}
  of $X$ is the function $e_X \in \LA$ which is equal to 1 on $X$ and
  vanishes elsewhere.
\end{definition}
%%% LEFT OFF HERE %%%
%%% new stuff starts here %%%

%% Next write up Corollary 4.1 p.49 and write EX 4.3 as an exercise %%
\begin{corollary}\label{cor:evalfun}
As usual, assume $A$ is a group of order $N$, and $B$ is a subgroup of $A$
having order $M = N/L$; \eg $A = \Z/N$ and $B=L\Z/N$. If $f \in \LA$ is the
evaluation function on $B$ and 
$\alpha \in \vs{L}(A^*)$ is the Fourier coefficient set of $f$, then $L\alpha$ is
the evaluation function on the dual subgroup $B_*$.
\end{corollary}
\begin{exercise}  Suppose $f \in \vs{L}(\Z/6)$ is the evaluation function on the
  subgroup $2\Z/6$ with Fourier coefficient set $\alpha$.  Show that
  $2\alpha$ is the evaluation function on the dual $3\Z/6$, as implied by
  corollary~\ref{cor:evalfun}.
\end{exercise} 
\begin{solution}
Recall,
%\[ 
%6\Z = \{\ldots,-6,0,6,\ldots\}, \quad \text{and}\quad 2\Z = \{\ldots,-2,0,2,\ldots\}
%\]
%So,
\[ 
\Z/6 \simeq \{0,1,2,3,4,5\},\quad 2\Z/6 \simeq \{0,2,4\},
\quad \text{and} \quad (2\Z/6)_* = 3\Z/6 \simeq \{0,3\}
\]
In terms of the foregoing discussion, this is the special case in which
$N=6$, $L=2$, and $M=3$. If $f$ is the evaluation function on  $2\Z/6$, then
\[
f(k) = \left\{ \begin{array}{ll}
1, & \quad k\in \{0,2,4\}\\
0, & \quad k\in \{1,3,5\}
\end{array}
\right.
\]
Therefore, $f$ is $2\Z/6$-periodic and, by
Theorem~\ref{thm:periodic-decimated}, $\alpha$ is $(2\Z/6)_*$-decimated;
that is, it is zero off of $3\Z/6$.  More precisely,
\begin{align*}
\alpha(a) & = \frac{1}{6}(f,a)= \frac{1}{6} \sum_{j=0}^5 f(j)\overline{\langle j,a\rangle} \\
&= \frac{1}{6} \sum_{j\in\{0,2,4\}}\E^{-\I 2 \pi \frac{ja}{6}}
= \frac{1}{6} \sum_{j=0}^2 v^{ja}\\
&= \frac{1}{6} \left( v^{0a} +  v^{a} +  v^{2a} \right), \qquad \text{ where } v = \E^{-\I 2 \pi/3}
\end{align*}
For example, $\alpha(1) = 0$, since $v^0 = 1$ and $v^1 = v^2 = -1/2$.
We still must show that $2\alpha$ is 1 on $3\Z/6 = \{0,3\}$. Obviously,
$\alpha(0) =1/2$.  Also, $\alpha(3) = (1 +  \E^{-\I 2 \pi} + \E^{-\I 4 \pi})/6 =
1/2$, {\it q.e.d.}
\end{solution}

\begin{definition}[Periodization]\\
Define $\Per_B f\in \LA$ by the formula 
\[
\Per_B f(a) = \sum_{x\in B} f(a+x), \qquad a\in A
\]
and call $\Per_B f$ the \emph{periodization of} $f$ \emph{over} $B$.  The function 
$\Per_B f$ is $B$-periodic.
\end{definition}

\begin{example}
For $f\in \vs{L}(\Z/6)$, the periodization of $f$ over $3\Z/6$ is given by 
\begin{alignat*}{2}
\Per_{3\Z/6} f(0) &= f(0) &+ f(3)\\
\Per_{3\Z/6} f(1) &= f(1) &+ f(4)\\
\Per_{3\Z/6} f(2) &= f(2) &+ f(5)
\end{alignat*}
with the remaining values given by periodicity.
\end{example}

In matrix form, the essential values of the periodization of $f$ over $3\Z/6$
can be given by
\[
\begin{pmatrix}%\left(\begin{array}{cccccc}
1&0&0&1&0&0\\
0&1&0&0&1&0\\
0&0&1&0&0&1
%\end{array}\right)
\end{pmatrix}\vec{f}
\]
\begin{example}
For $f\in \vs{L}(\Z/N)$, the periodization of $f$ over the subgroup $L\Z/N$, for
$N= LM$, is given by 
\[
\Per_{L\Z/N} f(a) = \sum_{m=0}^{M-1} f(a+mL), \qquad 0 \leq l < N
\]
In matrix form the periodization of $f$ over $L\Z/N$ is given by 
\[
\left( I_L\; I_L\; \cdots I_L \right) \vec{f}
\]
The matrix pre-multiplying $\vec{f}$ consists of $M$ copies of the 
$L\times L$ identity matrix $I_L$. 
\end{example}

The \emph{Poisson summation formula} describes the Fourier expansion of
$\Per_B f$.
\begin{theorem}[Poisson Summation Formula]\label{thm:poissonsummation}
Suppose $A$ is a finite abelian group of order $N$, and $B$ is a subgroup of
order $M$, where $N = LM$.  If $f\in \LA$ has Fourier expansion 
\[
f = \sum_{x^* \in A^*} \alpha(a^*) x^*
\]
then $\Per_B f$ has Fourier expansion
\[
\Per_B f = M \sum_{x^* \in B^*} \alpha(a^*) x^*
\]
\end{theorem}
\begin{proof}
For $a \in A$,
\begin{align*}
\Per_B f(x) 
&= \sum_{b \in B}\sum_{a^* \in A^*} \alpha(a^*) \langle x,a^*\rangle \langle b,a^*\rangle\\
&= \sum_{a^* \in A^*} \alpha(a^*) \langle x,a^*\rangle \sum_{b \in B}\langle b,a^*\rangle
\end{align*}
Recall that $\sum_{b \in B}\langle b,a^*\rangle$ vanishes unless $a^* \in B_*$,
in which case the sum equals $M$. \qed
\end{proof}
%This theorem applies to the examples above, \eg $A = \Z/N$, $B = L\Z/N$,
%and $B_* = M\Z/N$. 
\begin{example}
For $0\leq m < M, 0 \leq l < L$, 
\begin{align*}
f(l+mL) &= \sum_{b^* \in B^*} \alpha(b^*) \langle l, x^*\rangle \langle mL, x^*\rangle\\
        &= \sum_{b = 0}^{N-1} \alpha(b) \exp(2\pi \I \frac{lb}{N})\exp(2\pi \I \frac{mb}{M})
\end{align*}
Therefore,
\[
\Per_{L\Z/N} f(l)= \sum_{b = 0}^{N-1} \alpha(b) \exp(2\pi \I \frac{lb}{N})\sum_{m = 0}^{M-1} \exp(2\pi \I \frac{mb}{M})
\]
Let
\[
b = b_1 + b_2M, \qquad 0\leq b_1 < M, 0\leq b_2 < L
\]
so that each $b$ for which $b_1 = 0$ is an element of $M\Z/N = (L\Z/N)_*$.  Then,
%; \ie \[b_1=0 \Rightarrow b = b_2M \in \{0,M,2M,\ldots,(L-1)M\} = M\Z/N  = (L\Z/N)_*\]
\begin{align*}
\Per_{L\Z/N} f(l)&= 
\sum_{b_1 = 0}^{M-1}\sum_{b_2 = 0}^{L-1} \alpha(b_1+b_2M) 
\exp\{2\pi \I \frac{l(b_1+b_2M)}{N}\}\sum_{m = 0}^{M-1} \exp\{2\pi \I \frac{m(b_1+b_2M)}{M}\}\\
&= M \sum_{b_2 = 0}^{L-1} \alpha(b_2M) 
\exp(2\pi \I \frac{lb_2}{L})
\end{align*}
The last equality follows from the argument given in the proof of Theorem~\ref{thm:poissonsummation}.
\end{example}
By duality, we have identified $A$ with the character group basis of
$\vs{L}(A^*)$.  An element $a\in A$ determines the character of $A^*$.
\[
\varTheta(a):a^* \to \overline{\langle a,a^* \rangle}, \qquad a^*\in A^*
\]

Suppose $f \in \LA$ has Fourier coefficient  set $\alpha$. For $a^* \in A^*$,
\[
\alpha(a^*) = \frac{1}{N} (f,a^*)
\]
implying that the expansion of $\alpha$ over the character group basis $A$ of
$\vs{L}(A^*)$ is given by 
\begin{equation}\label{eq:6}
\alpha = \frac{1}{N} \sum_{a\in A}f(a)a
\end{equation}
(This is Tolimieri's notation, but the point is clearer
if we write~(\ref{eq:6}) as follows:
\[
\alpha = \frac{1}{N} \sum_{a\in A}f(a)\varTheta(a)
\]
This emphasizes that the role of $a$ here is that of a character.)

For $\alpha \in \vs{L}(A^*)$, define $\Per_{B_*}(\alpha)\in \vs{L}(A^*)$
by the formula
\[
\Per_{B_*}\alpha(a^*) = \sum_{x^*\in B_*} \alpha(a^*+x^*), \qquad a^*\in A^*
\]
\begin{theorem}
If $f\in \LA$ has Fourier coefficient set $\alpha\in \vs{L}(A^*)$, then 
\begin{equation}\label{eq:7}
\Per_{B_*}\alpha(a^*) = \frac{1}{M}\sum_{x\in B} f(x)x
\end{equation}
\end{theorem}
(Again, we might express the summand in~(\ref{eq:7}) as $f(x) \varTheta(x)$.)
\begin{proof}
For $a^* \in A^*$,
\begin{align*}
\Per_{B_*}\alpha(a^*) 
&= \frac{1}{N}\sum_{x^*\in B_*}\sum_{x\in A} f(x)\varTheta(x)(a^*+x^*)\\
&= \frac{1}{N}\sum_{x\in A} f(x)\overline{\langle x,a^*\rangle}
\sum_{x^*\in B_*}\overline{\langle x,x^*\rangle}
\end{align*}
By Corollary~\ref{cor:char1}, the sum
\[
\sum_{x^*\in B_*}\overline{\langle x,x^*\rangle}
\]
vanishes unless $x\in B$, in which case the sum is equal to $o(B_*) = N/M$. \qed
\end{proof}

%---- LEFT OFF HERE: next -- Section 4.3 of TFR ----%

%\begin{theorem}\label{thm:EquivalenceClass}

%%% END FILE INSERT: TFR.tex
