% This is the file named 'Audio.tex'
%% @LaTeX-file{
%% author = {William DeMeo},
%% filename = {'Audio.tex'}
%% date = {2004/02/18},
%% text = {main latex file for my recent audio algebra research}
%% }
\title{Digital Audio Processing: group filters}
\titlerunning{Digital Audio Processing: group filters}
\author{William DeMeo}\institute{{\tt williamdemeo@yahoo.com}}\authorrunning{ }
\maketitle
%\setcounter{minitocdepth}{2}\dominitoc
%\pagebreak

\section{Introduction}
The translation-invariance of most classical signal
processing transforms and filtering operations is largely
responsible for their widespread use, and is crucial for
efficient algorithmic implementation and interpretation 
of results. Underlying most digital signal processing (DSP)
algorithms is the group $\Z/N$ of integers modulo $N$, which
serves as the data indexing set.  Translations are defined
using addition modulo $N$, and basic operations, including
convolutions and Fourier expansions, are developed relative
to these translations.

DSP on finite abelian groups such as $\Z/N$ is
well-understood and has great practical utility.  Recently,
however, interest in the practical utility of nonabelian
groups has grown significantly. Although the theoretical
foundations of finite nonabelian groups is well established,
application of the theory to DSP has yet to  become
common-place; cf.~the NATO ASI ``Computational
Non-commutative Algebras,'' Italy, 2003. Another notable
exception is~\cite{An:2003}, which develops theory and
algorithms for indexing data with nonabelian
groups, defining translations with a (non-commutative) group
multiply operation, and performing typical DSP operations
relative to these translations. The work of An and Tolimieri
demonstrates that including nonabelian groups among the
possible data indexing strategies significantly broadens the
range of useful signal processing techniques.

The present work demonstrates the use of nonabelian groups
for indexing audio signals, and discusses the insights gained
from this approach computational advantages can be gained. 
For a standard noise reduction problem, we show
how the new framework simplifies the algorithm, while
simultaneously generalizing it to handle more complex noise
reduction tasks. Numerical results are provided along with
the Matlab source code for reproducing them. 
%-----------------------------------------------------------------------------

\section{Nonabelian group DSP}
\subsection{Two distinctions of consequence}
Abelian group DSP can be completely described in terms of a
special class of signals called the \emph{characters} of the
group. (For $\Z/N$, the characters are simply the exponentials.)
Each character of an abelian group represents a one-dimensional
translation-invariant subspace, and the set of all characters
spans the space of signals indexed by the group; any such
signal can be uniquely expanded as a linear combination over
the characters.

In contrast, the characters of a nonabelian group $G$, %on the other hand,
do not determine a basis for expanding signals indexed by
$G$.  However, a basis can be constructed by extending the
characters of an abelian subgroup $A$ of $G$, and then taking
certain translations of these extensions.  Some of the
characters of $A$ cannot be extended to characters of $G$,
but only to proper subgroups of $G$.  This presents some
difficulties involving the underlying translation-invariant
subspaces, some of which are now multi-dimensional.
However, it also presents opportunities for alternative
views of local signal domain information on these
translation-invariant subspaces. 

The other abelian/nonabelian distinction of primary importance 
concerns translations defined on the group.  In the abelian group
case, translations represent simple linear shifts in space or time.
When nonabelian groups index the data, however, translations are no
longer so narrowly defined. 

%-----------------------------------------------------------------------------
%-----------------------------------------------------------------------------
\subsection{Basic notations and definitions}
This section quickly summarizes the notations,
definitions and important facts needed below.
(See~\cite{An:2003} for elaboration.)  Throughout, $\C$ denotes complex
numbers, $G$ an arbitrary (nonabelian) group, %of order $N$, 
and $\LG$ the collection of complex valued functions on $G$.
  
%-----------------------------------------------------------------------------
\subsubsection{Translations}
For $y\in G$, the mapping $\T(y)$ of $\LG$ defined by 
\[
(\T(y)f)(x) = f(y^{-1}x), \quad x \in G
\]
is a linear operator of $\LG$ called 
\emph{left translation by} $y$.
The mapping $\lt{C}(f)$ of $\LG$ defined by 
\[
\lt{C}(f) = \sum_{y\in G} f(y) \T(y), \quad f\in \LG
\]
is a linear operator of $\LG$ called 
\emph{left convolution by} $f$.  By definition, 
\[
(\lt{C}(f)g)(x) = \sum_{y\in G} f(y) g(y^{-1}x), \quad g \in \LG,\; x\in G
\]
%The collection of all left convolutions of $\LG$ is
%$\vs{C}(G) = \left\{\lt{C}(f) : f \in G \right\}$.
For $f, g \in \LG$, the composition
\[
f * g  = \lt{C}(f)g
\]
is called the \emph{convolution product}.
The vector space $\LG$ paired with the convolution product
is an algebra, the \emph{convolution algebra over} $G$.

%-----------------------------------------------------------------------------
\subsubsection{The group algebra}
\label{sec:groupalgebraA}
The \emph{group algebra} $\CG$ is the space of all formal sums
\[
f = \sum_{x\in G} f(x) x, \quad f(x) \in \C
\]
with the following operations:
\[
f+g = \sum_{x\in G} (f(x) + g(x))x, \quad f, g \in \CG,
\]
\[
\alpha f = \sum_{x\in G} (\alpha f(x)) x, \quad \alpha \in \C, f \in \CG,
\]
\[ %\begin{equation}\label{eq:GFIP-1}
fg = \sum_{x\in G}\left(\sum_{y\in G} f(y)g(y^{-1}x)\right)x, \quad f, g \in \CG.
\] %\end{equation}
%The adds and multiplies inside the parentheses are the familiar operations in $\C$.

For $g\in \CG$, the mapping $\lt{L}(g)$ of $\CG$ defined by 
\[
\lt{L}(g)f = gf, \quad f\in \CG
\]
is a linear operator on the space $\CG$ called 
\emph{left multiplication by} $g$.  

Since $y\in G$ can be identified with the formal
sum $e_y \in \CG$ consisting of a single nonzero term,
\begin{equation}\label{eq:leftmult}
yf = \lt{L}(e_y)f = \sum_{x\in G}f(y^{-1}x) x
\end{equation}
In relation to translation of $\LG$, (\ref{eq:leftmult}) is the
$\CG$ analog. 

The mapping $\varTheta: \LG \to \CG$ defined by
\[
\varTheta(f) = \sum_{x\in G} f(x) x, \quad f\in \LG
\]
is an algebra isomorphism of the convolution algebra $\LG$
onto the group algebra $\CG$.  Thus we can identify
$\varTheta(f)$ with $f$, using  context to decide whether
$f$ refers to the function in $\LG$ or the formal sum in
$\CG$.  

%\begin{remark} \label{rem:group-algebra-basis}
%For $y\in G$, the formal sum consisting of a single nonzero term $y$ will be
%denoted by $y$.  We can view $G$ as a subset of $\CG$.  For $x, y \in G$, the
%product $xy$ in $G$ is equal to the product $xy \in \CG$, and the identity
%element 1  of $G$ is the identity element in the group algebra $\CG$.  Thus
%$G$ is a basis of the space \CG\ corresponding to the canonical basis $\{e_y :
%y\in G \}$ of $\LG$.
%\end{remark}
An important aspect of the foregoing isomorphism is the
correspondence between the translations of the spaces.
Translation of $\LG$ by $y\in G$ %$\T(y)$ 
corresponds to left multiplication of $\CG$ by $y\in G$.
%$\lt{L}(y)$ 
Convolution of $\LG$ by $f\in \LG$ corresponds to
left multiplication of $\CG$ by $f\in \CG$. To put it
symbolically,
\begin{eqnarray*}
      \LG & \leftrightarrow & \CG \\
\lt{T}(y) & \leftrightarrow & \lt{L}(y)\\
\lt{C}(f) & \leftrightarrow & \lt{L}(f)
\end{eqnarray*}

%-----------------------------------------------------------------------------

\subsubsection{Left ideals}
A subspace $\vs{V}$ of the space $\CG$ is called a
\emph{left ideal} if 
\[
u\vs{V} = \{uf : f \in \vs{V}\} \subset \vs{V}, \quad u \in G. 
\]
A left ideal of $\CG$ corresponds to a subspace of $\LG$
invariant under all left translations.  If $\vs{V}$ is a
left ideal, then, by linearity, 
$g\vs{V} \subset \vs{V}, \, g \in \CG$.

For $g\in \CG$, the set $\CG g$ defined by 
$\{fg : f \in \CG\}$ is %a left ideal of $\CG$ 
called 
\emph{the left ideal generated by} $g$ in $\CG$. 
%$\CG g = \CG$ if and only if $g$ is an invertible element in
%$\CG$. 

A left ideal $\vs{V}$ of $\CG$ is called \emph{irreducible} if the only left ideals
of $\CG$ contained in $\vs{V}$ are $\{0\}$ and $\vs{V}$. The sum of two distinct,
irreducible left ideals is always a direct sum.
% (\cite{An:2003}, p.~129).

For an abelian group $A$, the group algebra $\C A$ is
decomposed into a direct sum of irreducible ideals.  The
ideals are one-dimensional translation-invariant subspaces.  

%into irreducible ideals is constructed from the character basis
%$A^*$. In the abelian case, the result is a 

Similarly, for a nonabelian group $G$, the group algebra
$\CG$ is decomposed into a direct sum of left ideals.
Again, the ideals are translation-invariant
subspaces, but some of them must now be multi-dimensional,
and herein lies the potential advantage of using nonabelian
groups for indexing the data. The left translations
are more general and represent a broader class of 
transformations. Therefore, projections of data into the
resulting left ideals can reveal more complicated partitions
and structures as compared with the Fourier components in
the abelian group case. 

%----------------------------------------------------------------------------
%-----------------------------------------------------------------------------

\subsection{Fourier analysis on finite nonabelian groups}
A \emph{character} of $G$ is a group homomorphism of $G$
into $\C^{\times}$, where $\C^{\times} = \C \setminus \{0\}$.
In other words, the mapping $\varrho: G \to \C^{\times}$ is a character of
$G$ if it satisfies $\varrho(xy) = \varrho(x)\varrho(y), \, x, y \in G$.

By the aforementioned identification between $\LG$ and
$\CG$, a character $\varrho \in G^*$ can be viewed as a formal sum,
\begin{equation}\label{eq:GFIP-formal-sumA}
\varrho = \sum_{x\in G}\varrho(x)x,
\end{equation}
and therefore $G^* \subset \CG$.

%There always exists at least one character in $G^*$ -- namely, the \emph{trivial character} of
%$G$, which takes on the value 1 for all $y\in G$.
%
%Characters of nonabelian groups share many properties with abelian group
%characters.
\begin{theorem}
If $G$ has a nontrivial character $\varrho$, then
\[
\frac{1}{N} \sum_{x\in G} \varrho(x) = 0
\]
\end{theorem}
\begin{proof}
%Let $y\in G$ be such that $\varrho(y) \neq 0$, and $\varrho(y) \neq 1$.  (If no
%such $y$ exists, the result is obvious.)
For any $y\in G$, by a change of variables,
\[
\varrho(y) \sum_{x\in G} \varrho(x) = \sum_{x\in G} \varrho(yx) = \sum_{x\in G} \varrho(x)
\]
Therefore, either (a) $\varrho(y)=1, \forall y\in G$, or (b) $\sum \varrho(x)=0$.
Since (a) contradicts the hypothesis, (b) must be true. \qed
\end{proof}

\begin{theorem}\label{thm:char-actionA}
Suppose $\varrho$ is a character of $G$.  If $y\in G$, 
\[
y\varrho = \varrho y = \varrho(y^{-1})\varrho
\]
\end{theorem}
\begin{proof}
As above, write $\varrho$ as a formal sum and change variables.
\end{proof}

\newcommand\cor{\corollary}
%\begin{corollary}
\begin{cor}
Suppose $\varrho$ is a character of $G$.  If $f\in \CG$, then
\[
f\varrho = \varrho f = \hat{f}(\varrho)\varrho
\]
where $\hat{f}(\varrho) = \sum_{y\in G} f(y) \varrho(y^{-1})$.
%\end{corollary}
\end{cor}
\begin{proof}
By Theorem~\ref{thm:char-actionA},
\[
f\varrho = \sum_{y\in G} f(y) y\varrho = \left( \sum_{y\in G} f(y)\varrho(y^{-1})\right) \varrho
\]
Similarly for $\varrho f$, mutatis mutandis.
\qed
\end{proof}

%%% LEFT OFF with notes on p. 132 %%%

%-----------------------------------------------------------------------------
%-----------------------------------------------------------------------------
\subsection{Abelian by abelian semidirect products}
To determine whether a particular group is useful for a DSP
application, we must specify exactly how this group
represents the data.
The group representation may reduce computational
complexity, or it may simply make it easier to state,
understand, or model a given problem.

In this section we describe procedures for specifying and
studying a simple class of nonabelian groups that have
proven useful in applications -- the 
\emph{abelian by abelian semidirect products}. These are
perhaps the simplest extension of abelian groups
% case, these are
%groups of the form $G = A \varangle B$, where $A$ and $B$
%are abelian groups. Not surprisingly, 
and DSP over such groups closely resembles that over abelian
groups.  However, the resulting processing tools can have
vastly different characteristics. 

%%% LEFT OFF HERE --> resume from top of page 109
\subsubsection{Semidirect product}
Let $G$ be a finite group of order $N$, $K$ a subgroup of $G$,
and $H$ a normal subgroup of $G$. If $G = HK$ and $H \cap
K = \{1\}$, then we say that $G$ is the 
\emph{internal semidirect product} $G = H \varangle K$. 
It can be shown that $G = H \varangle K$ if and only if every $x \in
G$ has a \emph{unique representation} of the form $x = yz, \; y\in H,
z\in K$.

%% By definition, $K$ is a normal subgroup of $G$ if
%% \[y^{-1}Ky = K\]
%% for all $y\in G$.  In that case, $[H, K] = \{1\}$, and $G$ is simply
%% the usual direct product $H \times K$.  What is new about the internal
%% semidirect product is the possibility that members of $K$ act
%% non-trivially on $H$.

The mapping $\Psi: K \to Aut(H)$, defined by 
\[
\Psi_z(x) = zxz^{-1}, \quad z\in K, x\in H
\]
is a group homomorphism. 
Define the binary composition in $G$
%relative to the representation $G= H\varangle K$
in terms of $\Psi$ as follows:
%\begin{eqnarray*}
\[
x_1x_2 = (y_1z_1)(y_2z_2)= y_1\Psi_{z_1}(y_2)z_1z_2,
\]
\[y_1, y_2 \in H,\; z_1, z_2 \in K. \]
%\end{eqnarray*}

If $G = H\varangle K$ and $K$ is a normal subgroup of $G$,
then $G$ is the ``usual'' direct product (\ie the
cartesian product $H\times K$ along with component-wise
multiplication). What is new in the semidirect product is
the possibility that $K$ acts nontrivially on $H$.

\subsubsection{Cyclic groups}
Denote by $C_N(x)$ the cyclic group of order $N$ having
generator $x$, $\{ x^n : 0 \leq n < N\}$, and define binary
composition by $x^m x^n = x^{m+n}$, $0\leq m, n < N$, where
$m+n$ is addition modulo $N$.  

For an integer $L$, $ 0\leq L < N$, denote by $gp_N(x^L)$
the subgroup generated by $x^L$ in $C_N(x)$. If $L$ divides
$N$, then  
\[
gp_N(x^L) = \{x^mL : 0\leq m < M\}, \quad LM = N
\]
and $gp_N(x^L)$ is a cyclic group of order $M$.

\subsubsection{Group of units}
Multiplication modulo $N$ is a ring product on the group of
integers $\Z/N$. An element $m\in \Z/N$ is called a unit if
there exists an $n\in \Z/N$ such that $mn = 1$.  The set
$U(N)$ of all units in $\Z/N$ is a group with respect to
multiplication modulo $N$, and is called the 
\emph{unit group} of $\Z/N$. 

The unit group $U(N)$ can be characterized as the
set of all integers $0<m<N$ such that $m$ and $N$ are
relatively prime.  For example, $U(8) = \{1, 3, 5, 7\}$.

\subsubsection{Semidirect product example}
%\begin{theorem}
The mapping $\Psi: U(N) \to Aut(C_N(x))$ is a group
isomorphism.
%\end{theorem}
Under this identification, we can form $C_N(x)\varangle K$
for any subgroup $K$ of $U(N)$.  A typical point in 
$C_N(x)\varangle K$ is denoted 
$(x^n, u), \, 0\leq n<N,\, u\in K$ with multiplication given
by  
\[
(x^m, u)(x^n, v) = (x^{m+un},uv), \quad 
0\leq m, n < N,\, u, v \in K
\]
where $m+un$ is taken modulo $N$.
%-----------------------------------------------------------------------------
%-----------------------------------------------------------------------------
