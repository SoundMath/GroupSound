%% LaTeX file 'GroupTheoryND.tex'

%% latexfile{
%% author = {William DeMeo},
%% filename = {GroupTheoryND.tex},
%% date = {2002.08.07},
%% text = {Main latex file for Group Theory for Signals of Multiple Dimensions of dspmath book.}
%% }

%%% BEGIN FILE INSERT: GFIP.tex

\title{Nonabelian Groups and Multivariate DSP}
\titlerunning{Nonabelian Groups \& Multivariate DSP}
\author{Myoung An\inst{1}\and Richard Tolimieri \and William DeMeo\inst{2}}
\institute{{\tt psypher@tiac.net} \and {\tt williamdemeo@yahoo.com}}
\authorrunning{An, Tolimieri, DeMeo}
\maketitle
%\setcounter{minitocdepth}{2}\dominitoc %\pagebreak

%The main source of most material in this chapter is An and Tolimieri,
%{\it Group Filters and Image Processing}~\cite{An:2003}.

\section{Preliminaries}
We first recall some useful notations and definitions.
See~\cite{An:2003} for elaboration.  Throughout, $\C$ denotes complex
numbers, $G$ is an arbitrary (nonabelian) group, %of order $N$, 
and $\LG$ denotes the collection of complex valued functions on $G$.

\subsection{Translations}
For $y\in G$, the mapping $\T(y)$ of $\LG$ defined by 
\[
(\T(y)f)(x) = f(y^{-1}x), \qquad x \in G
\]
is a linear operator of $\LG$ called \emph{left translation by} $y$.  The
collection of all left translations of $\LG$ is
\[
\vs{T}(G) = \left\{\T(y) : y \in G \right\}.
\]
The mapping $\T:G \to \vs{T}(G)$ satisfies
\[
\T(yz) = \T(y)\T(z), \qquad y, z \in G,
\]
\[
\T(y^{-1}) = \T(y)^{-1}, \qquad y \in G.
\]
These formulae imply that $\vs{T}(G)$ is a group under operator composition and $\T$
is a group isomorphism of $G$ onto $\vs{T}(G)$.

If $G$ has order $N$, then $y^N = 1$ and 
\[
\T(y)^N = \T(y^N) = \T(1) = I, \qquad y \in G.
\]

For $y\in G$, define $e_y\in \LG$ as the function whose value is 1 at $y$ and
0 otherwise.  The set $\{e_y : y\in G\}$ is a basis called the \emph{canonical
  basis} of $\LG$.  Left translations permute the elements in the canonical
basis.
\[
\T(y)e_z = e_{yz}, \qquad y, z \in G
\]

For $f\in \LG$, the mapping $\lt{C}(f)$ of $\LG$ defined by 
\[
\lt{C}(f) = \sum_{y\in G} f(y) \T(y)
\]
is a linear operator of $\LG$ called \emph{left convolution by} $f$.  By
definition,
\[
(\lt{C}(f)g)(x) = \sum_{y\in G} f(y) g(y^{-1}x), \qquad g \in \LG,\; x\in G
\]
The collection of all left convolutions of $\LG$ is $\vs{C}(G) = \left\{\lt{C}(f) : f \in G \right\}$.

The mapping $\lt{C}: \vs{L}(G) \to \vs{C}(G)$ satisfies 
\[
\lt{C}(f+g) = \lt{C}(f) + \lt{C}(g), \qquad f, g \in \LG,
\]
\[
\lt{C}(\alpha f) = \alpha \lt{C}(f), \qquad \alpha \in \C,\; f \in \LG.
\]
These properties imply that $\vs{C}(G)$ is a vector space under addition and scalar
multiplication of operators and $\lt{C}$ is a linear isomorphism from $\LG$ onto
$\vs{C}(G)$.  Furthermore, since $\T(y) = \lt{C}(e_y),\; y\in G,$
\[
\vs{T}(G) \subset \vs{C}(G)
\]

For $f, g \in \LG$,
\[
\lt{C}(f)\lt{C}(g) = \sum_{y\in G}\sum_{z\in G} f(y)g(z)\T(y)\T(z) = \sum_{y\in
  G}\sum_{z\in G} f(y)g(z)\T(yz).
\]
By the change of variables $u = yz$,
\begin{equation}\label{eq:GFIP-1}
\lt{C}(f)\lt{C}(g) = \sum_{u\in G}\left(\sum_{y\in G} f(y)g(y^{-1}u)\right)\T(y).
\end{equation}
The inner summation is a function in $\LG$, so $\vs{C}(G)$ is closed under
composition and is a subalgebra of the algebra of all linear operators on
$\LG$. 

For $f, g \in \LG$, define the \emph{convolution product}, denoted $*$, as
follows: 
\[
f * g  = \lt{C}(f)g.
\]
By~(\ref{eq:GFIP-1}),
\[
\lt{C}(f)\lt{C}(g) = \lt{C}(f*g).
\]
The vector space $\LG$ paired with the convolution product is an algebra,
the \emph{convolution algebra over} $G$.
\begin{theorem}
The mapping $\lt{C}: \vs{L}(G) \to \vs{C}(G)$ is an algebra isomorphism of the
convolution algebra $\LG$ onto the algebra of left convolutions of $\LG$.
\end{theorem}

Note that for the general case in which $G$ is nonabelian, the convolution
product is noncommutative.  For, by the change of variables $y = z^{-1}x$, it
is clear that
\[
(g * f)(x) = \sum_{z\in G} g(z)f(z^{-1}x) = \sum_{y\in G} f(y) g(xy^{-1}), \qquad f, g \in \LG,\;
x\in G.
\]
Since $G$ is nonabelian, $xy^{-1} \neq y^{-1}x$ for some $x, y
\in G$. Thus convolution is not a commuting product.

\subsection{The Group Algebra $\CG$} 
\label{sec:groupalgebra}
The \emph{group algebra} $\CG$ is the space of all formal sums
\[
f = \sum_{x\in G} f(x) x, \qquad f(x) \in \C
\]
with the following operations:
\[
f+g = \sum_{x\in G} (f(x) + g(x))x, \qquad f, g \in \CG,
\]
\[
\alpha f = \sum_{x\in G} (\alpha f(x)) x, \qquad \alpha \in \C, f \in \CG,
\]
and
\[ %\begin{equation}\label{eq:GFIP-1}
fg = \sum_{x\in G}\left(\sum_{y\in G} f(y)g(y^{-1}x)\right)x, \qquad f, g \in \CG.
\] %\end{equation}
The adds and multiplies inside the parentheses are the familiar operations in $\C$.

$\CG$ is an algebra and the mapping $\varTheta: \LG \to \CG$ defined by
\[
\varTheta(f) = \sum_{x\in G} f(x) x, \qquad f\in \LG
\]
is an algebra isomorphism of the convolution algebra $\LG$ onto the group
algebra $\CG$.  Thus we are free to denote $\varTheta(f)$ by $f$, using 
context to decide whether $f$ symbolizes a function in $\LG$
or a formal sum in $\CG$. 

\begin{remark}\label{rem:group-algebra-basis}
For $y\in G$, the formal sum consisting of a single nonzero term $y$ will be
denoted by $y$.  We can view $G$ as a subset of $\CG$.  For $x, y \in G$, the
product $xy$ in $G$ is equal to the product $xy \in \CG$, and the identity
element 1  of $G$ is the identity element in the group algebra $\CG$.  Thus
$G$ is a basis of the space \CG\ corresponding to the canonical basis $\{e_y :
y\in G \}$ of $\LG$.
\end{remark}

\begin{remark}
Keep in mind that multiplication in $\CG$ corresponds to convolution in $\LG$
and, as such, does not behave like simple multiplication over $\R$ or $\C$.
In particular, the group algebra can have \emph{zero divisors}.  This means
that there exist $f, g \in \CG$ such that $fg = 0$, but $f\neq 0$ and $g \neq 0$.
\end{remark}

\subsection{Left Ideal Decompositions}
For $g\in \CG$, the mapping $\lt{L}(g)$ of $\CG$ defined by 
\[
\lt{L}(g)f = gf, \qquad f\in \CG
\]
is a linear operator on the space $\CG$ called \emph{left multiplication by}
$g$.  

Recall remark~\ref{rem:group-algebra-basis} which identifies $u \in G$ with $u\in
\CG$.  The mapping $\lt{L}$ defined above can be thought of as an ``overloaded''
operator in the sense that it represents translation or 
convolution depending on its argument.  More precisely, left translation by
$u\in G$ corresponds to the left multiplication $\lt{L}(u)$, whereas 
left convolution by $g \in \LG$ corresponds to the left multiplication $\lt{L}(g)$.

Denote by $L(G)$ the collection of all left multiplications $\lt{L}(y),\; y\in G$.
Since 
\[
\lt{L}(yz) = \lt{L}(y)\lt{L}(z), \qquad y, z \in G,
\]
\[
\lt{L}(y^{-1}) = \lt{L}(y)^{-1}, \qquad y \in G,
\]
$L(G)$ is a group under operator composition and the mapping $\lt{L}: G\to
L(G)$ is a group isomorphism from $G$ onto $L(G)$.  $L(G)$ is a noncommuting
family of linear operators of $\CG$ and, consequently, there cannot exist an
$L(G)$-eigenvector basis.  

A subspace $\vs{V}$ of the space $\CG$ is called a \emph{left ideal} if
\[
u\vs{V} = \{uf : f \in \vs{V}\} \subset \vs{V}, \qquad u \in G. 
\]
A left ideal of $\CG$ corresponds to a subspace of $\LG$ invariant under all
left translations.  If $\vs{V}$ is a left ideal, then, by linearity,
\[
g\vs{V} \subset \vs{V}, \qquad g \in \CG.
\]
For $g\in \CG$, the set $\CG g$ defined by 
\[
\CG g = \{fg : f \in \CG\}
\]
is a left ideal of $\CG$ called \emph{the left ideal generated by} $g$ in $\CG$.
$\CG g = \CG$ if and only if $g$ is an invertible element in $\CG$.

A left ideal $\vs{V}$ of $\CG$ is called \emph{irreducible} if the only left ideals
of $\CG$ contained in $\vs{V}$ are $\{0\}$ and $\vs{V}$. The sum of two distinct,
irreducible left ideals is always a direct sum (\cite{An:2003}, p.~129).

For an abelian group, $A$, a direct sum decomposition of $\C A$
into ideals is constructed from the character basis
$A^*$. In the abelian case, these irreducible ideals are one-dimensional.
Direct sum decompositions of $\CG$ into left ideals play the same
role with respect to left translations defined by $G$ (\ie $\lt{L}(u), \;u \in G$)
as direct sum decompositions of $\C A$ into ideals play with respect to
classical translations.   However, \emph{a potential advantage in using
  nonabelian groups as data indexing sets is the wide scope and complexity of
  nonabelian group translations.}  
Projections of data into these left ideals
can usually reveal more complicated partitions and structures in the data as
compared with the Fourier components in the abelian group case.

\section{Fourier Analysis on Finite Nonabelian Groups}
A \emph{character} of $G$ is a group homomorphism of $G$ into $\C^{\times}$,
where $\C^{\times}$ denotes the non-zero complex numbers.
In other words, a mapping $\varrho: G \to \C^{\times}$ is a character of
$G$ if it satisfies 
\[
\varrho(xy) = \varrho(x)\varrho(y), \qquad x, y \in G
\]
Denote the collection of all characters of $G$ by $G^*$.  Under the usual
identification between $\LG$ and $\CG$, we can view a character $\varrho \in G^*$
as a formal sum in $\CG$, 
\begin{equation}\label{eq:GFIP-formal-sum}
\varrho = \sum_{x\in G}\varrho(x)x,
\end{equation}
and $G^*$ is a subset of \CG.

There always exists at least one character in $G^*$ -- namely, the \emph{trivial character} of
$G$, which takes on the value 1 for all $y\in G$.

Characters of nonabelian groups share many properties with abelian group
characters.
\begin{theorem}
If $G$ has a nontrivial character $\varrho$, then
\[
\frac{1}{N} \sum_{x\in G} \varrho(x) = 0
\]
\end{theorem}
\begin{proof}
%Let $y\in G$ be such that $\varrho(y) \neq 0$, and $\varrho(y) \neq 1$.  (If no
%such $y$ exists, the result is obvious.)
For any $y\in G$, by a change of variables,
\[
\varrho(y) \sum_{x\in G} \varrho(x) = \sum_{x\in G} \varrho(yx) = \sum_{x\in G} \varrho(x)
\]
Therefore, either (a) $\varrho(y)=1, \forall y\in G$ or (b) $\sum_{x\in G} \varrho(x) = 0$.
Since (a) contradicts the assumption that $G$ has a nontrivial
character, (b) must be true. \qed
\end{proof}

\begin{theorem}\label{thm:char-action}
Suppose $\varrho$ is a character of $G$.  If $y\in G$, 
\[
y\varrho = \varrho y = \varrho(y^{-1})\varrho
\]
\end{theorem}
\begin{proof}
Exercise.\\
Hint: a straight-forward, two-line proof follows from the standard machine (\ie write
$\varrho$ as the formal sum~(\ref{eq:GFIP-formal-sum}) and change variables;
\cite{An:2003}, p.131). 
\end{proof}

\begin{corollary}
Suppose $\varrho$ is a character of $G$.  If $f\in \CG$, then
\[
f\varrho = \varrho f = \hat{f}(\varrho)\varrho
\]
where $\hat{f}(\varrho) = \sum_{y\in G} f(y) \varrho(y^{-1})$.
\end{corollary}
\begin{proof}
By Theorem~\ref{thm:char-action},
\[
f\varrho = \sum_{y\in G} f(y) y\varrho = \left( \sum_{y\in G} f(y)\varrho(y^{-1})\right) \varrho
\]
and
\[
\varrho f = \sum_{y\in G} f(y) \varrho y = \left( \sum_{y\in G} f(y)\varrho(y^{-1})\right) \varrho.
\]
\qed
\end{proof}

%%% LEFT OFF with notes on p. 132 %%%

\subsection{Abelian by Abelian Semidirect Products}
In order to determine whether a particular group is useful for a DSP application, we must
specify exactly how this group represents the data and decide whether the resulting 
indexing scheme is helpful in some way.  For example, the group representation
may (or may not) reduce computational complexity, or it may simply allow us to
more easily state, understand, or model the given problem.

In this section we describe procedures for specifying and studying a simple
class of nonabelian groups that have proven useful in applications -- the
\emph{abelian by abelian semidirect products}.
These are groups of the form $G = A \varangle B$, where $A$ and $B$ are abelian groups,
are they perhaps the simplest generalizations of abelian groups.  Not surprisingly,
DSP over such groups closely resembles that of abelian groups.  However, the
resulting processing tools can have vastly different characteristics.  

%%% LEFT OFF HERE --> resume from top of page 109

\subsection{Examples}

\begin{example}[Semidirect Product, $G_4 = A \varangle C_2(k_c)$]\\
(An and Tolimieri~\cite{An:2003}, p.198)
\begin{itemize}
\item $G_4 = A \varangle C_2(k_c)$, where $A = C_N(x) \times C_N(y)$,
\[
c =  \begin{pmatrix}
      \,1 \,& \,0\, \\
      \,1\, &\, -1\,
     \end{pmatrix}
\]
\item Group multiplication
\[
x^N = y^N = k_c^2 = 1, \quad xy = yx,\quad k_c x^m y^n = x^m y^{m-n}k_c.
\]
%\item Group actions on characters
\end{itemize}
A function $f\in \C G_4$ is given by the formal sum
\[ 
 f = \sum_{l=0}^1 \sum_{m=0}^{N-1} \sum_{n=0}^{N-1}
      f(x^m y^n k_c^l)x^m y^n k_c^l
\]
Specifying an ordering of elements in $G_4$ by the index array,
\[
\begin{pmatrix}
\, 1\, \\
\, k_c\,
\end{pmatrix}
                                \otimes 
\begin{pmatrix}
\left\{ x^m y^n \right\}_{m,n}
\end{pmatrix} 
\begin{pmatrix}
1      & y  &  \ldots & y^{N-1} \\
x      & xy &         & \\
\vdots &    & \ddots  &  \\
x^{N-1}&    &         &  x^{N-1} y^{N-1}\\
\multicolumn{4}{c}{ \text{ ----------- } } \\
k_c    & yk_c  & \ldots & y^{N-1}k_c \\
xk_c   & xyk_c &         & \\
\vdots &       & \ddots  &  \\
x^{N-1}k_c&    &         &  x^{N-1} y^{N-1}k_c 
\end{pmatrix}
\]
uniquely specifies a $2N \times N$ matrix representation of $f$ given by the coefficients,
\[
\mathbf{f} = 
\begin{pmatrix}
         \left\{ f(x^m y^n)\right\}_{m,n} \\
\text{ -------- }\\
         \left\{ f(x^m y^n k_c)\right\}_{m,n} \\
\end{pmatrix}
\]
\end{example}
%\left(
%\begin{array}{c}
%\begin{matrix}
%            f(x^0 y^0 k_c^0) & f(x^0 y^1 k_c^0) & \cdots & f(x^0 y^{N-1}k_c^0)\\
%            f(x^1 y^0k_c^0) & f(x^1 y^1k_c^0) & & \\
%            \vdots &  & \ddots & \\
%            f(x^{N-1} y^0k_c^0) & & & f(x^{N-1} y^{N-1}k_c^0)\\
%\end{matrix}
%\\
%--
%\\
%\begin{matrix}
%            f(x^0 y^0 k_c^1) & f(x^0 y^1 k_c^1) & \cdots & f(x^0 y^{N-1} k_c^1)\\
%            f(x^1 y^0 k_c^1) & f(x^1 y^1 k_c^1) && \\
%            \vdots &  & \ddots & \\
%            f(x^{N-1} y^0 k_c^1) & & & f(x^{N-1} y^{N-1} k_c^1)
%\end{matrix}
%\end{array}
%\right)
%\]
%These are the coefficients corresponding to the index matrix
%\[\begin{pmatrix}\, 1\, \\\, k_c\,\end{pmatrix}\otimes \left[ x^m y^n \right]_{0\leq m,n<N}\]

\begin{example}[Semidirect Product, $G_5 = A \varangle C_6(k_d)$]\\
(An and Tolimieri~\cite{An:2003}, p.204)
\begin{itemize}
\item $G_5 = A \varangle C_6(k_d)$, where $A = C_N(x) \times C_N(y)$,
$N = 3 \times 2^K$ for an integer $K\geq 2$, and 
\[
d =
\begin{pmatrix}
\,-1 \,& \,2^K+1\, \\
\,2^K-1 \,& \,2^K\,
\end{pmatrix}
\]
\item Group multiplication\\
Let $M = 2^K$.  Since $N=3 \times 2^K$, this yields the mod
$N$ relations
\[
M^2 \equiv 
  \begin{cases}
    M, & \text{if $K$ is even}, \\
    2M,& \text{if $K$ is odd}.
  \end{cases} 
\]
Consider $K$ even, in which case,
\[
M^2 \equiv M, \quad (M+1)^2 \equiv 1, \quad (2M-1)^2 \equiv 1
\]
Therefore, 
\[
d = \begin{pmatrix}
     \,-1 \,& \,M+1\, \\
     \,M-1 \,& \,M\,
    \end{pmatrix},\quad
d^2 =\begin{pmatrix}
     \,M \,& \,M-1\, \\
     \,2M+1 \,& \,2M-1\,
    \end{pmatrix},\quad
d^3 =\begin{pmatrix}
     \,M+1 \,& \,2M\, \\
     \,0 \,& \,1\,
    \end{pmatrix}
\]
\[
d^4 =\begin{pmatrix}
     \,2M-1 \,& \,2M+1\, \\
     \,M-1 \,& \,M\,
    \end{pmatrix}, \quad
d^5 =\begin{pmatrix}
     \,2M \,& \,-1\, \\
     \,2M+1 \,& \,2M-1\,
    \end{pmatrix}, \quad
d^6 = 1.
\]
and group multiplication satisfies the following relations:
\[
x^N = y^N = k_d^6 = 1,\quad xy = yx,\quad k_d x^m y^n = x^{-m+(M+1)n} y^{(M-1)m+Mn}k_d.
\]
%\item Group actions on characters
\end{itemize}
A function $f\in \C G_5$ is given by the formal sum
\[ 
 f = \sum_{l=0}^5 \sum_{m=0}^{N-1} \sum_{n=0}^{N-1}
      f(x^m y^n k_d^l)x^m y^n k_d^l
\]
\end{example}
%%% END FILE INSERT: GFIP.tex
