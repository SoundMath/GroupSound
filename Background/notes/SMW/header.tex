%%-----------------------------------------------------------------
% header.tex
% General header content for LaTeX paper: notes.tex
% Author: William DeMeo
% Modified: 00.11.04
% Usage: Start document with the lines
%           %%-----------------------------------------------------------------
% header.tex
% General header content for LaTeX paper: notes.tex
% Author: William DeMeo
% Modified: 00.11.04
% Usage: Start document with the lines
%           %%-----------------------------------------------------------------
% header.tex
% General header content for LaTeX paper: notes.tex
% Author: William DeMeo
% Modified: 00.11.04
% Usage: Start document with the lines
%           %%-----------------------------------------------------------------
% header.tex
% General header content for LaTeX paper: notes.tex
% Author: William DeMeo
% Modified: 00.11.04
% Usage: Start document with the lines
%           \input{header}
%           \begin{document}
%        Then add title with commands \title, \author, \maketitle,
%        \tableofcontents, etc.
%%-----------------------------------------------------------------
%**start of header
%\documentclass[notitlepage,12pt]{article}
\documentclass[11pt]{article}
%\documentclass{amsart}
%\usepackage{spconf,amsmath,epsfig}
\usepackage{amsmath,amssymb,amscd,epsfig}

\usepackage{fullpage}

\usepackage{ifthen}

%% Should figures be compiled?
\newboolean{nofigures} % initially false ==> compile with figures
% comment out the following line to compile with figures
%\setboolean{nofigures}{true} 

%% Should citation footnotes on section headings be compiled?
\newboolean{nofootnotes}
% comment out the following line to include all citation foonotes.
\setboolean{nofootnotes}{true} 

%\usepackage{boxedminipage}
\usepackage{theorem}
{\theorembodyfont{\rmfamily} 
 \newtheorem{definition}{Definition}[section]
 \newtheorem{lemma}{Lemma}[section] % Lemmas are for proving theorems and facts
 \newtheorem{prop}{Proposition}[section] % Props state any new results of this document
 \newtheorem{fact}{Fact}[section]   % Facts are for less important, well-known results
 \newtheorem{example}{Example}[section]   
 \newtheorem{theorem}{Theorem}[section]  % Theorems are for important, well-known results
 {\theoremstyle{marginbreak}  
   \newtheorem{define}{}  % This definition style is for the glossary
    \newtheorem{Theorem}{Theorem}[subsection]  % Theorems are for important, well-known results
 }
}
\renewcommand{\thedefine}{}

% THE FILE
%   /usr/share/texmf/doc/latex/hyperref/manual.pdf
%  documents the hyperref package
%\usepackage{hyperref}

% THE FILE
%   /usr/share/texmf/doc/latex/general/lshort.dvi (p.12)
%  documents the pagestyle command
%\pagestyle{headings}
% 2004.03.30: commented next six lines:
%\usepackage{fancyhdr}
%\pagestyle{fancy}
%\lhead{\thesection} \chead{} \rhead{\thepage} 
%\lfoot{Do Not Duplicate} \cfoot{\bfseries CONFIDENTIAL} \rfoot{Do Not Duplicate} 
%\renewcommand{\headrulewidth}{0pt} \renewcommand{\footrulewidth}{0.4pt}

% THE FILE
%   /usr/share/texmf/doc/pdflatex/base/example.tex
% suggests the following:
%\input pdfcolor.tex
%\pdfoutput=1\relax      % turn on PDF output; otherwise output DVI file
                         % this primitive cannot be specified  *after* shipping
                         % out the *first* page
%\pdfpagewidth=8.26in    % page width of PDF output
%\pdfpageheight=11.69in  % page height of PDF output
%\usepackage[pdftex]{graphicx}
%\DeclareGraphicsExtensions{.jpg,.pdf,.mps,.png}

%-------------------------------------------------------------
% Definitions.
% ------------
%\def\HOME{/home/moonchild/pub/research/dsp/DeMeo/Notes}
\def\HOME{/home/sonic/pub/research/dsp/DeMeo/Notes}
\def\R{\mathbb{R}}    % real numbers 
\def\C{\mathbb{C}}    % complex numbers 
\def\e{\mathrm{e}}    % exponential
% Two notations for real part of a complex number
   \def\Real{\mbox{Re}}  %\def\Real{\Re}
% Two notations for integration over R
   \def\integral{\int_{-\infty}^{\infty}}
     %\def\integral{\int_{-\infty}^{+\infty}} 
% Operator Theory
   \def\A{\operatorname{A}}
   \def\E{\operatorname{E}}
   \def\H{\operatorname{H}}
   \def\I{\operatorname{I}}
   \def\P{\operatorname{P}}
   \def\pv{\operatorname{pv}}
   \def\S{\operatorname{S}}
   \def\T{\operatorname{T}}
   \def\W{\operatorname{W}}
   \def\Hilbert{\mathcal{H}}
   \def\Hone{\mathcal{H}_1}
   \def\Htwo{\mathcal{H}_2}
   \def\Banach{\mathcal{B}(\Hilbert,\Hilbert)}
   \def\Banachonetwo{\mathcal{B}(\Hone,\Htwo)}
   \def\Banachtwoone{\mathcal{B}(\Htwo,\Hone)}
   \def\Lone{L^1}
   \def\Ltwo{L^2}
   \def\ltwo{\ell^2}
   \def\LoneR{L^1(\mathbb{R})}
   \def\LtwoR{L^2(\mathbb{R})}
   \def\ltwoR{\ell^2(\mathbb{R})}
   \def\Null{\mathcal{N}}  % nullspace
% Signals, Time-Frequency shifts, etc.
   \def\scale{\frac{1}{\sqrt{s}}}
   \def\transcale{\left(\frac{t-u}{s}\right)}
   \def\xpull{x_{\frac{\tau}{2}, -\frac{\xi}{2}}}
   \def\xpush{x_{-\frac{\tau}{2}, \frac{\xi}{2}}}
   \def\xpullpush{x_{\frac{\tau}{2}, \frac{\xi}{2}}}
   \def\xpushpull{x_{-\frac{\tau}{2}, -\frac{\xi}{2}}}
   \def\xfpull{x_{-\frac{\Delta\nu}{2}}}
   \def\xfpush{x_{\frac{\Delta\nu}{2}}}
   \def\apull{a_{-}}
   \def\apush{a_{+}}

   \def\ytnu{y_{t,\nu}}
   \def\ytnut{\tilde{y}_{t,\nu}}
   \def\half{{\scriptstyle \frac{1}{2}}}
   \def\halftau{{\scriptstyle \frac{\tau}{2}}}
   \def\halfxi{{\scriptstyle \frac{\xi}{2}}}
   \def\halfnu{{\scriptstyle \frac{\nu}{2}}}
   \def\halfDnu{{\scriptstyle \frac{\Delta\nu}{2}}}

   \def\xtpull{x\left(t+\halftau\right)}
   \def\xtpush{x\left(t-\halftau\right)}
%   (shouldn't need this) \def\xtpullconj{x^*\left(t+\halftau\right)}
   \def\xtpushconj{x^*\left(t-\halftau\right)}
   \def\Xtpull{X\left(\nu+\halfxi\right)}
   \def\Xtpush{X\left(\nu-\halfxi\right)}
   \def\Xtpullconj{X^*\left(\nu+\halfxi\right)}
%   (shouldn't need this) \def\Xtpushconj{X^*\left(\nu-\halfxi\right)}

   \def\ytpull{y\left(t+\halftau\right)}
   \def\ytpush{y\left(t-\halftau\right)}
%   (shouldn't need this) \def\ytpullconj{y^*\left(t+\halftau\right)}
   \def\ytpushconj{y^*\left(t-\halftau\right)}
   \def\Ytpull{Y\left(\nu+\halfxi\right)}
   \def\Ytpush{Y\left(\nu-\halfxi\right)}
   \def\Ytpullconj{Y^*\left(\nu+\halfxi\right)}
%   (shouldn't need this) \def\Ytpushconj{Y^*\left(\nu-\halfxi\right)}

% Language
   \def\FT{Fourier transform}
   \def\WT{Wigner transform}
   \def\WV{Wigner-Ville}

%%-------------------------------------------------------------
%**end of header
%\includeonly{./theory/ambiguity,./theory/energydensity}

%           \begin{document}
%        Then add title with commands \title, \author, \maketitle,
%        \tableofcontents, etc.
%%-----------------------------------------------------------------
%**start of header
%\documentclass[notitlepage,12pt]{article}
\documentclass[11pt]{article}
%\documentclass{amsart}
%\usepackage{spconf,amsmath,epsfig}
\usepackage{amsmath,amssymb,amscd,epsfig}

\usepackage{fullpage}

\usepackage{ifthen}

%% Should figures be compiled?
\newboolean{nofigures} % initially false ==> compile with figures
% comment out the following line to compile with figures
%\setboolean{nofigures}{true} 

%% Should citation footnotes on section headings be compiled?
\newboolean{nofootnotes}
% comment out the following line to include all citation foonotes.
\setboolean{nofootnotes}{true} 

%\usepackage{boxedminipage}
\usepackage{theorem}
{\theorembodyfont{\rmfamily} 
 \newtheorem{definition}{Definition}[section]
 \newtheorem{lemma}{Lemma}[section] % Lemmas are for proving theorems and facts
 \newtheorem{prop}{Proposition}[section] % Props state any new results of this document
 \newtheorem{fact}{Fact}[section]   % Facts are for less important, well-known results
 \newtheorem{example}{Example}[section]   
 \newtheorem{theorem}{Theorem}[section]  % Theorems are for important, well-known results
 {\theoremstyle{marginbreak}  
   \newtheorem{define}{}  % This definition style is for the glossary
    \newtheorem{Theorem}{Theorem}[subsection]  % Theorems are for important, well-known results
 }
}
\renewcommand{\thedefine}{}

% THE FILE
%   /usr/share/texmf/doc/latex/hyperref/manual.pdf
%  documents the hyperref package
%\usepackage{hyperref}

% THE FILE
%   /usr/share/texmf/doc/latex/general/lshort.dvi (p.12)
%  documents the pagestyle command
%\pagestyle{headings}
% 2004.03.30: commented next six lines:
%\usepackage{fancyhdr}
%\pagestyle{fancy}
%\lhead{\thesection} \chead{} \rhead{\thepage} 
%\lfoot{Do Not Duplicate} \cfoot{\bfseries CONFIDENTIAL} \rfoot{Do Not Duplicate} 
%\renewcommand{\headrulewidth}{0pt} \renewcommand{\footrulewidth}{0.4pt}

% THE FILE
%   /usr/share/texmf/doc/pdflatex/base/example.tex
% suggests the following:
%\input pdfcolor.tex
%\pdfoutput=1\relax      % turn on PDF output; otherwise output DVI file
                         % this primitive cannot be specified  *after* shipping
                         % out the *first* page
%\pdfpagewidth=8.26in    % page width of PDF output
%\pdfpageheight=11.69in  % page height of PDF output
%\usepackage[pdftex]{graphicx}
%\DeclareGraphicsExtensions{.jpg,.pdf,.mps,.png}

%-------------------------------------------------------------
% Definitions.
% ------------
%\def\HOME{/home/moonchild/pub/research/dsp/DeMeo/Notes}
\def\HOME{/home/sonic/pub/research/dsp/DeMeo/Notes}
\def\R{\mathbb{R}}    % real numbers 
\def\C{\mathbb{C}}    % complex numbers 
\def\e{\mathrm{e}}    % exponential
% Two notations for real part of a complex number
   \def\Real{\mbox{Re}}  %\def\Real{\Re}
% Two notations for integration over R
   \def\integral{\int_{-\infty}^{\infty}}
     %\def\integral{\int_{-\infty}^{+\infty}} 
% Operator Theory
   \def\A{\operatorname{A}}
   \def\E{\operatorname{E}}
   \def\H{\operatorname{H}}
   \def\I{\operatorname{I}}
   \def\P{\operatorname{P}}
   \def\pv{\operatorname{pv}}
   \def\S{\operatorname{S}}
   \def\T{\operatorname{T}}
   \def\W{\operatorname{W}}
   \def\Hilbert{\mathcal{H}}
   \def\Hone{\mathcal{H}_1}
   \def\Htwo{\mathcal{H}_2}
   \def\Banach{\mathcal{B}(\Hilbert,\Hilbert)}
   \def\Banachonetwo{\mathcal{B}(\Hone,\Htwo)}
   \def\Banachtwoone{\mathcal{B}(\Htwo,\Hone)}
   \def\Lone{L^1}
   \def\Ltwo{L^2}
   \def\ltwo{\ell^2}
   \def\LoneR{L^1(\mathbb{R})}
   \def\LtwoR{L^2(\mathbb{R})}
   \def\ltwoR{\ell^2(\mathbb{R})}
   \def\Null{\mathcal{N}}  % nullspace
% Signals, Time-Frequency shifts, etc.
   \def\scale{\frac{1}{\sqrt{s}}}
   \def\transcale{\left(\frac{t-u}{s}\right)}
   \def\xpull{x_{\frac{\tau}{2}, -\frac{\xi}{2}}}
   \def\xpush{x_{-\frac{\tau}{2}, \frac{\xi}{2}}}
   \def\xpullpush{x_{\frac{\tau}{2}, \frac{\xi}{2}}}
   \def\xpushpull{x_{-\frac{\tau}{2}, -\frac{\xi}{2}}}
   \def\xfpull{x_{-\frac{\Delta\nu}{2}}}
   \def\xfpush{x_{\frac{\Delta\nu}{2}}}
   \def\apull{a_{-}}
   \def\apush{a_{+}}

   \def\ytnu{y_{t,\nu}}
   \def\ytnut{\tilde{y}_{t,\nu}}
   \def\half{{\scriptstyle \frac{1}{2}}}
   \def\halftau{{\scriptstyle \frac{\tau}{2}}}
   \def\halfxi{{\scriptstyle \frac{\xi}{2}}}
   \def\halfnu{{\scriptstyle \frac{\nu}{2}}}
   \def\halfDnu{{\scriptstyle \frac{\Delta\nu}{2}}}

   \def\xtpull{x\left(t+\halftau\right)}
   \def\xtpush{x\left(t-\halftau\right)}
%   (shouldn't need this) \def\xtpullconj{x^*\left(t+\halftau\right)}
   \def\xtpushconj{x^*\left(t-\halftau\right)}
   \def\Xtpull{X\left(\nu+\halfxi\right)}
   \def\Xtpush{X\left(\nu-\halfxi\right)}
   \def\Xtpullconj{X^*\left(\nu+\halfxi\right)}
%   (shouldn't need this) \def\Xtpushconj{X^*\left(\nu-\halfxi\right)}

   \def\ytpull{y\left(t+\halftau\right)}
   \def\ytpush{y\left(t-\halftau\right)}
%   (shouldn't need this) \def\ytpullconj{y^*\left(t+\halftau\right)}
   \def\ytpushconj{y^*\left(t-\halftau\right)}
   \def\Ytpull{Y\left(\nu+\halfxi\right)}
   \def\Ytpush{Y\left(\nu-\halfxi\right)}
   \def\Ytpullconj{Y^*\left(\nu+\halfxi\right)}
%   (shouldn't need this) \def\Ytpushconj{Y^*\left(\nu-\halfxi\right)}

% Language
   \def\FT{Fourier transform}
   \def\WT{Wigner transform}
   \def\WV{Wigner-Ville}

%%-------------------------------------------------------------
%**end of header
%\includeonly{./theory/ambiguity,./theory/energydensity}

%           \begin{document}
%        Then add title with commands \title, \author, \maketitle,
%        \tableofcontents, etc.
%%-----------------------------------------------------------------
%**start of header
%\documentclass[notitlepage,12pt]{article}
\documentclass[11pt]{article}
%\documentclass{amsart}
%\usepackage{spconf,amsmath,epsfig}
\usepackage{amsmath,amssymb,amscd,epsfig}

\usepackage{fullpage}

\usepackage{ifthen}

%% Should figures be compiled?
\newboolean{nofigures} % initially false ==> compile with figures
% comment out the following line to compile with figures
%\setboolean{nofigures}{true} 

%% Should citation footnotes on section headings be compiled?
\newboolean{nofootnotes}
% comment out the following line to include all citation foonotes.
\setboolean{nofootnotes}{true} 

%\usepackage{boxedminipage}
\usepackage{theorem}
{\theorembodyfont{\rmfamily} 
 \newtheorem{definition}{Definition}[section]
 \newtheorem{lemma}{Lemma}[section] % Lemmas are for proving theorems and facts
 \newtheorem{prop}{Proposition}[section] % Props state any new results of this document
 \newtheorem{fact}{Fact}[section]   % Facts are for less important, well-known results
 \newtheorem{example}{Example}[section]   
 \newtheorem{theorem}{Theorem}[section]  % Theorems are for important, well-known results
 {\theoremstyle{marginbreak}  
   \newtheorem{define}{}  % This definition style is for the glossary
    \newtheorem{Theorem}{Theorem}[subsection]  % Theorems are for important, well-known results
 }
}
\renewcommand{\thedefine}{}

% THE FILE
%   /usr/share/texmf/doc/latex/hyperref/manual.pdf
%  documents the hyperref package
%\usepackage{hyperref}

% THE FILE
%   /usr/share/texmf/doc/latex/general/lshort.dvi (p.12)
%  documents the pagestyle command
%\pagestyle{headings}
% 2004.03.30: commented next six lines:
%\usepackage{fancyhdr}
%\pagestyle{fancy}
%\lhead{\thesection} \chead{} \rhead{\thepage} 
%\lfoot{Do Not Duplicate} \cfoot{\bfseries CONFIDENTIAL} \rfoot{Do Not Duplicate} 
%\renewcommand{\headrulewidth}{0pt} \renewcommand{\footrulewidth}{0.4pt}

% THE FILE
%   /usr/share/texmf/doc/pdflatex/base/example.tex
% suggests the following:
%\input pdfcolor.tex
%\pdfoutput=1\relax      % turn on PDF output; otherwise output DVI file
                         % this primitive cannot be specified  *after* shipping
                         % out the *first* page
%\pdfpagewidth=8.26in    % page width of PDF output
%\pdfpageheight=11.69in  % page height of PDF output
%\usepackage[pdftex]{graphicx}
%\DeclareGraphicsExtensions{.jpg,.pdf,.mps,.png}

%-------------------------------------------------------------
% Definitions.
% ------------
%\def\HOME{/home/moonchild/pub/research/dsp/DeMeo/Notes}
\def\HOME{/home/sonic/pub/research/dsp/DeMeo/Notes}
\def\R{\mathbb{R}}    % real numbers 
\def\C{\mathbb{C}}    % complex numbers 
\def\e{\mathrm{e}}    % exponential
% Two notations for real part of a complex number
   \def\Real{\mbox{Re}}  %\def\Real{\Re}
% Two notations for integration over R
   \def\integral{\int_{-\infty}^{\infty}}
     %\def\integral{\int_{-\infty}^{+\infty}} 
% Operator Theory
   \def\A{\operatorname{A}}
   \def\E{\operatorname{E}}
   \def\H{\operatorname{H}}
   \def\I{\operatorname{I}}
   \def\P{\operatorname{P}}
   \def\pv{\operatorname{pv}}
   \def\S{\operatorname{S}}
   \def\T{\operatorname{T}}
   \def\W{\operatorname{W}}
   \def\Hilbert{\mathcal{H}}
   \def\Hone{\mathcal{H}_1}
   \def\Htwo{\mathcal{H}_2}
   \def\Banach{\mathcal{B}(\Hilbert,\Hilbert)}
   \def\Banachonetwo{\mathcal{B}(\Hone,\Htwo)}
   \def\Banachtwoone{\mathcal{B}(\Htwo,\Hone)}
   \def\Lone{L^1}
   \def\Ltwo{L^2}
   \def\ltwo{\ell^2}
   \def\LoneR{L^1(\mathbb{R})}
   \def\LtwoR{L^2(\mathbb{R})}
   \def\ltwoR{\ell^2(\mathbb{R})}
   \def\Null{\mathcal{N}}  % nullspace
% Signals, Time-Frequency shifts, etc.
   \def\scale{\frac{1}{\sqrt{s}}}
   \def\transcale{\left(\frac{t-u}{s}\right)}
   \def\xpull{x_{\frac{\tau}{2}, -\frac{\xi}{2}}}
   \def\xpush{x_{-\frac{\tau}{2}, \frac{\xi}{2}}}
   \def\xpullpush{x_{\frac{\tau}{2}, \frac{\xi}{2}}}
   \def\xpushpull{x_{-\frac{\tau}{2}, -\frac{\xi}{2}}}
   \def\xfpull{x_{-\frac{\Delta\nu}{2}}}
   \def\xfpush{x_{\frac{\Delta\nu}{2}}}
   \def\apull{a_{-}}
   \def\apush{a_{+}}

   \def\ytnu{y_{t,\nu}}
   \def\ytnut{\tilde{y}_{t,\nu}}
   \def\half{{\scriptstyle \frac{1}{2}}}
   \def\halftau{{\scriptstyle \frac{\tau}{2}}}
   \def\halfxi{{\scriptstyle \frac{\xi}{2}}}
   \def\halfnu{{\scriptstyle \frac{\nu}{2}}}
   \def\halfDnu{{\scriptstyle \frac{\Delta\nu}{2}}}

   \def\xtpull{x\left(t+\halftau\right)}
   \def\xtpush{x\left(t-\halftau\right)}
%   (shouldn't need this) \def\xtpullconj{x^*\left(t+\halftau\right)}
   \def\xtpushconj{x^*\left(t-\halftau\right)}
   \def\Xtpull{X\left(\nu+\halfxi\right)}
   \def\Xtpush{X\left(\nu-\halfxi\right)}
   \def\Xtpullconj{X^*\left(\nu+\halfxi\right)}
%   (shouldn't need this) \def\Xtpushconj{X^*\left(\nu-\halfxi\right)}

   \def\ytpull{y\left(t+\halftau\right)}
   \def\ytpush{y\left(t-\halftau\right)}
%   (shouldn't need this) \def\ytpullconj{y^*\left(t+\halftau\right)}
   \def\ytpushconj{y^*\left(t-\halftau\right)}
   \def\Ytpull{Y\left(\nu+\halfxi\right)}
   \def\Ytpush{Y\left(\nu-\halfxi\right)}
   \def\Ytpullconj{Y^*\left(\nu+\halfxi\right)}
%   (shouldn't need this) \def\Ytpushconj{Y^*\left(\nu-\halfxi\right)}

% Language
   \def\FT{Fourier transform}
   \def\WT{Wigner transform}
   \def\WV{Wigner-Ville}

%%-------------------------------------------------------------
%**end of header
%\includeonly{./theory/ambiguity,./theory/energydensity}

%           \begin{document}
%        Then add title with commands \title, \author, \maketitle,
%        \tableofcontents, etc.
%%-----------------------------------------------------------------
%**start of header
%\documentclass[notitlepage,12pt]{article}
\documentclass[11pt]{article}
%\documentclass{amsart}
%\usepackage{spconf,amsmath,epsfig}
\usepackage{amsmath,amssymb,amscd,epsfig}

\usepackage{fullpage}

\usepackage{ifthen}

%% Should figures be compiled?
\newboolean{nofigures} % initially false ==> compile with figures
% comment out the following line to compile with figures
%\setboolean{nofigures}{true} 

%% Should citation footnotes on section headings be compiled?
\newboolean{nofootnotes}
% comment out the following line to include all citation foonotes.
\setboolean{nofootnotes}{true} 

%\usepackage{boxedminipage}
\usepackage{theorem}
{\theorembodyfont{\rmfamily} 
 \newtheorem{definition}{Definition}[section]
 \newtheorem{lemma}{Lemma}[section] % Lemmas are for proving theorems and facts
 \newtheorem{prop}{Proposition}[section] % Props state any new results of this document
 \newtheorem{fact}{Fact}[section]   % Facts are for less important, well-known results
 \newtheorem{example}{Example}[section]   
 \newtheorem{theorem}{Theorem}[section]  % Theorems are for important, well-known results
 {\theoremstyle{marginbreak}  
   \newtheorem{define}{}  % This definition style is for the glossary
    \newtheorem{Theorem}{Theorem}[subsection]  % Theorems are for important, well-known results
 }
}
\renewcommand{\thedefine}{}

% THE FILE
%   /usr/share/texmf/doc/latex/hyperref/manual.pdf
%  documents the hyperref package
%\usepackage{hyperref}

% THE FILE
%   /usr/share/texmf/doc/latex/general/lshort.dvi (p.12)
%  documents the pagestyle command
%\pagestyle{headings}
% 2004.03.30: commented next six lines:
%\usepackage{fancyhdr}
%\pagestyle{fancy}
%\lhead{\thesection} \chead{} \rhead{\thepage} 
%\lfoot{Do Not Duplicate} \cfoot{\bfseries CONFIDENTIAL} \rfoot{Do Not Duplicate} 
%\renewcommand{\headrulewidth}{0pt} \renewcommand{\footrulewidth}{0.4pt}

% THE FILE
%   /usr/share/texmf/doc/pdflatex/base/example.tex
% suggests the following:
%\input pdfcolor.tex
%\pdfoutput=1\relax      % turn on PDF output; otherwise output DVI file
                         % this primitive cannot be specified  *after* shipping
                         % out the *first* page
%\pdfpagewidth=8.26in    % page width of PDF output
%\pdfpageheight=11.69in  % page height of PDF output
%\usepackage[pdftex]{graphicx}
%\DeclareGraphicsExtensions{.jpg,.pdf,.mps,.png}

%-------------------------------------------------------------
% Definitions.
% ------------
%\def\HOME{/home/moonchild/pub/research/dsp/DeMeo/Notes}
\def\HOME{/home/sonic/pub/research/dsp/DeMeo/Notes}
\def\R{\mathbb{R}}    % real numbers 
\def\C{\mathbb{C}}    % complex numbers 
\def\e{\mathrm{e}}    % exponential
% Two notations for real part of a complex number
   \def\Real{\mbox{Re}}  %\def\Real{\Re}
% Two notations for integration over R
   \def\integral{\int_{-\infty}^{\infty}}
     %\def\integral{\int_{-\infty}^{+\infty}} 
% Operator Theory
   \def\A{\operatorname{A}}
   \def\E{\operatorname{E}}
   \def\H{\operatorname{H}}
   \def\I{\operatorname{I}}
   \def\P{\operatorname{P}}
   \def\pv{\operatorname{pv}}
   \def\S{\operatorname{S}}
   \def\T{\operatorname{T}}
   \def\W{\operatorname{W}}
   \def\Hilbert{\mathcal{H}}
   \def\Hone{\mathcal{H}_1}
   \def\Htwo{\mathcal{H}_2}
   \def\Banach{\mathcal{B}(\Hilbert,\Hilbert)}
   \def\Banachonetwo{\mathcal{B}(\Hone,\Htwo)}
   \def\Banachtwoone{\mathcal{B}(\Htwo,\Hone)}
   \def\Lone{L^1}
   \def\Ltwo{L^2}
   \def\ltwo{\ell^2}
   \def\LoneR{L^1(\mathbb{R})}
   \def\LtwoR{L^2(\mathbb{R})}
   \def\ltwoR{\ell^2(\mathbb{R})}
   \def\Null{\mathcal{N}}  % nullspace
% Signals, Time-Frequency shifts, etc.
   \def\scale{\frac{1}{\sqrt{s}}}
   \def\transcale{\left(\frac{t-u}{s}\right)}
   \def\xpull{x_{\frac{\tau}{2}, -\frac{\xi}{2}}}
   \def\xpush{x_{-\frac{\tau}{2}, \frac{\xi}{2}}}
   \def\xpullpush{x_{\frac{\tau}{2}, \frac{\xi}{2}}}
   \def\xpushpull{x_{-\frac{\tau}{2}, -\frac{\xi}{2}}}
   \def\xfpull{x_{-\frac{\Delta\nu}{2}}}
   \def\xfpush{x_{\frac{\Delta\nu}{2}}}
   \def\apull{a_{-}}
   \def\apush{a_{+}}

   \def\ytnu{y_{t,\nu}}
   \def\ytnut{\tilde{y}_{t,\nu}}
   \def\half{{\scriptstyle \frac{1}{2}}}
   \def\halftau{{\scriptstyle \frac{\tau}{2}}}
   \def\halfxi{{\scriptstyle \frac{\xi}{2}}}
   \def\halfnu{{\scriptstyle \frac{\nu}{2}}}
   \def\halfDnu{{\scriptstyle \frac{\Delta\nu}{2}}}

   \def\xtpull{x\left(t+\halftau\right)}
   \def\xtpush{x\left(t-\halftau\right)}
%   (shouldn't need this) \def\xtpullconj{x^*\left(t+\halftau\right)}
   \def\xtpushconj{x^*\left(t-\halftau\right)}
   \def\Xtpull{X\left(\nu+\halfxi\right)}
   \def\Xtpush{X\left(\nu-\halfxi\right)}
   \def\Xtpullconj{X^*\left(\nu+\halfxi\right)}
%   (shouldn't need this) \def\Xtpushconj{X^*\left(\nu-\halfxi\right)}

   \def\ytpull{y\left(t+\halftau\right)}
   \def\ytpush{y\left(t-\halftau\right)}
%   (shouldn't need this) \def\ytpullconj{y^*\left(t+\halftau\right)}
   \def\ytpushconj{y^*\left(t-\halftau\right)}
   \def\Ytpull{Y\left(\nu+\halfxi\right)}
   \def\Ytpush{Y\left(\nu-\halfxi\right)}
   \def\Ytpullconj{Y^*\left(\nu+\halfxi\right)}
%   (shouldn't need this) \def\Ytpushconj{Y^*\left(\nu-\halfxi\right)}

% Language
   \def\FT{Fourier transform}
   \def\WT{Wigner transform}
   \def\WV{Wigner-Ville}

%%-------------------------------------------------------------
%**end of header
%\includeonly{./theory/ambiguity,./theory/energydensity}
