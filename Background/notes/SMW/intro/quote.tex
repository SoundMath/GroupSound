\subsection{Introduction\protect\footnotemark}
\footnotetext{The quote is from Mallat~\cite{Mallat:1998} p.~1.}  
\begin{quote}
``After a few minutes in a restaurant we cease to notice the annoying
hubbub of surrounding conversations, but a sudden silence reminds us
of the presence of neighbors.  Our attention is clearly attracted by
transients and movements as opposed to stationary stimuli, which we
soon ignore.  Concentrating on transients is probably a strategy for
selecting important information from the overwhelming amount of data
recorded by our senses. Yet, classical signal processing has devoted
most of its efforts to the design of time-invariant and
space-invariant operators, that modify stationary signal properties.
This has led to the indisputable hegemony of the Fourier transform,
but leaves aside many information-processing applications.

The world of transients is considerably larger and more complex than
the garden of stationary signals.  The search for an ideal
Fourier-like basis that would simplify most signal processing is
therefore a hopeless quest.  Instead, a multitude of different
transforms and bases have proliferated, among which wavelets are just
one example.''
\end{quote}

                                                                            