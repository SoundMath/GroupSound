% -*- mode: LaTeX; tex-main-file: "../Notes.tex"; -*-
% (begin: insert file ambiguity.tex)
%\subsection{Quadratic Time-Frequency Energy\protect\footnotemark}
%\footnotetext{Mallat~\cite{Mallat:1998} (p.107),
%Mertins~\cite{Mertins:1999} (p.265)} 
\section{Energy Distributions}
\label{sec:energy}
Wavelet and windowed Fourier transforms are computed by
correlating the signal with families of time-frequency atoms.  The
time and frequency resolution of these transforms is thus limited by
the time-frequency resolution of the corresponding atoms.  Ideally,
one would like to define a density of energy in a time-frequency
plane with no loss of resolution.  This section presents a
different class of time-frequency representation (TFR) which is not
restricted by the uncertainty principle.  

%\footnotetext{Mertins~\cite{Mertins:1999} (p.265)}
The \emph{Wigner-Ville time-frequency representation} is computed by
correlating $x$ with a time and frequency translation of itself.
(Below we refer to the Wigner-Ville TFR simply as the ``\WV'' and,
sometimes, as the ``\WT.'')  Though it yields some remarkable
properties, the quadratic from of this representation can also limit
its application because of the inevitable cross terms that appear in 
quadratic forms.  In such cases, these so-called ``interference
terms'' can be attenuated by a time-frequency averaging, but this
procedure results in a loss of resolution. Below we see that the
spectrogram, the scalogram and all squared time-frequency
decompositions can be written as a time-frequency averaging of the
\WT. 

Although the TFR's of this section do not yield positive
distributions in all cases, they allow extremely good insight into
signal properties within certain applications.  

\ifthenelse{\boolean{nofootnotes}}{\subsection{Ambiguity Function}}
{\subsection{Ambiguity Function\protect\footnotemark}
\footnotetext{Mertins~\cite{Mertins:1999} (p.265)}}
\label{sec:ambig-funct}
This section describes TFR's that are computed by correlating a signal,
$x(t)$, with time and frequency shifted versions of itself.  We start
by looking at time and frequency shifts separately.

\begin{define}{\bf Time-Shifted Signals. }  We define the 
\emph{temporal autocorrelation function}, $r_{x}(\tau)$, by
correlating $x(t)$ with a time translation of itself:
%\begin{align}
\begin{equation}
\label{eqn:autocorrelation} 
r_{x}(\tau) = \langle x_\tau,x \rangle 
             = \integral x(t+\tau)x^*(t)\,dt\\
%&= \integral x(t)x^*(t-\tau)\,dt = \langle x,x_{-\tau} \rangle 
%\end{align}
\end{equation}
\end{define}
A simple change of variable %in (\ref{eqn:autocorrelation}) 
shows that $r_{x}(\tau) = \langle x,x_{-\tau} \rangle$.

The distance $d(x, x_\tau)$ between an energy signal $x(t)$ and its
time-shifted version $x_\tau(t) = x(t+\tau)$ is related to %$r_x(\tau)$ 
the autocorrelation function according to the following equation: 
\[
d(x, x_\tau)^2 = 2\|x\|^2 - 2 \,\Real[r_{x}(\tau)]
\] 

Let $Y(\omega)$ denote the Fourier transform of $y(t)$ and recall the
following correspondences:\\
\begin{center}
\begin{tabular}{l|c|c}
&$y(t)$&$Y(\omega)$\\
\hline
translation & $x(t-t_0)$ & $e^{-i2\pi\omega t_0}X(\omega)$\\
modulation & $e^{i2\pi\omega_0t}x(t)$ & $X(\omega-\omega_0)$\\
conjugation & $x^*(t)$ &$ X^*(-\omega)$
\end{tabular}
\end{center}
\vspace{4mm}
Also, recall Parseval's relation from section~\ref{sec:Parseval}: 
$\langle X,Y\rangle = \langle x,y\rangle$. 
Now, if we put $y(t) = x(t+\tau)$ in
equation~(\ref{eqn:autocorrelation}), it is easy to see that
$r_{x}(\tau) = \langle y,x\rangle = \langle Y,X\rangle$.  More
explicitly, we have 
\begin{eqnarray}
r_{x}(\tau)&=&\integral Y(\omega) X^*(\omega)\,d\omega \nonumber \\
&=&\integral e^{i2\pi\omega\tau}X(\omega) X^*(\omega)\,d\omega\nonumber\\
%&=& \integral S_{x}(\omega)e^{i2\pi \omega \tau}\,d\omega 
&=& \integral  |X(\omega)|^2 \,e^{i2\pi \omega \tau}\,d\omega 
\label{eq:6}
\end{eqnarray}
%where $S_{x}(\omega) = |X(\omega)|^2$ 
where $|X(\omega)|^2$ is the \emph{spectral energy density.}  Thus,
inverting~(\ref{eq:6}), we see that the spectral energy density is the
\FT\ of the temporal autocorrelation function.

\begin{define}{\bf Frequency-Shifted Signals. }  Frequency shifted
versions of a signal $x(t)$ are often produced due to the Doppler
effect.  If we wish to estimate such a frequency shift in order to
determine the velocity of a  moving object, we consider the distance
between a signal $x(t)$ and its frequency-shifted, or
\emph{modulated}, version $x_\xi(t) = x(t)e^{i2\pi\xi t}$. 
The distance is given by
\[
d(x, x_\xi)^2 = 2\|x\|^2 - 2 \,\Real\langle x_\xi,x\rangle
\] 
\end{define}
The foregoing inner product will be denoted $\rho_{x}(\xi)$.  Thus,
\begin{align*}
\rho_{x}(\xi)&= \langle x_\xi,x \rangle 
              = \langle x,x_{-\xi}\rangle\\
             &= \integral e^{i2\pi\xi t} x(t)x^*(t)\,dt\\
%&=& \integral s_{x}(t)e^{i\xi t}\,dt 
             &= \integral |x(t)|^2 \,e^{i2\pi\xi t}\,dt 
\end{align*}
%where $s_{x}(t) = |x(t)|^2$ 
where $|x(t)|^2$ is the \emph{temporal energy density.}  It is also
called the \emph{instantaneous power} of the signal.

It is possible to view $\rho_{x}(\xi)$ as the spectral analog of the
temporal autocorrelation function, $r_{x}(\tau)$, when we write the
former in as a \emph{spectral autocorrelation function.}  That is,
using the Fourier relations reviewed above and the Parseval identity,
% we can write 
\[
\rho_{x}(\xi) = \integral X(\omega - \xi)X^*(\omega)\,d\omega
\]
In other words, $\rho_{x}(\xi)$ is the autocorrelation function of the
spectrum $X(\omega)$.

\begin{define}{\bf Time and Frequency-Shifted Signals. }  
We now consider shifting a signal in both the time and frequency domains
simultaneously.  For this, a transform known as the 
%following transform is often useful.
%\end{define}
%\begin{definition}{\bf Ambiguity Function. } The 
\emph{ambiguity function} plays a central role as the 
\emph{time-frequency autocorrelation function}.  
It is defined as follows: 
\[
\A_{x}(\xi,\tau) = \integral \xtpull \xtpushconj e^{i2\pi \xi t}\,dt
\]
Once again, Parseval provides for the definition as a frequency
integration
%on the frequency domain,  
\[
\A_{x}(\xi,\tau) = \integral \Xtpush \Xtpullconj e^{i2\pi \nu \tau}\,d\nu
\]
\end{define}
Analogous to the cross correlation function is the
\emph{cross ambiguity function} defined by 
\begin{eqnarray}
\label{eqn:crossAmbiguity}
  \A_{xy}(\xi,\tau) 
   &=& \integral \xtpull \ytpushconj e^{i2\pi \xi t}\,dt\\
   &=& \integral \Xtpush \Ytpullconj e^{i2\pi \xi t}\,dt\nonumber \\
\end{eqnarray}
% (end: insert file ambiguity.tex)

At this point it is helpful to summarize what we have shown thus
far.  The functions $r_x(\tau)$ and $\rho_x(\xi)$ are the
autocorrelation functions associated with shifts in time and frequency,
respectively.  The \emph{spectral} energy density is the \FT\ of the
\emph{temporal} autocorrelation function:
\[
\left|X(\nu)\right|^2 = \integral r_x(\tau)e^{-i2\pi\nu\tau}\,d\tau
\]
The \emph{temporal} energy density is the \FT\ of the
\emph{spectral} autocorrelation function:
\[
\left|x(t)\right|^2 = \integral \rho_x(\xi)e^{-i2\pi\xi t}\,d\xi
\]
The spectral and temporal energy densities can be viewed as
``marginal'' energy densities in their respective variables.
The ambiguity function generalizes the ``marginal'' autocorrelation
functions, $r_x(t)$ and $\rho_x(\xi)$, to simultaneously account for
both time and frequency shifts.  
Thus, it is the ``joint'' autocorrelation function.  The
natural next step, then, is to consider the \FT\ of the ambiguity
function.  Perhaps this defines a useful ``joint'' energy density,
with the correct marginal densities, 
$\left|X(\nu)\right|^2$ and $\left|x(t)\right|^2$.  We define such a 
transform in the next section, and relate it to the foregoing
interpretation as a ``joint energy density'' in
section~\ref{sec:wign-ambig-relat}.

% (begin: insert file Wigner.tex)
\ifthenelse{\boolean{nofootnotes}}{\subsection{Wigner Transform}}
{\subsection{Wigner Transform\protect\footnotemark}
\footnotetext{Mertins~\cite{Mertins:1999} (p.265)}}
%\begin{define}{\bf Wigner Transform. } The quadratic form 
The ambiguity function represents the signal energy %of the signal 
in the {\it frequency-time} plane.  We now transform this
representation into the {\it time-frequency} plane.
%The \WV\ transform is also computed by integrating the product of $x$
%with time and frequency translations of itself.  However, the
%variable over which we integrate is now the shift variable.  

The quadratic form
\begin{equation} \label{eqn:WignerVille}
\W_x(t,\nu) = 
\integral \xtpull \xtpushconj e^{-i2\pi \tau\nu}\,d\tau
\end{equation}
is known as the \emph{Wigner-Ville distribution}, or \emph{\WT}.
As we show in the next section, it is the two-dimensional \FT\ of the
ambiguity function.  Also, as the one-dimensional \FT\ of 
$\xtpushconj \xtpull$, which has a Hermitian symmetry in $\tau$, the
\WV\ transform is real valued. 

Time and frequency play a symmetric role, % Mallat:1998 (p.108)
and the transform can be written as a frequency integral by
applying the Parseval formula:
\begin{equation} \label{eqn:WignerVilleF}
\W_x(t,\nu) = %\frac{1}{2\pi}
\integral \Xtpush \Xtpullconj \e^{-i2\pi\xi t}\,dt
%\hat{f}\left(\xi+\frac{\gamma}{2}\right)
%\hat{f}^*\left(\xi-\frac{\gamma}{2}\right)
\end{equation}

%\begin{define}{\bf Time-Frequency Support. }  
The \WV\ transform localizes the time-frequency structures of $x$.  If
the energy of $x$ is well concentrated in time around $t_0$ and in
frequency around $\nu_0$ then $\W_x$ has its energy centered at
$(t_0,\nu_0)$, with a spread equal to the time and frequency spread of
$x$. 
%\end{define}

% Mallat:1998 (p.109)
\begin{define} {\bf Instantaneous Frequency. }
%\footnote{Mallat~\cite{Mallat:1998} (p.109)}
Ville's original motivation for studying time-frequency decompositions
was to compute the instantaneous frequency of a signal.  Assume $x$
has the form $x(t) = a(t)\cos\phi(t)$ and let $x_a$ be the analytic
part of $x$ obtained by setting $X(\nu)=0$ for $\nu < 0$.  Equivalently,
$x_a(t) = a(t)\exp[i\phi(t)]$ so that $x = \Real[x_a]$, and
$\nu=\phi'(t)$ is the instantaneous frequency.  The following
proposition states that $\phi'(t)$ is the ``average'' frequency
computed relative to the transform $\W_x$.
\end{define}
\begin{prop}
If $x_a(t) =  a(t)\exp[i\phi(t)]$, then
\[
  \phi'(t) =
  \frac{\int \nu \W_{x_a}(t,\nu)\,d\nu}
       {\int \W_{x_a}(t,\nu)\,d\nu}
\]
where integration is over all $\R$.
\end{prop}
This proposition shows that, for a fixed $t=t_0$, the mass of
$\W_{x_a}(t_0,\nu)$ is typically concentrated in the neighborhood of the
instantaneous frequency $\nu_0 = \left.\phi'(t)\right|_{t=t_0}$.
% (end: insert file Wigner.tex)

% (begin: insert file WignerAmbiguity.tex)
\subsection{Wigner-Ambiguity Relations}\label{sec:wign-ambig-relat}
%\begin{define} {\bf Wigner-Ambiguity Relations. }
To help further motivate the definition of the \WV, we relate it
%let us consider its relation 
to other entities, about which we might have better intuition.  
First, let us simplify notation by introducing the signals
\begin{equation}\label{eq:8}%alignat}{2}
\xpullpush(t) = \xtpull e^{i2\pi \halfxi t},\qquad 
\xpushpull(t) = \xtpush e^{-i2\pi \halfxi t}%\label{eq:8}
\end{equation}
\begin{equation}\label{eq:9}%alignat}{2}
\ytnu(\tau)  = \xtpull e^{-i2\pi \nu \halftau},\qquad 
\ytnu(-\tau) = \xtpush e^{i2\pi \nu \halftau}%\nonumber\\
\end{equation}
which are time and frequency shifted versions of 
%one another, centered around 
$x(t)$. Note that, because of its role in defining the \WV\ transform,
the function $\ytnu$ is defined as a function of the delay variable
$\tau$. 
% The following diagram displays these shifted signals in
%relation to one another (the arrows point in the directions of
%positive time translation and frequency modulation):
%\[\begin{CD} \xpullpush @> +\tau >> \ytnut\\
%@A\text{freq.}AA  @AA+\xi A \\
%\ytnu @>> \text{ time }> \xpushpull
%\end{CD}\]

The notations in~(\ref{eq:8}) and~(\ref{eq:9}) allow us to define the
ambiguity function and the \WV\ as inner products,
\begin{eqnarray}%\label{eqn:AmbiguityDistance}
\A_{x}(\xi,\tau) &=& \langle \xpullpush,\xpushpull \rangle\\
%\label{eqn:WignerDistance}
\W_{x}(t,\nu) &=& \langle \ytnu, \ytnut \rangle
\end{eqnarray}
where $\ytnut(\tau) = \ytnu(-\tau)$.
This notation elucidates the role of the transforms
% Wigner transform and ambiguity function, 
%$\A_{x}$ renders transparent 
in relation to the distance between shifted signals.
%~(\ref{eqn:xpull}) and~(\ref{eqn:xpush}): 
\begin{eqnarray*}
%d\left(\xpushpull,\xpullpush\right) &=& 
d(\xpushpull,\xpullpush) &=& 
    2\|x\|^2 - 2 \,\Real\langle \xpullpush,\xpushpull \rangle\\
&=& 2\|x\|^2 - 2 \,\Real[\A_{x}(\xi,\tau)]
\end{eqnarray*}
\begin{eqnarray*}
%d\left(\xpush,\xpull\right) &=& 
d(\ytnut,\ytnu) &=& 
    2\|x\|^2 - 2 \,\Real\langle \ytnu,\ytnut \rangle\\
&=& 2\|x\|^2 - 2 \,\Real[\W_{x}(t,\nu)]
\end{eqnarray*}

Next, recall the temporal and spectral autocorrelation functions of
section~\ref{sec:ambig-funct}
\begin{alignat*}{2}
r_{x}(\tau)   &= \langle x_\tau,x\rangle 
              &&= \integral x(t+\tau)x^*(t)\,dt\\
\rho_{x}(\xi) &= \langle x_\xi, x\rangle 
              &&= \integral X(\omega - \xi)X^*(\omega)\,d\omega
\end{alignat*}
and restrict the ambiguity function % $\A_x(\xi,\tau)$ 
to the ``delay axis,'' $\{(\xi,\tau): \xi=0\}$ -- that is, fix the 
%coordinates $(\xi,\tau)$ for which the 
Doppler variable at $\xi \equiv 0$.
On the delay axis the ambiguity function is identical to the temporal
autocorrelation function: 
\[
   \A_x(0,\tau) = r_x(\tau)
\]
From this we derive the spectral energy density by means of the following
\FT :
\begin{eqnarray*}
%S_x(\nu) 
|X(\nu)|^2&=& \integral r_x(\tau) e^{-i2\pi\nu\tau}\,d\tau\\
         &=& \integral \A_x(0,\tau) e^{-i2\pi\nu\tau}\,d\tau
\end{eqnarray*}
On the other hand, by projecting $\A_x(\xi,\tau)$ onto the Doppler
axis, where $(\xi,\tau) \equiv (\xi,0)$, we arrive at the
autocorrelation function 
%$\rho_x(\xi)$ 
of the spectrum:
%$X(\nu)$:
\[
   \A_x(\xi,0) = \rho_x(\xi)
\]
The temporal energy density %$s_x(t)$ 
$|x(t)|^2$ is the \FT\ of
$\rho_x(\nu)$:
\begin{eqnarray*}
%s_x(t) 
|x(t)|^2&=& \integral \rho_x(\nu) e^{-i2\pi\xi t}\,d\xi\\
       &=& \integral \A_x(\nu,0) e^{-i2\pi\xi t}\,d\xi
\end{eqnarray*}
On the entire frequency-time plane, the \FT\ with respect to $\xi$
yields the product: 
%temporal autocorrelation function\footnote{If $x(t)$ was assumed to
%be a random process, $\E\{\phi_x(t,\tau)\}$ would be the
%autocorrelation function of the process.} 
\begin{eqnarray}
\phi_x(t,\tau) &=& 
    \integral \A_x(\xi,\tau) e^{-i2\pi\xi t}\,d\xi\nonumber\\
&=& \xtpull \xtpushconj \label{eq:phi}
\end{eqnarray}
The \FT\ with respect to $\tau$ yields%of $\A_x(\xi,\tau)$ 
\begin{eqnarray}
\Phi_x(\xi,\nu) &=& 
    \integral \A_x(\xi,\tau) e^{-i2\pi\nu\tau}\,d\tau \nonumber\\
&=& \Xtpush \Xtpullconj \label{eq:Phi}
\end{eqnarray}
%The function $\Phi_x(\xi,\nu)$ is the autocorrelation function of
%$X(\nu)$ -- that is, the spectral autocorrelation function.
By equations~(\ref{eq:phi}) and~(\ref{eq:Phi}), and the definition of
the \WV\ transform (\ref{eqn:WignerVille}), 
%it is now clear that its relation to the ambiguity function is given by 
we have the following two-dimensional \FT: 
\begin{equation}\label{eq:ambig-ft}
\W_x(t,\nu) = 
  \iint\limits_{\R^2} 
    \A_x(\xi,\tau) e^{-i2\pi(\xi t + \nu\tau)}\,d\xi\,d\tau
\end{equation}
%\end{define}
%This 2-D Fourier transform can be viewed as two subsequent
%1-D Fourier transforms with respect to $\xi$ and $\tau$.  

The foregoing also makes clear the connections between the \WV\
transform and the energy densities $|x(t)|^2$ and $|X(\nu)|^2$.
Indeed, we have the Fourier relations
\begin{eqnarray}
\phi_x(t,\tau) &=& 
    \integral \W_x(t,\nu) e^{i2\pi\nu \tau}\,d\nu\nonumber \\
\Phi_x(\xi,\nu) &=& 
    \integral \W_x(t,\nu) e^{i2\pi\xi t}\,dt\label{eq:7}
\end{eqnarray}
Restricting $\phi(t,\tau)$ to the set $\{(t,\tau): \tau=0\}$ in 
equation~(\ref{eq:phi}) yields
\[%\begin{eqnarray*}
|x(t)|^2 =\phi_x(t,0) = \integral \W_x(t,\nu)\,d\nu
\]%\end{eqnarray*}
Similarly, %restricting $\Phi(\xi,\nu)$ to $\{(\xi,\nu): \xi=0\}$ in
equations~(\ref{eq:phi}) and~(\ref{eq:7}) imply
\[%\begin{eqnarray*}
|X(\nu)|^2 =\Phi_x(0,\nu)= \integral \W_x(t,\nu)\,dt
\]%\end{eqnarray*}
Thus we have proved that the ``marginal'' temporal energy density is
recovered by integrating the Wigner transform with respect to the
frequency variable.  Similarly, the ``marginal'' spectral energy
density is recovered by an integration with respect to time. 
%\begin{prop} For any $x\in\LtwoR$
%\[\integral \W_x(t,\nu)\,dt = |X(\omega)|^2\]
%\[\integral \W_x(t,\nu)\,d\nu = |x(t)|^2\]
%\end{prop}
Therefore, it is intuitively appealing to call $\W_x$ a ``joint
time-frequency energy density.''  
However, the \WV\ transform misses one fundamental property of a density
function -- positivity.  In general, the \WV\ is an oscillating
function that takes on negative values.  In fact, one can prove that
translated and frequency modulated Gaussians are the only functions
whose \WV\ transforms remain positive. 

The negative values, popularly called ``interferences,'' are created
by the quadratic properties of the \WV\ transform.  These
interferences can be attenuated or removed by averaging the transform
with appropriate kernels which yield positive time-frequency
densities.  However, this reduces the time-frequency resolution.  The
spectrograms of Fourier analyses as well as the scalograms of wavelet
analyses are examples of positive quadratic densities obtained by
smoothing the \WT. In fact, for any family $\{\phi_{\gamma}\}$ of
time-frequency atoms, and the associated  transform $\T[x]$, the
energy density $E_{T[x]}(t,\nu)$, is an averaging of the \WT.    
We will consider these ideas more carefully in the following sections.
%, but first let's study the connections between the \WT, the ambiguity
%function, and the ``marginal'' energy densities.

We conclude this section by summarizing the relationships between the
\WV, ambiguity, and (deterministic)\footnote{If $x(t)$ was assumed to
be a random process, $\E\{\phi_x(t,\tau)\}$ would be the
autocorrelation function of the process.} autocorrelation functions
%are sumarized by the diagram below, 
using the following diagram:
\begin{equation}%\label{dgm:params}
\begin{CD}
\phi_x(t,\tau)  @<\mathcal{F}<< A_x(\xi,\tau)\\
@V\mathcal{F}VV             @VV\mathcal{F}V \\
W_x(t,\nu) @<<\mathcal{F}< \Phi_x(\xi,\nu)
\end{CD}
\end{equation}
Here $\mathcal{F}$ indicates that the Fourier transform operates in
the direction of the arrow.

The Diagram %(\ref{dgm:params}) 
is helpful for remembering the Fourier
relationships because it agrees with the standard geometrical
framework in which a change in the first (resp.~second) coordinate
corresponds to movement along the horizontal (resp.~vertical) axis.
However, this interpretation cannot be taken literally since movement
along each axis really corresponds to an entire change of coordinates.
The next diagram shows the coordinate systems in which the
associated objects of the previous diagram represent the signal
energy: 
%$x(t)$:  
\[
\begin{CD}
%\text{ Time-Time }(t,\tau)  @<\mathcal{F}<< \text{ Frequency-Time: }
%(\xi,\tau)\\
\text{ time-time }  @<\mathcal{F}<< \text{ frequency-time }\\
@V\mathcal{F}VV             @VV\mathcal{F}V \\
%\text{ Time-Frequency: } (t,\nu) @<<\mathcal{F}< 
%\text{ Frequency-Frequency: } (\xi,\nu)
\text{ time-frequency } @<<\mathcal{F}< \text{ frequency-frequency }
\end{CD}
\]
For further insight into the different perspectives that each of these
parameterizations provide, we refer the reader to Flandrin's excellent 
treatment in~\cite{Flandrin:1999}, pages~187--194.
% (end: insert file WignerAmbiguity.tex)

% (begin: insert file interference.tex)
\subsection{Interference Structure}% of Bilinear Transforms}
% (begin: inserted from file energydensity.tex)
% Mallat (p.112), Flandrin (p.227)
%\begin{define}{\bf Cross Terms of the Wigner Transform}
Because the \WV\ transform is a sesquilinear form of the signal, it
cannot submit to the principle of linear superposition.  As in
the quadratic equation,
\[
(a+b)^2 = a^2 + b^2 + ab + ba
\]
it is easy to verify that
%Let $x=x_1+x_2$ be a composite signal.  Since the \WV\ transform is a
%quadratic form, \[\W_x = \W_{x_1} + \W_{x_2} +\W_{x_1x_2}+\W_{x_2x_1}\]
\begin{equation}
\label{eqn:Wignersum} 
  \W_{x+y}(t,\nu) = \W_x(t,\nu) + \W_y(t,\nu) +
                    \W_{xy}(t,\nu) +\W_{yx}(t,\nu) 
\end{equation}
where $\W_{xy}$ is the cross \WV\ transform of two signals, defined by
%\[\W_{xy}(t,\nu) = \integral x\left(t+\frac{\tau}{2}\right) 
%y^*\left(t-\frac{\tau}{2}\right)\e^{-i2\pi\nu\tau}d\tau\]
\begin{eqnarray}
\label{eqn:crossWigner}
  \W_{xy}(t,\nu) 
  &=& \integral \xtpull \ytpushconj e^{-i2\pi\nu\tau}\,d\tau\\
  &=& \integral \Xtpush \Ytpullconj e^{-i2\pi\xi t}\,d\xi\nonumber
\end{eqnarray}
%\end{define}
It can also be shown that~(\ref{eqn:crossWigner}) is the
two-dimensional \FT\ of the cross ambiguity function defined above in
equation~(\ref{eqn:crossAmbiguity}).

We define the \emph{interference term} of
equation~(\ref{eqn:Wignersum}) by
%\[I_{x_1x_2} = \W_{x_1x_2} + \W_{x_2x_1}\]
\begin{eqnarray*}
  I_{xy}(t,\nu) &=& \W_{xy}(t,\nu) + \W_{yx}(t,\nu) \\
                 &=& 2\,\Real\left[\W_{xy}(t,\nu)\right]
\end{eqnarray*}
It is a real valued function that creates non-zero values at
unexpected locations of the time-frequency %$(t,\nu)$ 
plane. In general, for any linear combination of signal components,
\[
  x(t) = \sum_{n=1}^N a_n x_n(t)
\]
the \WV\ transform is
\begin{equation}\label{eq:WignerSum} 
  \W_{x} = \sum_{n=1}^N|a_n|^2 \W_{x_n}(t,\nu) + 
                 2\sum_{n=1}^{N-1}\sum_{k=n+1}^N\,
                 \Real\left[a_n a_k^* \W_{x_nx_k}(t,\nu)\right]
\end{equation}
Hence, for a signal with $N$ components, the \WV\ transform contains 
$N(N-1)/2$
additional components.  They result from the interaction of different
components of the signal, and are called ``interference terms'' for
two reasons.  First, the mechanism of their creation is analogous to
the usual interference, which can be observed for physical waves.  A
second reason for this terminology lies in the disturbing
effect that these terms can have on the %concerning the readability of a
time-frequency diagram of the signal energy.  As they amount to a
combinatorial proliferation of additional, ``specious'' signal
components, they can inhibit our ability to discern ``true'' signal 
components in the diagram.  

The presence of cross terms in a \WV\ transform can be regarded as a
natural consequence of its bilinear structure.  On the other hand,
this very structure also leads to most of the good properties of the
transform (such as localization).   
%Although it first looks like a drawback, the presence of interference 
%terms is the price to pay for gaining other advantages.
No matter whether one views the cross terms as helpful or hindering,
it is important to understand fully the mechanism of their creation.
This is indispensable for drawing the correct interpretation from the 
representation of an unknown signal, and for reducing the importance
of these terms if desired.  In our present work, we study the cross
terms in order to understand how this measure of signal interference
relates to measures of ``musical interference,'' e.g.~dissonance. 

%As a simple example, consider a well localized time-frequency
%atom, $x(t)$, centered at $t=0$, and the two atoms which are time and 
%frequency shifted versions of $x(t)$: 
%\[
%\xpull(t) = 
% \apull \xtpull e^{-i2\pi (\halfxi) t} \quad \text{ and }\quad  
%\xpush(t) = \apush \xtpush e^{i2\pi (\halfxi) t}
%\]

The canonical example used to describe the structure of the 
%\WV\ transform
cross terms begins with a well localized time-frequency atom $x(t)$
centered at $t=0$.  From this we construct two atoms which are time
and frequency shifted versions of $x(t)$,
\begin{alignat*}{2}%\[
x_1(t)&=a_1 x(t-t_1) e^{i2\pi(\nu_1t+\phi_1)}, \qquad a_1 &\geq 0\\
x_2(t)&=a_2 x(t-t_2) e^{i2\pi(\nu_2t+\phi_2)}, \qquad a_2 &\geq 0
\end{alignat*}%\]
Now consider the \WV\ transform of the composite signal $x_1(t)+x_2(t)$,
\[
  \W_{x_1+x_2}(t,\nu) 
    = \W_{x_1}(t,\nu)+\W_{x_2}(t,\nu) + I_{x_1x_2}(t,\nu)
\]
The \emph{covariance property} of the \WV\ transform ensures that
the shifted atoms, taken individually, have Wigner representations
given by  
\begin{align*}
\W_{x_1}(t,\nu) &= a^2_1\W_x(t-t_1,\nu-\nu_1)\\
\W_{x_2}(t,\nu) &= a^2_2\W_x(t-t_2,\nu-\nu_2)
\end{align*}
Since the energy of $\W_x$ is centered at $(0,0)$, the energy of
$\W_{x_1}$ and $\W_{x_2}$ is concentrated in the neighborhoods
$(t_1,\nu_1)$ and $(t_2,\nu_2)$, respectively.  A direct calculation
verifies that the interference term is
%\[I_{x_1x_2}(t,\nu)=2a_1a_2\W_x(t-t_m,\nu-\nu_m)
%\cos[\vartheta(t_1,t_2,\nu_1,\nu_2,\phi_1,\phi_2)]\]
%\cos\vartheta\]
%where
%\[\vartheta \equiv (t-t_m)\Delta \nu-(\nu-\nu_m)\Delta t +\Delta \phi\]
%\left[(t-t_m)\Delta \nu-(\nu-\nu_m)\Delta t +\Delta \phi \right]\]
\[
  I_{x_1,x_2}(t,\nu)=
    2a_1a_2\W_x(t-t_m,\nu-\nu_m)
    \cos\left\{2\pi \left[(t-t_m)\Delta\nu 
               -(\nu-\nu_m)\Delta t +\Delta\phi\right]\right\}
\]
where %and
\begin{alignat*}{2}%\[%begin{eqnarray*}
  t_m    &= \frac{t_1+t_2}{2}, \quad 
  \nu_m &&= \frac{\nu_1+\nu_2}{2}\\
  \Delta t    &= t_1-t_2,\quad
  \Delta \nu &&= \nu_1-\nu_2
\end{alignat*}
\[\Delta \phi = \phi_1-\phi_2 + t_m\Delta \nu\]
This is an oscillatory waveform concentrated in a neighborhood of the
point in the time-frequency plane that is the geometric midpoint 
%, $(t_m,\nu_m)$,
between the individual components.  The frequency of the oscillations
is proportional to the Euclidean distance
$\sqrt{\Delta \nu^2 + \Delta t^2}$ that separates the points
$(t_1,\nu_1)$ and $(t_2,\nu_2)$, where the individual atoms are
concentrated.  The direction of these oscillations is perpendicular to
the line that joins these two center points.
% $(t_1,\nu_1)$ and $(t_2,\nu_2)$.

\begin{define}{\bf Physical Interpretation. }  It is possible to attach
physical meaning to the interference structure of the \WV\ transform.
For the most basic case, in which the signal is a simple superposition
of pure frequencies, the cross term % of the \WV\ 
can be regarded as a signature of the \emph{beat frequency} resulting
from the interaction between %coexistence of 
the individual frequencies.  For example, suppose we start with a pure
sinusoidal $x(t) = e^{i2\pi\nu_m t}$ at the (mid-point) frequency
$\nu_m$.  Then consider the two frequency shifted versions of $x(t)$,
\[%\begin{align*}
x_1(t)%=\xfpull(t) = x(t)e^{-i2\pi(\halfDnu)t}\\
      =e^{i2\pi(\nu_m-\halfDnu)t},\qquad
x_2(t)%=\xfpush(t) = x(t)e^{i2\pi(\halfDnu)t}\\
      = e^{i2\pi(\nu_m+\halfDnu)t}
\]%\end{align*}
\end{define}
The \WV\ transform of the signal $x_1(t) + x_2(t)$ is given by
\begin{align}%at}{3}
\label{eq:WignerBeats}
  \W_{x_1+x_2}(t,\nu)
    &= \W_{x_1}(t,\nu)+\W_{x_2}(t,\nu) + I_{x_1x_2}(t,\nu)\\
    &=\delta(\nu-(\nu_m-\halfDnu))+\delta(\nu-(\nu_m+\halfDnu))
      + \delta(\nu-\nu_m)\,2\cos(2\pi \Delta\nu t) \nonumber
\end{align}%at}
Now let us relate this expression to the physical phenomenon of
beats, which are perceived most easily when the distance between
signal components %, $\Delta\nu$, 
is small.  To do so, we write the signal as follows:
\begin{align}
x_1(t)+x_2(t) 
    &= e^{i2\pi(\nu_m-\halfDnu)t} 
       + e^{i2\pi(\nu_m+\halfDnu)t}\nonumber \\
    &= (e^{-i2\pi\halfDnu t} + e^{i2\pi\halfDnu t})\,
       e^{i2\pi\nu_m t}\nonumber \\ 
    &= 2\,\cos(2\pi\halfDnu t)\,e^{i2\pi\nu_m t}\label{eq:sigbeats}
\end{align}
%This expresses the signal as a tone with frequency $\nu_m$ and
%modulated amplitude $2\cos(2\pi\halfDnu t)$.  
%This is a sinusoidal of frequency $\nu_m$ with 
When the components $x_1(t)$ and $x_2(t)$ are close together in
frequency -- that is, when $\Delta\nu$ is small -- the cosine term is
slowly varying as compared to the exponential term, and the resulting
signal can be viewed as a simple tone of frequency $\nu_m$ with
%slowly varying 
a modulated amplitude envelope, with modulation frequency $\Delta\nu$.
The term ``beats,'' or ``beating,'' refers to such
amplitude modulations. % due to interaction of signal components.

Comparing~(\ref{eq:sigbeats}) with~(\ref{eq:WignerBeats}), it is
clearly the interference term of the \WV\ transform which specifies
the existence and nature of beats in the composite signal.
% of interacting signal components.
%existence of the beat frequency component 
%and stands for its realization in the time-frequency plane.   
To make for easier comparison, we can also write this signal as
\[
  x_1(t)+x_2(t) 
    = \half \,e^{i2\pi(\nu_m-\halfDnu)t} 
      + \half \,e^{i2\pi(\nu_m+\halfDnu)t}
      + \cos(2\pi\halfDnu t)\,e^{i2\pi\nu_m t}
\]
\begin{define}{\bf Interferences in the Ambiguity Plane. }
The ambiguity function is also a bilinear form, has its own
interference structure, and represents both signal and interference
components in the frequency-time, or ``Doppler-delay,'' plane.
A superposition of two signal components, 
$x_1(t) + x_2(t)$, 
yields the following frequency-time representation:
\[
  \A_{x_1+x_2}(\xi,\tau) 
    = \A_{x_1}(\xi,\tau)+\A_{x_2}(\xi,\tau)
      + \A_{x_1x_2}(\xi,\tau) + \A_{x_2x_1}(\xi,\tau) 
\]
\end{define}
Let the interference term be denoted
\[
 J_{x_1x_2}(\xi,\tau) = \A_{x_1x_2}(\xi,\tau) + \A_{x_2x_1}(\xi,\tau) 
\]
The Fourier relation of equation~(\ref{eq:ambig-ft}), between the \WV\
and the ambiguity function, proves that 
\[
 I_{x_1x_2} = \iint\limits_{\R^2} J_{x_1x_2}(\xi,\tau) 
              e^{-i2\pi(\xi t + \nu\tau)}\,d\xi\,d\tau
\]

For the previous example, in which 
\[%\begin{align*}
x_1(t)%\xfpull(t)% = x(t)e^{-i2\pi(\halfDnu)t}\\
      =e^{i2\pi(\nu_m-\halfDnu)t},\quad \text{ and } \quad
x_2(t)%\xfpush(t)% = x(t)e^{i2\pi(\halfDnu)t}\\
      = e^{i2\pi(\nu_m+\halfDnu)t}
\]%\end{align*}
direct calculation yield
\[%\begin{alignat*}{2}
\A_{x_1}(\xi,\tau)=\delta(\xi)\,e^{i2\pi(\nu_m-\halfDnu)\tau},\quad
\A_{x_1x_2}(\xi,\tau)=\delta(\xi+\Delta\nu)\,e^{i2\pi\nu_m\tau}\]%\\
\[
\A_{x_2}(\xi,\tau)=\delta(\xi)\,e^{i2\pi(\nu_m+\halfDnu)\tau},\quad
\A_{x_2x_1}(\xi,\tau)= \delta(\xi-\Delta\nu)\,e^{i2\pi\nu_m\tau}
\]%\end{alignat*}
Summing these and simplifying yields
\begin{align*}
%\label{eq:ambig-beats}
\A_{x_1+x_2}(\xi,\tau) 
%  &= \delta(\xi)(e^{i2\pi(\nu_m-\halfDnu)\tau} 
%                +e^{i2\pi(\nu_m+\halfDnu)\tau})
%   + (\delta(\xi+\Delta\nu)+\delta(\xi-\Delta\nu))e^{i2\pi\nu_m\tau}\\
  &= [2\cos(2\pi\halfDnu\tau)\,\delta(\xi)
   +  \delta(\xi+\Delta\nu)+\delta(\xi-\Delta\nu)]\,e^{i2\pi\nu_m\tau}
\end{align*}
%\begin{define}{\bf Physical Interpretations in the Ambiguity Plane. }
The interference term for this example is
\[
J_{x_1x_2}(\xi,\tau) =    
  [\delta(\xi+\Delta\nu)+\delta(\xi-\Delta\nu)]\,e^{i2\pi\nu_m\tau}
\]
which demonstrates that interferences appear in the ambiguity plane
where the Doppler variable $\xi$ is equal to $\pm\Delta\nu$.  This
quantity represents the frequency difference between the two signal
components.   

In the context of a dissonance analysis, we could interpret the
interferences that appear near the origin as are responsible for beating or
``roughness'' as it is here that signal components are %found to be
closest in frequency.  However, such an analysis relies on our ability
to separate signal and interference components when computing the
time-frequency representation.  This is crucial, particularly for the
ambiguity function in which the true signal components appear
near the origin.  In that case it is impossible to discern
interference terms resulting from interaction among components with
small frequency differences, unless we study the interferences terms
by themselves.  We consider one method of separating signal and
interference energy in the following section.

%The instantaneous power of the signal is 
%\[\left|x_1(t)+x_2(t)\right|^2 = 2 [1+\cos(2\pi\xi t)]\]
%Hence, it is governed by fluctuations that have a longer period if the
%two frequencies are closer to each other.  Owing to the correct
%marginal energy densities of the \WV\ transform, this value must
%coincide with the sum of all amplitudes of the signal terms
%\emph{and} the cross terms at this instant.  

\subsection{Energy Separation}% via Atomic Decomposition}
\label{sec:energy-separation}
Recall the matching pursuit decomposition of section~\ref{sec:MP},
\[%begin{equation}\label{eq:MP2}
x(t) = \sum_{n=0}^\infty C(R^nx,g_{\gamma_n})g_{\gamma_n}(t) 
\]%end{equation}
Referring to equation~(\ref{eq:WignerSum}), we see that the 
corresponding \WV\ representation is
\begin{equation}\label{eq:WVMP}
W_x(t,\nu) = \sum_{n=0}^\infty 
    \left|C(R^nx,g_{\gamma_n})\right|^2 \W_{g_{\gamma_n}}(t,\nu) 
    + 2\sum_{n=0}^{\infty}\sum_{k=n+1}^\infty\,
     \Real\left[C(R^nx,g_{\gamma_n}) C^*(R^nx,g_{\gamma_k})
                \W_{g_{\gamma_n}g_{\gamma_k}}(t,\nu)\right] 
\end{equation}
However, in the literature~\cite{Gribonval:1996}, we find a
simpler representation of signal energy, defined by
\begin{equation}\label{eq:MP2}
E_x(t,\nu) = \sum_{n=0}^\infty 
    \left|C(R^nx,g_{\gamma_n})\right|^2 \W_{g_{\gamma_n}}(t,\nu) 
\end{equation}
Thus, only the first term of%the \WV\ representation in
~(\ref{eq:WVMP}) appears in the definition of $E_x$,
% representation of~(\ref{eq:MP2}). 
the idea being that this term accounts for the energy of the
``true'' signal components.  Since this is the primary concern in
the matching pursuit literature, the definition of signal energy 
found in such literature does not include interference terms.

Denoting the cross terms of~(\ref{eq:WVMP}) by
\[
I_x(t,\nu) = 2\sum_{n=0}^{\infty}\sum_{k=n+1}^\infty\,
     \Real\left[C(R^nx,g_{\gamma_n}) C^*(R^nx,g_{\gamma_k})
                \W_{g_{\gamma_n}g_{\gamma_k}}(t,\nu)\right] 
\]
we can write the \WV\ as
\[
W_x(t,\nu)=E_x(t,\nu) +I_x(t,\nu) 
\]
and we call $E_x(t,\nu)$ the \emph{signal energy} and $I_x(t,\nu)$ the
\emph{interference energy.}
% (end: inserted from file energydensity.tex)

% (end: insert file interference.tex)






