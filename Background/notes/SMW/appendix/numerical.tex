\section{A Note on Numerical Precision}
\subsection{Numerical Support of Gabor Atoms}
Here we remark on some practical considerations which we may want to
account for in future (more numerically precise) constructions of the
Gabor dictionary. 

Suppose we begin with a general gaussian function
\[
g(t) = e^{-\alpha t^2}, \qquad t \in \R,\; \alpha > 0
\]
Then we translate $g$ in time by $p$, scale it by $s$, and
modulate it by $f$ Hz, yielding 
\[
g_\gamma(t) = Kg\left(\frac{t-p}{s}\right)e^{i2\pi ft} 
= Ke^{-\alpha \left(\frac{t-p}{s}\right)^2}e^{i2\pi ft} 
\]
where $\gamma = (p,s,f)$ denotes the paramter vector and $K$ is chosen
so that $\|g_\gamma\|=1$.

Clearly $g > 0$ for all $t$, and $g(t)\rightarrow 0$ as $t\rightarrow
\pm \infty$.  Thus, the \emph{theoretical support} of %both $g$ and
$g_\gamma$ is the entire real line (minus the zeros resulting from
the frequency modulation). However, for practical applications in
which we are restricted to finite precision arithmetic, it is
important to consider the \emph{numerical support} of $g_\gamma$,
which is necessarily a finite subset of $\R$.  In other words, we must
consider for what values of $t$ is it the case that $g_\gamma(t) \geq
\epsilon_m$. Here $\epsilon_m$ is the \emph{machine epsilon}, defined 
as the smallest positive number attainable by the computer.  For
example, on our Pentium II hardware, the Matlab function {\tt eps}
returns
\[
\epsilon_m \approx  2.22 \times 10^{-16}
\]
Clearly, when $t$ is such that $|g_\gamma(t)| < \epsilon_m$, the
computer perceives $g_\gamma(t)$ as no different from 0, and we must
assume that the function is not numerically supported on these values
of $t$.

Let us now find the numerical support of $g_\gamma$, for $p = 0$; that
is, for $\gamma = (0,s,f)$.
% (This can be generalized for $p>0$ with a
%simple time translation.)
\begin{eqnarray}
|g_\gamma(t)| \geq \epsilon_m 
&\Leftrightarrow& 
|K e^{-\alpha \left(\frac{t}{s}\right)^2}e^{i2\pi ft}| \geq
\epsilon_m \\ 
&\Leftrightarrow& 
K e^{-\alpha \left(\frac{t}{s}\right)^2} \geq \epsilon_m \\ 
&\Leftrightarrow& 
-\alpha \left(\frac{t}{s}\right)^2 
\geq \log\epsilon_m - \log K \nonumber\\ 
& \Leftrightarrow & \left(\frac{t}{s}\right)^2 \leq
\frac{\log K -\log\epsilon_m}{\alpha}\nonumber\\ 
\label{eqn:support}
& \Leftrightarrow & 
|t| \leq s\left(\frac{\log K -\log\epsilon_m}{\alpha}\right)^{1/2}
\end{eqnarray}
Since $\epsilon_m$ is a small positive number,
$\log K-\log\epsilon_m$ is positive.  Thus, the square root in
the last inequality of (\ref{eqn:support}) results in a positive real
number, and defines the numerical support of $g_\gamma$. 

It is convenient to choose $\alpha$ such that the numerical support of
$g_\gamma$ is $[-s/2,s/2]$.  
%$g$ is $[-1/2,1/2]$.  
%Then scaling by $s$ results in a scaled gaussian,
%$g_s(t) = g(t/s)$, with numerical support $[-s/2,s/2]$.  
However, such an $\alpha$ is difficult to acheive analytically because
the normalization constant $K$ depends on $\alpha$ and $s$.
So, we proceed by deriving $K$ and $\alpha$ analytically as far as
possible, and then we consider a numerical optimization 
that will yield Gabor atoms having the desired numerical support.

We seek $K$ such that $\|g_\gamma\| = 1$, where
\[
g_\gamma(t) = Ke^{-\alpha \left(\frac{t}{s}\right)^2}e^{i2\pi ft},
\qquad t \in \R,\; \alpha > 0
\]
and
%\begin{eqnarray*}\begin{align*}
\[
\|g_\gamma\|^2 = \integral g_\gamma(t)g_\gamma^*(t)\,dt
               = K^2\integral
        e^{-2\alpha\left(\frac{t}{s}\right)^2}\,dt\\ 
\]
%\end{eqnarray*}\end{align*}
%Recall that a gaussian density integrates to 1. For a density with
%variance $\sigma^2$, this means 
%\begin{equation}\label{eq:gaussint}
%\frac{1}{\sigma \sqrt{2\pi}} \integral e^{-t^2/2\sigma^2}\,dt = 1
%\end{equation}
%Using this fact, we derive the following
Such a $K$ is derived as follows:
\begin{align*}
1 = \|g_\gamma\|^2 &= K^2\integral
    e^{-2\alpha\left(\frac{t}{s}\right)^2}\,dt\\ 
\Leftrightarrow \quad K^{-2} &=
\integral e^{-2\alpha\left(\frac{t}{s}\right)^2}\,dt
%\end{align*}
%This holds if, and only if,
%\begin{eqnarray*}
\\
\Leftrightarrow \quad K^{-2}
&=\integral\exp\left\{-\frac{t^2}{2(s/2\sqrt{\alpha})^2}\right\}\,dt
\end{align*}
Letting $\sigma= \frac{s}{2\sqrt{\alpha}}$, the foregoing is
equivalent to 
\[%\begin{equation}\label{eq:K}
K^{-2} = %\sigma \sqrt{2\pi} \frac{1}{\sigma \sqrt{2\pi}}
\integral e^{\frac{-t^2}{2 \sigma^2}}\,dt\\
       = \sigma \sqrt{2\pi}
\]%\end{equation}
The final equality obtains from the fact that a gaussian density integrates to 1.
%Fact~(\ref{eq:gaussint}) 
%Finally, equation~(\ref{eq:K}) 
This shows that the value of $K$ resulting in atoms of unit
norm is  
\[
K=(\sigma \sqrt{2\pi})^{-1/2}=\left(\frac{2\alpha}{\pi s^2}\right)^{1/4}
\]
Inserting this expression into equation (\ref{eqn:support}), which
defines the numerical support, we have
\begin{equation}
\label{eq:alpha}
\frac{\log K -\log\epsilon_m}{\alpha} =\frac{1}{4}
\left(\frac{\log\alpha-2\log s - 4\log\epsilon_m+\log(2/\pi)}
           {\alpha}\right) 
\end{equation}
Recall from (\ref{eqn:support}) that we seek a value of $\alpha$ such
that the right hand side of equation~(\ref{eq:alpha}) is
equal to 1/4.  This occurs when 
\[
\alpha = \log \alpha -2\log s - 4\log \epsilon_m+\log(2/\pi)
\]
which is nonlinear in $\alpha$.  Therefore, for given values of $s$,
we select numerically optimal values of $\alpha$ by minimizing the
function 
\[
L_s(\alpha)=
|\alpha - \log\alpha + 2\log s + 4\log\epsilon_m-\log(2/\pi)|
\]
The Matlab routine {\tt fminbnd.m} accomplishes this effectively.  We
use it as follows:
\begin{verbatim}
>> fn = inline('abs(x-log(x)+4*log(eps)+2*log(P1)-log(2/pi))',1);
>> alpha = fminbnd(fn,1,200,optimset('Display','off'),S);
\end{verbatim}

%However, numerical
%experimetation shows that the $\log \alpha$ term is well approximated
%by a linear function of $r = \log_2 s$.  After some simulation,
%we find that the following definition of $\alpha$ yields suitable
%numerical supports for the scaled gaussians $g_\gamma(t)$,
%\[
%\alpha =  5 - 0.01 r - 2\log s - 4\log \epsilon_m + \log(2/\pi),
%\quad \text{ where } s = 2^r.
%\]





%%% (erroneous) characteristic function derivation %%%
%
%&=& K^2 \integral e^{i4\pi ft} e^{-2 \alpha \left(\frac{t}{s}\right)^2}\,dt\\
%&=& K^2 \integral 
%    e^{i4\pi ft} \exp\left\{\frac{-t}{2 (s/2\sqrt{\alpha})^2}\right\}\,dt
%\end{eqnarray*}
%Now let $\sigma= \frac{s}{\sqrt{\alpha}}$, yielding
%\begin{eqnarray*}
%\|g_\gamma\|^2 
%&=& K^2 \sigma \sqrt{2\pi} \frac{1}{\sigma \sqrt{2\pi}}
%\integral e^{i4\pi ft} \exp\left\{\frac{-t^2}{2 \sigma^2}\right\}\,dt\\
%&=& K^2 \sigma \sqrt{2\pi} \phi_X(f)\\
%&=& K^2 \sigma \sqrt{2\pi} 
%    \exp\left\{\frac{-\sigma^2(4\pi f)^2}{2}\right\}
%\end{eqnarray*}
%The last equality holds because the integral that we have labelled
%$\phi_X(f)$ is the characteristic function of $X$, a normal random
%variable with mean 0 and variance $\sigma^2$.
%
