% -*- mode: LaTeX; tex-main-file: "../notes.tex"; -*-
\section{Perception of Complex Tones}
This section contains notes on the perception of complex tones.  In
particular, we focus on two concepts that are related to 
%have connections to 
the ideas in the paper, but which are not described therein, as they
are not immediately relevant.  This material is included here because
it might be useful in future work. 
%The last part of this section presents some notes on the music we use
%in the experiments of section~\ref{sec:practical}. 
% (begin: insert file fusion.tex)
\subsection{Tonal Fission and Fusion} 
In this section are some notes on tonal fusion.  First are excerpts
from Sethares~\cite{Sethares:1997} (p. 25, 26), and then some brief notes
from Hartmann~\cite{Hartmann:1998}. 

Almost all musical sounds consist of a great many partials, whether they are
harmonically related or not.  Using techniques such as selective damping and
the selective excitation models, it is possible to learn to ``hear out'' these
partials, to directly perceive the spectrum of the sound.  This kind of
listening is called \emph{analytic} listening, as compared to \emph{holistic}
listening in which the partials fuse together into one perceptual entity.

Consider the closing chord of a string quartet.  At one extreme, is the fully
analytic listener that ``hears out'' a large number of individual partials.
At the other extreme is the fully holistic listener who hears the chord as one
grand ``tone,'' all four instruments fusing into a single rich and complex
sonic texture.  Typical listening lies somewhere in between: the partials of
each instrument fuse, but the instruments remain individually perceptible,
each with its own pitch, loudness, vibrato, etc.  What physical cluesmake this
remarkable feat of perception possible?

When listeners are presented with clusters of partials and asked how many
distinct voices, notes, etc. they hear, various features of the presentation
reliably encourage tonal fusion.  Some of these features are whether the
partials:
\begin{enumerate}
\item begin at the same time (attack synchrony),
\item have similar envelopes (amplitudes change similarly over time),
\item are harmonically related, or
\item have the same vibrato rate.
\end{enumerate}
Almost any common feature of a subgroup of partials helps them to be perceived
together. 

Hartmann~\cite{Hartmann:1998} (p. 134) states that the most important mediator of
perceptual segregation and integration is the onset time of the tones, which
he calls \emph{onset synchronicity} (cf. attack syncrony of Sethares).  If two
sets of partials have different onset times, then the auditory system
segregates them as different entities.  

A study by Rasch (1978) indicates that 30ms advanced onset leads to 40dB
increase in level perception and allows the listener to segregate while
remaining unaware of the 30ms asynchrony.  However, research by Risset,
Matthews (1969) and Beauchamp (1975) shows that the harmonics of high
frequencies lag those of low frequencies by greater than 30ms\footnote{A
  diagram of this phenomena is in the comp book at (00.07.10).}
% (end: insert file fusion.tex)

% (begin: insert file dominance.tex)
\ifthenelse{\boolean{nofootnotes}}{\subsection{Spectral Dominance}}
{\subsection{Spectral Dominance\protect\footnotemark }
\footnotetext{This topic is also discussed in Hartmann
\cite{Hartmann:1998}.}}
A periodic complex tone has a pitch that is equal to, or slightly less
than, the frequency of the fundamental. 
Assuming the various overtones of a complex tone fuse together to
create an overall pitch, just how these overtones are combined is a
question of great interest.  In particular, if a complex tone is
composed of pure sinusoidal partials, each having a given frequency
and amplitude, which of these partials will be more pronounced and make a
greater contribution to the determination of the fundamental (low) pitch of
the complex tone?  A promising experimental approach is to mistune one
or more partials and study the effect on the synthesized low pitch.
Experiments by Ritsma (1967) and Moore, Glasberg, and Peters (1985)
led to the conclusion that the partials that are most important in
determining the low pitch are those that are resolved by the auditory
periphery.  Retsma concluded that for fundamental frequencies between
100 and 400 Hz, partials three, four, and five are dominant.  Terhardt
(1974) proposed a dominant frequency range centered on 700 Hz, instead
of particular harmonic numbers.  Moore {\it et al}.~concluded that the
dominant partials were among the first six, but individual differences
prevented them from being more specific.

The observed importance of the resolved harmonics supports the idea
that pitch perception takes place at high levels where excitations
from different peripheral channels are recombined.  Pattern matching
or template fitting models focus on this idea.  The model studied by
Goldstein (1973) derives the low pitch from a pattern match to the
frequencies of the higher partials.
% (end: insert file dominance.tex)
