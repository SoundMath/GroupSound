% -*- mode: LaTeX; tex-main-file: "../notes.tex"; -*-
\section{Matlab Code}
\subsection{Synthesis}
The following Matlab code generates Figure~\ref{fig:synth1}
and~\ref{fig:synth2}:
\begin{verbatim}
% synthAnal.m
% Matlab code for synthesis of a complex tone
% based on Sethares, "Tuning, Timbre, Spectrum, Scale"
%
% sr = sampling rate
% time = duration of wave to generate (in seconds)
% freq = frequencies of partials
% amp = amplitudes of partials

sr = 44100; time = 1; freq = [220 440 440]; 
amp =  [1.2 1.0 1.0]; pi2 = 2*pi; t = 0:pi2/sr:pi2*time; 
wave=0*ones(length(freq)+2,length(t)); env = ones(size(t));

% Compute first 3 pure sine waves at 220, 440, 440 Hz
for i = 1:length(freq)
  wave(i,:)=wave(i,:)+amp(i)*env.*cos(freq(i)*t+pi2*rand);
end
% Compute sum of sine waves 1 and 2
wave(length(freq)+1,:)= wave(length(freq)+1,:)+wave(1,:)+wave(2,:);
% Compute sum of sine waves 1 and 3
wave(length(freq)+2,:)= wave(length(freq)+2,:)+wave(1,:)+wave(3,:);

% Plot pure sine waves 1 and 2
subplot(3,1,1); plot(t(1:2000)/pi2,wave(1,1:2000)); 
grid; axis([0,0.04,-1.4,1.4])
set(gca,'YTick',-1:1);set(gca,'XTickLabel',[]);
subplot(3,1,2); plot(t(1:2000)/pi2,wave(2,1:2000)); 
grid; axis([0,0.04,-1.4,1.4])
set(gca,'YTick',-1:1);set(gca,'XTickLabel',[]);
% Plot sum of waves 1 and 2
subplot(3,1,3); plot(t(1:2000)/pi2,wave(4,1:2000)); 
grid; axis([0,0.04,-2.4,2.4])
xlabel('time (in seconds)'); 
figure
% Plot pure sine wave 1 and 3
subplot(3,1,1); plot(t(1:2000)/pi2,wave(1,1:2000)); 
grid; axis([0,0.04,-1.4,1.4])
set(gca,'YTick',-1:1);set(gca,'XTickLabel',[]);
subplot(3,1,2); plot(t(1:2000)/pi2,wave(3,1:2000)); 
grid; axis([0,0.04,-1.4,1.4])
set(gca,'YTick',-1:1);set(gca,'XTickLabel',[]);
% Plot sum of waves 1 and 3
subplot(3,1,3); plot(t(1:2000)/pi2,wave(5,1:2000)); 
grid; axis([0,0.04,-2.4,2.4])
xlabel('time (in seconds)'); 

\end{verbatim}

\subsection{Spectrum of a Sine Wave}\label{sec:Sinewave}
The following Matlab code generates Figures~\ref{fig:FTaperiodic},
\ref{fig:FTperiodic},~\ref{fig:hamming}, and~\ref{fig:FThamming}.
\begin{verbatim}
function [y,t]=Sinewave(CPS, SR, N)
% Sinewave.m
% Matlab code for computing a sine wave and ploting it (if PLOT=1).
%
% inputs: CPS  is the Cycles Per Second (frequency) of the sine wave in Hz.
%              This is equivalent to (1/period).
%         SR   is the Sample Rate or number of sample points per second.
%         N    is the total number of samples to generate

PLOT=0;
t=(0:(N-1))/SR;
y=sin(2*pi*CPS*t);
if PLOT==1
  figure
  disp(sprintf('Genrating plot of first 4 periods...'));
  indx=find(t < 4/CPS + 1/SR & t > 4/CPS);
  plot(t(1:indx),y(1:indx)); axis([0,t(indx),-1,1]);
end

%% end of file: Sinewave.m

% sineFFT.m
%
% Matlab code to:
%  1. generate sine wave by calling function Sinewave() 
%  2. take its FFT 
%  3. plot the wave and FFT. (if PLOT=1)

PLOT=1;   % 1 => draw plots;
INTEGRALPERIODS=0;  % 1 => adjust frequ to fit sample size;
CPS=220;  % frequency (Hz)
N=2^15;   % sample points
SR=44100; % sample rate
LL=-1;    % (for lower limit of FFT plot)

% get a close approximation to CPS for which the sine wave 
% will have a period that is an integer multiple of SR/N.
cycs=floor(N/SR *CPS);  
if INTEGRALPERIODS==1
  CPS = SR/N * cycs; % new CPS
  LL=-15;
end

[y,t]=Sinewave(CPS,SR,N);
Y=fft(y);
magspec=abs(Y);

if PLOT==1
  ppc=floor(SR/CPS);
  tper=(1:ppc)/SR; lenp=length(tper);
  tper=tper+ones(size(tper))*t(N);
  lT=[t tper];
  figure;  subplot(2,1,1);
  st=(cycs-4)/CPS; % start time of 2nd to last cycle
  indx=find(t > (st-1/SR) & t <= st);
  plot(lT(indx:N),y(indx:N)); grid;
  title('Last four periods of exactly 220 Hz sine wave');
  axis([lT(indx),lT(length(lT)),-1,1]);
  hold on
  plot(lT(N+1:length(lT)),y(1:lenp),'--');grid;
  xline=[t(N) t(N)]; yline=[-1 1];
  line(xline,yline,'LineStyle','--');
  xlabel('time (seconds)');
  subplot(2,1,2);
  f=(0:(N-1)/2)*(SR/N);
  semilogy(f(1:N/4),magspec(1:N/4));grid;
  axis([0,1000,10^(LL),10^5]);
  xlabel('frequency (Hz)');
end

%% end of file: sineFFT.m

% hammingFFT.m
%
% Matlab code to:
%  1. generate sine wave by calling function Sinewave() 
%  2. multiply it by a Hamming window
%  3. take its FFT 
%  4. plot the wave and FFT. (if PLOT=1)

PLOT=1;    % 1 => draw frequency plots; 2 => draw hamming window
CPS=20;   % frequency (Hz)
N=2^15;    % sample points
SR=44100;  % sample rate
LL=-1;     % (for lower limit of FFT plot)
cycs=floor(N/SR *CPS); 

[y,t]=Sinewave(CPS,SR,N);
ht=(0:(N-1))/N;
ham=0.54*ones(1,length(ht)) - 0.46*cos(2*pi*ht);
yh = y.*ham;

Y=fft(yh);
magspec=abs(Y);

if PLOT==2
  figure  
  subplot(3,1,1); plot(t,y);grid;
  axis([t(1),t(N),-1,1]);
  subplot(3,1,2); plot(t,ham);grid;  
  axis([t(1),t(N),-1,1]);
  subplot(3,1,3); plot(t,yh);grid;
  axis([t(1),t(N),-1,1]);
  xlabel('time (seconds)');
end

if PLOT==1
  ppc=floor(SR/CPS);
  tper=(1:ppc)/SR; lenp=length(tper);
  tper=tper+ones(size(tper))*t(N);
  lT=[t tper];
  figure;  subplot(3,1,1);
  st=(cycs-20)/CPS; % start time of 10th to last cycle
  indx=find(t > (st-1/SR) & t <= st);
  subplot(2,1,1);
  plot(lT(indx:N),yh(indx:N));grid;
  title('Last ten periods of exactly 220 Hz sine wave');
  axis([lT(indx),lT(length(lT)),-1,1]);
  hold on
  plot(lT(N+1:length(lT)),yh(1:lenp),'--');
  xline=[t(N) t(N)]; yline=[-1 1];
  line(xline,yline,'LineStyle','--');
  xlabel('time (seconds)');
  f=(0:(N-1)/2)*(SR/N);
  subplot(2,1,2);
  semilogy(f(1:N/4),magspec(1:N/4));grid;
  axis([0,1000,10^(LL),10^5]);
  xlabel('frequency (Hz)');
end

%% end of file: hammingFFT.m
\end{verbatim}

\subsection{Consonance}\label{sec:3DissCurves}
The following Matlab code generates Figure~\ref{fig:3DissCurves}:
\begin{verbatim}
% 3DissCurves.m
%
% Matlab code for generating dissonance curves using the function
% dissmeasure.m.  
%
% Based on Sethares, "Tuning, Timbre, Spectrum, Scale" p.45, 301.
%
freq=[220 440 880]'; amp=1;
range=2.3; inc=0.01;
figure; hold on;
% Call function dissmeasure for each interval
for i=1:length(freq)
  diss=[0];
  for alpha = 1+inc:inc:range
    f=[freq(i,1) alpha*freq(i,1)];
    a=[amp, amp];
    d=dissmeasure(f,a);
    diss=[diss d];
  end
  plot(1:inc:range,diss)
end
grid; set(gca,'XTick',1:0.0833:2);
set(gca,'XTickLabel',{'';'';'';'';'';'fourth';'';'fifth';'';'';'';'';'octave'});
axis([1,2,0,0.9]);

%% end of file: 3DissCurves.m

function d=dissmeasure(f,amp)
%
% Matlab code for calculating instrinsic dissonance of a complex tone.
% Based on Sethares, "Tuning, Timbre, Spectrum, Scale" p.301.
% Given a set of partials in f, with amplitudes in amp, this routine
% calculates the `sensory' dissonance among them.
%
% constants for calculation (initialization only)
if ~exist('firstpass')
  Dstar = 0.24; S1=0.0207; S2=18.96; C1=5; C2=-5; 
  A1=-3.51; A2=-5.75; firstpass=1; 
end
N=length(f); D=0;
[f,ind] = sort(f);
ams=amp(ind);
for i=2:N
  Fmin = f(1:N-i+1);
  S = Dstar./(S1*Fmin+S2); % equation (E.3)
  Fdif = f(i:N)-f(1:N-i+1);
  a = ams(i:N).*ams(1:N-i+1);
  Dnew = a.*(C1*exp(A1*S.*Fdif)+C2*exp(A2*S.*Fdif));
  D = D+Dnew*ones(size(Dnew))';
end
d=D;

%% end of file: dissmeasure.m
\end{verbatim}
%%%% New Matlab Code begins here:
\subsection{Energy Separation}\label{sec:DEnergy}
\begin{verbatim}
function [WVE,WVI,P_DEnergy,C_DEnergy] = DEnergy(sig,DEBUG,DISP,P,C);
% DEnergy.m
% Matlab code for computing energy and interference using matching
% pursuit algorithm, MZMP, Wigner transform, WVTrans_AF, and
% cross-Wigner transform, WVTrans_AFC.
%
% Inputs
%   sig       1-dimensional signal of interest
%   DEBUG     (default 1) DEBUG=0 => do not print extra debugging info
%   DISP      (default 1) DISP=0 => do not display plot of results
%   P         optional input array of atom parameters from a prior run
%   C         optional input vector of atom correlations from a prior run 
%
% Outputs
%   WVE       complex-valued matrix representing the sum of the
%             Wigner transforms of each atom in the collection of atoms
%             used to represent the signal. Rows correspond to
%             frequencies and columns corresponding to times.
%   WVI       Same as WVE except that *cross* Wigner transforms are
%             summed.
%   P_DEnergy optionally stored array of atom parameters
%   C_DEnergy optionally stored vector of atom correlations
%
% Usage
%  >> pause off; [WVE,WVI,P_DEnergy,C_DEnergy]=DEnergy;
%  >> [WVE,WVI] = DEnergy(sig,1,1,P_DEnergy,C_DEnergy);
%
% See also:  MZMP, gaborMZ, WVTrans_AF, WVTrans_AFC
%
% Author: William DeMeo <william.demeo@verizon.net>
% Date: 2001 July 20
% Copyright (c) 2001, William DeMeo

if nargin < 1, 
  disp('DEnergy: no signal input... proceding with synthetic signal demo.');
  K=8; N=pow2(K+1);       % signal length
  %%% synthetic signal params (signal should really be an input) %%%%
  f1 = .1; f2 = 2*f1; f3 = 3*f1; %f1 = 440/N; f2 = 2*f1; f3 = 3*f1;
  sig=atoms(N,[1,f1,pow2(K-2),.75; 1,f2,pow2(K-2),.25; ...
    pow2(K),f2,pow2(K-2),1; 3*pow2(K-1),f3,pow2(K-2),.5]);
else,  
  N=length(sig); K=nextpow2(N)-1; 
end;
if nargin < 2, DEBUG=1; end;
if nargin < 3, DISP=1; end;

% Matching Pursuit parameters
MAXITERS=20; 
TOLERANCE=0.05;

if nargin < 5,
  if(DEBUG),disp('DEnergy: computing new atoms with MZMP...');end;
  %%% CALL MATCHING PURSUIT FUNCTION %%%
  [P,C] = MZMP(sig,MAXITERS,TOLERANCE,DEBUG);
end;
% store a local copy of parameter/correlation book of atoms 
P_DEnergy = P; C_DEnergy = C;  % (and return it if requested)

% P is at most a size MAXITERSx4 array; C is at most 1xMAXITERS
M = size(P,1);     % M = number of atoms used in signal representation
WVE = zeros(N,N);
WVI = zeros(N,N);

if(DEBUG),disp('DEnergy: computing energy & interference...');end;
for n=1:M-1,
  [g1,Ks] = gaborMZ(K,P(n,1),P(n,2),P(n,3));
  WV = WVTrans_AF(g1,0);
  WVE = WVE + (abs(C(n))^2)*WV;
  for m=n+1:M,
    [g2,Ks] = gaborMZ(K,P(m,1),P(m,2),P(m,3));
    WV = WVTrans_AFC(g1,g2,0);
    WVI = WVI + 2*real(C(n)*conj(C(m))*WV);
  end;
  if(DEBUG),disp(sprintf('DEnergy: finished atom %d of %d',n,M));end;
end;

[g1,Ks] = gaborMZ(K,P(M,1),P(M,2),P(M,3));
WV = WVTrans_AF(g1,0);
WVE = WVE + (abs(C(M))^2)*WV;

if(DISP),
  figure(1);clf;
  absE = real(WVE);
  tfmax = max(max(absE));
  tfmin = min(min(absE));
  colormap(1-gray(256));
  image(linspace(0,1,N),linspace(0,1,N),256*(absE-tfmin)/(tfmax-tfmin));
  axis('xy'); title('Wigner Energy'); xlabel('Time'); ylabel('Frequency')

  figure(2);clf;
  absI = real(WVI);
  tfmax = max(max(absI)); tfmin = min(min(absI));
  colormap(1-gray(256));
  image(linspace(0,1,N),linspace(0,1,N),256*(absI-tfmin)/(tfmax-tfmin));
  axis('xy'); title('Wigner Interference'); xlabel('Time'); ylabel('Frequency');

  figure(3);clf;  
  abstf = real(WVE+WVI);
  tfmax = max(max(abstf)); tfmin = min(min(abstf));
  colormap(1-gray(256));
  image(linspace(0,1,N),linspace(0,1,N),256*(abstf-tfmin)/(tfmax-tfmin));
  axis('xy'); title('Wigner Transform'); xlabel('Time'); ylabel('Frequency')
end;

%% end of file: DEnergy.m
\end{verbatim}

\subsection{Matching Pursuit}\label{sec:MZMP}
\begin{verbatim}
function [MaxP,MaxC,P_MZMP,C_MZMP]=MZMP(sig,MAXITERS,TOLERANCE,DEBUG,P,C)
% MZMP.m 
% Matlab function for Mallat/Zhang Matching Pursuit (MZMP) algorithm.
%
% Inputs
%    sig       signal (or part thereof) with which to correlate atoms;
%              signal length must be a power of 2
%    MAXITERS  (default 20) stop after MAXITERS iterations, even
%              if TOLERANCE is not reached.
%    TOLERANCE (default 0.05) stop after relative error is less 
%              than TOLERANCE (i.e. |Rx|/|x| < TOLERANCE)
%    DEBUG     (default 0) DEBUG=1 ==> print extra debugging information
%    P         optionally input NAtomsx4 array of all atom parameters
%    C         optionally input 1xNAtoms array of all inner products
%
% Outputs
%    MaxP   MAXITERSx4 array of atom parameters. MaxP(i,:) are the 4
%           parameters specifying the the atoms having correlation
%           MaxC(i) with the signal remainder at step i.
%    MaxC   1xMAXITERS vector where MaxC(i) is the correlation of the
%           signal remainder with max correl atom at iteration i.
%    P_MZMP optionally stored NAtomsx4 array of all atom parameters
%    C_MZMP optionally stored 1xNAtoms array of all inner products
%
% Notes
%    The total number of atoms in the dictionary is computed as follows:
%     (K+2)*4*N = (K+2)*2^(K+3) = 2^(K+3)    Diracs
%                               + 2^(K+3)*K  Gabor atoms
%                               + 2^(K+3)    Complex exponentials
%
%    This version does not consider whether the signal is real or
%    complex and, therefore, the resulting decomposition comprises
%    complex atoms. See MZRealMP for special handling of real signals.
%
% Other Notes
%    Now that this program has been tested and seems to work, all
%    major debugging conditionals have been removed in the interest of
%    efficiency.  See MZMP_DEBUG.m
%
% See also: gaborMZ.m, IP.m, DisplayParams.m
%
% Author: William DeMeo <william.demeo@verizon.net>
% Date: 2001 May 4
% Copyright (c) 2001, William DeMeo

if nargin < 1, 
  if(DEBUG),
    disp('MZMP: no signal input...');
    disp('MZMP: ...proceding with synthetic signal demo.');
  end;
  K=8; N=pow2(K+1);       % signal length
  %%% synthetic signal params (signal should really be an input) %%%%
  f1 = .1; f2 = 2*f1; f3 = 3*f1; %f1 = 440/N; f2 = 2*f1; f3 = 3*f1;
  sig=atoms(N,[1,f1,pow2(K-2),.75; 1,f2,pow2(K-2),.25; ...
    pow2(K),f2,pow2(K-2),1; 3*pow2(K-1),f3,pow2(K-2),.5]);
else,  
  N=length(sig); K=nextpow2(N)-1; 
end;

if nargin < 2, MAXITERS = 20; end;     % max number of iterations
if nargin < 3, TOLERANCE = 0.05; end;  % stop when |Rx|/|x| < TOLERANCE
if nargin < 4, DEBUG=0; end;

x=conj(sig(:)); x = x'; % signal is stored as ROW vector x
if(isreal(x)),
  error('use MZMPReal.m for real signals');
end;
x = hilbert(x);         % make signal analytic

pK3 = pow2(K+3);
OPT=3;                  % optimization coefficient
Nopts = pow2(OPT)-1;    % number of frequencies over which to optimize
                        % e.g. 3 => optimize over 7 frequencies
NAtoms = (K+2)*pK3;     % number of atoms in dictionary
% Ex. of (K,NAtoms) pairs: (9,45056) (10,98304) (11,212992) (12,458752)

MaxP=zeros(MAXITERS,4); % store all max parameters
MaxC=zeros(1,MAXITERS); % store all max correl values
RxoxE=zeros(1,MAXITERS);% est. norm of remainder at each iter.

DEBUG=1;  % DEBUG=1 ==> print basic tests and messages 
          % DEBUG=2 ==> extra msgs. (now appears only in MZMPoptDEBUG.m)
if(DEBUG),
  disp('pause is on, hit any key...'); pause;
  disp('============================');
  disp('MZMP: STEP I. Initialization');
  disp(sprintf('MZMP: %d atoms in dictionary...',NAtoms));
end;

if nargin < 6,
  P = zeros(NAtoms,4);    %%% Parameters (P), Correlations (C) for:
  C = zeros(1,NAtoms);    %   Diracs         ji = 0, 
                          %   Gabor atoms    0 < j < K+1,
                          %   Fourier basis  ji = K+1
  indx = 0;
  for j0=0:K+1, pj0 = pow2(j0);
    for k1=(1:2*pj0), fn = (k1-1)/(2*pj0);
      for u1 = (1:2*N/pj0), u=(u1-1)*pj0/2;
	indx = indx + 1; % == j0*pK3 + (k1-1)*pow2(K-j0+2) + u1;
	[g, Ks] = gaborMZ(K,j0,u,fn);
	C(indx) = x*g';  % same: C(i)=sum(x.*conj(g)); BECAUSE PRIMED!
	P(indx,:) = [j0 u fn Ks];
      end;
    end; 
    if(DEBUG),disp(sprintf('MZMP:       ...finished index %d',indx));end;
  end;
else,
  disp('MZMP: ...using previously computed atoms.');
end;
P_MZMP = P; C_MZMP = C;

[maxC,maxCi]=max(abs(C));  % get max abs corr. value and index
MaxC(1) = C(maxCi);        % not the same as maxC (which is an abs value)
MaxP(1,:) = P(maxCi,:);    %[j0 u fn Ks];

if(DEBUG),
  DisplayParams('MZMP: (before opt) MaxP(1,:) = ',MaxP(1,:),4);
  DisplayParams('MZMP:              MaxC(1) = ',MaxC(1),1);
end;

%%% I.b. OPTIMIZE %%% (only for non-Dirac, modulated atoms)
if(P(maxCi,1)>0 & P(maxCi,3)>0),          % non-Dirac atoms w/ pos. freq.
  j0 = P(maxCi,1); pj0 = pow2(j0); u0 = P(maxCi,2); f0 = P(maxCi,3);
  fopt = f0 + (-OPT:OPT)/(pow2(OPT)*pj0); % 2^OPT - 1 freqs around f0
  if(DEBUG & find(fopt > 1)),
    disp('WARNING: frequency values must be no greater than 1');
  end;
  [g, Ks] = gaborMZ(K,j0,u0,fopt);   % Nopts rows, one atom per row
  Copt = x*g';                       % Nopts x 1 col. of ips
  [maxCopt,maxCoptI]=max(abs(Copt)); % get opt max corr value and index
  if(maxCopt > maxC),                % use new opt freq 
    MaxC(1)=Copt(maxCoptI); MaxP(1,3)=fopt(maxCoptI); % MaxP(1,4)=Ks;
    if(DEBUG),
      disp(sprintf('MZMP: (compare) |MaxC| = %g, |MaxCopt| = %g',...
	  maxC,maxCopt));
      DisplayParams('MZMP: (after opt)  MaxP(1,:) = ',MaxP(1,:),4);
      DisplayParams('MZMP:              MaxC(1) = ',MaxC(1),1);
    end;
  end;
end;

%% Relative norm of estimated remainder %%
normx = norm(x);  % normx = sqrt(x*x');
normDiff = normx^2 - abs(MaxC(1))^2;
normRx = sqrt(abs(normDiff));  RxoxE(1) = normRx/normx;
if(DEBUG),
  if(normDiff < 0),  
    disp('WARNING: |correlation| > |signal|');
    disp(sprintf('MZMP: %g = abs(MaxC(1)) > normx = %g',...
	abs(MaxC(1)),normx)); 
  end;
  DisplayParams('MZMP: est. |Rx|/|x| = ',RxoxE(1),1);
  disp('MZMP:        ...STEP I. complete.');
  disp('---------------------------------');
  disp('MZMP: Step II. Update and Iterate');
end; n=0; 
while n < MAXITERS,
  n = n+1;
  if(DEBUG), disp(sprintf('MZMP: ~~~~~~~ iteration %d ~~~~~~~',n)); end;
  ip = IP(K,MaxP(n,:),P',1,0);    % 1 x NAtoms vect of inner prods
  C = C - MaxC(n) .* ip;          % update correlations
  [maxC,maxCi]=max(abs(C));       % get max abs corr. value and index
  MaxC(n+1) = C(maxCi);  MaxP(n+1,:) = P(maxCi,:);
  if(DEBUG), 
    cip = MaxC(n)*ip(maxCi);
    DisplayParams('MZMP: (before opt) MaxP(n+1,:) = ',MaxP(n+1,:),4);
    DisplayParams('MZMP:              MaxC(n+1) = ',MaxC(n+1),1);
    disp(sprintf('MZMP:              MaxC(%d)*ip(%d) = (%g, %g)',...
	n,maxCi,real(cip),imag(cip)));
  end;

  %%% II.b. OPTIMIZE %%% (only for non-Dirac, modulated atoms)
  if(MaxP(n+1,1)>0 & MaxP(n+1,3)>0),  % non-Dirac atoms with pos. freq.
    j0 = MaxP(n+1,1); pj0 = pow2(j0);  
    u0 = MaxP(n+1,2); f0 = MaxP(n+1,3); K0 = MaxP(n+1,4);
    fopt = f0 + (-OPT:OPT)/(pow2(OPT)*pj0);
    if(DEBUG & find(fopt > 1)), error('freq must be <= 1'); end;

    [g, Ks] = gaborMZ(K,j0,u0,fopt);  % Nopts rows, one atom per row
    Copt = x*g';                      % 1 x Nopts row of ips
    Popt = [j0*ones(Nopts,1) u0*ones(Nopts,1) fopt' K0*ones(Nopts,1)];
    ip = IP(K,MaxP(1:n,:),Popt',0,0); % n x Nopts matrix of inner prods
    Copt = Copt - MaxC(1:n)*ip;       % update correlations
    [maxCopt,maxCoptI]=max(abs(Copt));% get opt max corr value and index
    if(maxCopt > maxC),
      MaxC(n+1) = Copt(maxCoptI); MaxP(n+1,3) = fopt(maxCoptI); 
      % MaxP(n+1,4)=Ks; % (no change)
      if(DEBUG),
	hNopts = (Nopts+1)/2;
	cip0 = MaxC(n)*ip(n,hNopts); cip1 = MaxC(n)*ip(n,maxCoptI);
	disp(sprintf('MZMP: (compare) |MaxC| = %g, |MaxCopt| = %g',...
	    maxC,maxCopt));
	DisplayParams('MZMP: (after opt)  MaxP(n+1,:) = ',MaxP(n+1,:),4);
	DisplayParams('MZMP:              MaxC(n+1) = ',MaxC(n+1),1);
	disp(sprintf('MZMP:              MaxC(%d)*ip(%d)=(%g, %g)',...
	    n,hNopts,real(cip0),imag(cip0)));
        disp(sprintf('MZMP:              MaxC(%d)*ip(%d)=(%g, %g)',...
	    n,maxCoptI,real(cip1),imag(cip1)));
      end;
    end;
  end;
  %% Relative norm of estimated remainder %%
  normDiff = normRx^2 - abs(MaxC(n+1))^2;
  normRx = sqrt(abs(normDiff)); RxoxE(n+1) = normRx/normx;
  if(DEBUG),
    if(normDiff < 0),
      disp('WARNING: |correlation| > |remainder|');
      disp(sprintf('MZMP: %g = abs(MaxC(%d)) > normRx = %g',...
	  abs(MaxC(n+1)),n+1,normRx)); 
    end;
    DisplayParams('MZMP: est. |Rx|/|x| = ',RxoxE(n+1),1);    
    disp(sprintf('MZMP: iteration %d complete. hit any key...',n));pause;
  end;
  if( RxoxE(n+1) < TOLERANCE ),
    MaxC = MaxC(1:n+1);  MaxP = MaxP(1:n+1,:);  RxoxE = RxoxE(1:n+1); 
    if(DEBUG),
      disp(sprintf('tol. reached at iteration %d:  |Rx|/|x| < %1.6f',...
	  n+1,TOLERANCE));
    end;
    n = MAXITERS; % terminate while loop
  end;
end;

if(DEBUG==1), xEst = Atoms2Sig(K,MaxC,MaxP); end; % construct est signal
DISP=1;
if(DEBUG & (DISP > 0)),
  figure(2); clf; % plot: signal, estimated signal, and error
  subplot(3,1,1); plot(real(x)); grid; axis([0,512,-1.0,1.0]);
  subplot(3,1,2); plot(real(xEst)); grid; axis([0,512,-1.0,1.0]);
  subplot(3,1,3); plot(real(x)-real(xEst));grid; axis([0,512,-1.0,1.0]);
  figure(3); clf; % plot: signal, estimated signal, and error
  subplot(3,1,1); plot(angle(x));
  subplot(3,1,2); plot(angle(xEst));
  subplot(3,1,3); plot(angle(x)-angle(xEst));
  figure(4); clf; % norm of error at each iteration
  if(DEBUG>1),
    subplot(2,1,1);stem(real(RxoxT));
    subplot(2,1,2);stem(real(RxoxE));
  else,
    stem(real(RxoxE));
  end;
end;

%% end of file: MZMP.m
\end{verbatim}

\subsection{Wigner Transform}\label{sec:WVTrans_AFC}
\begin{verbatim}
function afwig = WVTrans_AFC(sig1,sig2,DISP)
% WVTrans_AFC -- Alias-Free Cross Wigner Transform
%  Usage
%    afwig = WVDist_AFC(sig1,sig2,1)
%  Inputs
%    sig1    1-d signal
%    sig2    1-d signal
%    DISP    DISP=1 => produce image plot of Wigner Transform
%            DISP=0 => do not produce image plot (default)
%  Outputs
%    afwig   complex-valued matrix representing the alias-free
%            cross Wigner-Ville distribution of zero-extended signal
%            with rows corresponding to frequencies and columns
%            corresponding to times.
%
%  Side Effects
%    Image Plot of the alias-free Wigner-Ville distribution 
%    (if DISP>0)
%
%  See Also
%    WVDist, WVDist_AF, ImagePhasePlane
%
%  References
%   Jechang Jeong and William J. Williams,
%   "Alias-Free Generalized Discrete-Time Time-Frequency Distribution,"
%   IEEE Transactions on Signal Processing, vol. 40, pp. 2757-2765.
%
if nargin < 2,
  WVDist_AF(sig1); 
else,
  if nargin < 3, DISP=0; end;
  
  sig1 = sig1(:); sig2 = conj(sig2); sig2 = sig2(:);
  n = length(sig1);
  if(n~=length(sig2)), error('signals must have equal length'); end;
  zerosn = zeros(n,1);
  f1 = [zerosn; sig1; zerosn]; f2 = [zerosn; sig2; zerosn];
  afwig = zeros(n, n);
  ix  = 0:(n/2-1);
  
  for t=1:n,
    tplus  = n + t + ix;
    tminus = n + t - ix;
    x = zerosn;
    % even indices
    x(1:2:n) = f1(tplus) .* f2(tminus); 
    % odd indices    
    x(2:2:(n)) = (f1(tplus+1).*f2(tminus) + f1(tplus).*f2(tminus-1))/2; 
    afwig(:, t) = 2* (fftshift(fft(x)));
  end

  %% display
  if(DISP),
    abstf = real(afwig);
    tfmax = max(max(abstf));
    tfmin = min(min(abstf));
    colormap(1-gray(256))
    image(linspace(0,1,n),linspace(0,1,n),256*(abstf-tfmin)/(tfmax-tfmin));
    axis('xy')
    title('Alias Free Wigner Distribution');
    xlabel('Time')
    ylabel('Frequency')
  end;
end;

%%% 
% MODIFIED (to handle *cross* Wigner transform)
% Author: William DeMeo <william.demeo@verizon.net>
% Date: 2001 May 25
% Copyright (c) 2001, William DeMeo
%%%

% Copyright (c) 1994-5, Shaobing Chen

% Part of WaveLab Version 802
% Built Sunday, October 3, 1999 8:52:27 AM
% This is Copyrighted Material
% For Copying permissions see COPYING.m
% Comments? e-mail wavelab@stat.stanford.edu

%% end of file: WVTrans_AFC.m
\end{verbatim}   

\subsection{Gabor Atoms}\label{sec:gaborMZ}
\begin{verbatim}
function [g, Ks] = gaborMZ(k,s,p,f,Disp,DEBUG)
%
% Matlab code for computing complex gabor atom used in Matching
% pursuit algorithm of Mallat and Zhang
%
% Inputs
%    k    (default k=14) gabor window (atom) will have 2^(k+1) 
%          sample values.  Window half-length is w = 2^k
%    s    (default s=k+1)  scale is S = 2^s 
%         (S = effective support = # samples for which g > eps)
%    p    (default p=0 <--> center at w = 2^k.)
%          integer in {-2^(k+1)+1,...,-1,0,1,2,...,2^(k+1)-1} 
%          equal to amount by which window is translated
%    f    (default f=0 <--> no modulation)
%          real number in {0,2,...,2^(k+1)-1}/2^(k+1)
%          specifying the *normalized* modulation frequency
%    Disp (default Disp=0 <--> do not produce plot)
%          indicator to decide whether to plot window 
%    DEGUB (default DEBUG=0 <--> do not print debugging info)
%          indicator to decide whether to print debugging info
%
% Outputs
%    g    complex row vector of length 2^(k+1) containing atom values
%    Ks   the normalization constant by which g was multiplied
%
% Examples
% >> g = gabor; plot(real(g)); % default (512 sample points)
%
% Author: William DeMeo <william.demeo@verizon.net>
% Date: 2001 May 4
% Copyright (c) 2001, William DeMeo

acnt=1;      if nargin < acnt,  k = 14;   end;
acnt=acnt+1; if nargin < acnt,  s = k+1;  end;
acnt=acnt+1; if nargin < acnt,  p = 0;    end;
acnt=acnt+1; if nargin < acnt,  f = 0;    end;
%% (obsolete) acnt=acnt+1; if nargin < acnt,  phi = 0; end;
acnt=acnt+1; if nargin < acnt,  Disp = 0; end;
acnt=acnt+1; if nargin < acnt,  DEBUG = 0;end;

%%% define some constants %%%
S = pow2(s);            % scale
w = pow2(k);            % window half-length
W = pow2(k+1);          % window length (samples per window)

%%% test arguments %%%
if ((s < 0) | (s > k+1)),
 error('scale argument s ust be in {0,..., k+1}');
end
if ((p<=-W) | (p>=W)),
 error(sprintf('p == %g; should have -2^(k+1) < p < 2^(k+1)',p));
end
if ((f<0) | (f>1))
  error('frequency argument f must be in {0,2,...,2^(k+1)-1}/2^(K+1)');
end

if (s==0),   % Special Case: discrete Diracs
  if(DEBUG),disp('s==0');end;
  g = zeros(1,W);  
  p = mod(p+W,W); flp = floor(p); % ensure p is a positive integer
  imp = mod(w+flp-1,W)+1;         % center of Dirac impulse
  g(imp)=1; Ks=1;
else,
  n=(-w+1:w);                % discrete time parameter
  if (s < k+1),              % s must be in {1,...,k}
    g = (2^(.25)/sqrt(S)).*exp(-pi*((n/S).^2));
    gp=g;
    if (p), % translate right by p samples, otherwise leave it equal to g
      p=mod(p+W,W); flp=floor(p); % ensure p is a positive integer  
      indx = mod((1:W)+flp-1,W)+1;
      gp(indx) = g;
    end; 
    %%% if(DEBUG),disp('s<k+1');disp(sprintf('norm(gp) = %g',norm(gp)));
  else,      % Special Case: Fourier basis of complex exponentials
    if(DEBUG),disp('s==k+1');end;
    gp=ones(1,W);
  end;
  Ks = 1/norm(gp);    % normalization constant
  lf = length(f);
  if(lf == 1),  
    gp = Ks .* gp;               % normalize
    g = gp .* exp(i*2.0*pi*f*n); % modulate
  else,
    %   new code to allow handling multiple frequencies
    gp = Ks .* gp;      % normalize
    % OLD METHOD: (too slow)
    % f = f(:);   gp = Ks .* diag(gp);% normalize
    % g = exp(i*2.0*pi*f*n)*gp;       % modulate
    g = zeros(lf,W);
    for (mm = 1:lf),
      g(mm,:) = gp .* exp(i*2.0*pi*f(mm)*n); % modulate
    end;
  end;
end;

if Disp > 0,
  if(k<6), 
    figure(1); clf; stem(real(g)); 
  else,
    figure(1); clf; plot(real(g)); 
  end;
end;

%% end of file: gaborMZ.m
\end{verbatim}   

\subsection{Inner Product of Gabor Atoms}\label{sec:IP}
\begin{verbatim}
function ip = IP(K,P1,P2,ALL2,DEBUG)
%
% Matlab code for computing inner product of two Gabor atoms with the
% specified parameters.
%
% Inputs
%    K      2^(K+1) is the size of the entire domain
%    P1     a real N1 x 4 vector of atom parameters, where
%           P1(j,1) = s1   the scale of atom j is 2^s
%           P1(j,2) = p1   amount by which atom j is translated in time
%           P1(j,3) = f1   (normalized) modulation freq. of atom j
%           P1(j,4) = Ks1  normalization constant (unused in IP.m)
%    P2     a 4 x N2 paramter vector similar to transpose of P1.
%    ALL2   specifies whether 3rd argument, P2, represents all atom
%           parameters (used to optimize algorithm)
%    DEBUG  (default 0) DEBUG=1 ==> print extra debugging information
%
% Outputs
%    ip     a N1 x N2 matrix of inner products between atoms of P1
%           and atoms of P2. 
%
% See also: IPDirac.m
%
% Author: William DeMeo <william.demeo@verizon.net>
% Date: 2001 April 13
% Copyright (c) 2001, William DeMeo

nargs=3;       if nargin < nargs, error('USAGE: IP(K,P1,P2,...)'); end; 
nargs=nargs+1; if nargin < nargs, ALL2=0; end; 
nargs=nargs+1; if nargin < nargs, DEBUG=0; end;

N1=size(P1,1); N2=size(P2,2); ip=zeros(N1,N2);
N = pow2(K+1);  w = pow2(K);  Nt4 = 4*N; %g2=zeros(1,N); g1=zeros(1,N);
maxexp = 11.4731;  % const. for deciding if inner prod is negligeable
                   % approx = (1/pi)*log(1/eps); 

if(DEBUG),disp(sprintf('IP: computing %d inner products...',N1*N2)); end;

for n1=1:N1,
  if (P1(n1,1)==0), %%% 1st arg is a Dirac (use conjugate) %%%
    next2=1;
    if(ALL2), %%% first 2^(K+3)=4*N atoms of 2nd arg are also Diracs %%%
      ip(n1,1:Nt4) = (floor(P1(n1,2))==floor(P2(2,1:Nt4)));
      next2=Nt4+1;
    end;
    for n2=next2:N2,
      ip(n1,n2) = IPDirac(K,P1(n1,2),P2(:,n2),1);
    end;
  else, %%% 1st arg is NOT a Dirac %%%
    p1=P1(n1,2); f1=P1(n1,3); K1=P1(n1,4);
    nd = 1:N; argI=(nd'-w);
    if (P1(n1,1)==K+1), %%% 1st arg is a Fourier basis element %%%
      s1=sqrt(2); 
      argR1=zeros(1,N); indxs1=1:N;
    else,
      s1=pow2(P1(n1,1)); 
      argR1=((mod(nd-p1+N-1,N)+1-w)/s1).^2; indxs1=find(argR1<maxexp);
    end;
    %% if(indxs1), %% only proceed if where inner prod is non-negligeable
    %% But this always holds (proof: the following line never executes)
    %% if(length(indxs1)==0),disp('no non-negligeable indices');end;
    if (ALL2), 
      %%% first 2^(K+3)=4*N atoms of 2nd arg are Diracs %%%
      ip(n1,1:Nt4) = IPDirac(K,P2(2,1:Nt4),P1(n1,:),0);
      %%% atoms of 2nd arg which are neither Dirac nor complex exp %%%
      for n2=(Nt4+1:N2-Nt4),
	s2=pow2(P2(1,n2)); p2=P2(2,n2); df=f1-P2(3,n2); K2=P2(4,n2);
	argR2 = ((mod(nd-p2+N-1,N)+1-w)/s2).^2;
	% of indxs1 (support of 1st atom), find those in support of 2nd
	indxs2 = find(argR2(indxs1)<maxexp);
	indxs = indxs1(indxs2);
	if(indxs),
	  %%% NB: * operator implicitly takes conjugate of 2nd arg
	  %%%     WHEN THE 2ND ARG IS TRANSPOSED
	  ip(n1,n2)=exp(-pi.*(argR1(indxs)+argR2(indxs)))*...
	      exp(i*2.0*pi*df.*argI(indxs));
	  ip(n1,n2)=ip(n1,n2)*K1*K2*sqrt(2)/sqrt(s1*s2);
	end;
      end;
      for n2=(N2-Nt4+1:N2),%%% last 4*N atoms of 2nd arg are compl exp
	df=f1-P2(3,n2); K2=P2(4,n2); argR2=zeros(1,N); s2=sqrt(2);
	ip(n1,n2)=exp(-pi.*argR1(indxs1))*exp(i*2.0*pi*df.*argI(indxs1));
	ip(n1,n2)=ip(n1,n2)*K1*K2*sqrt(2)/sqrt(s1*s2);
      end;
    else, %%% ALL2==0 => we don't know form of 2nd argument %%%
      for n2=1:N2,
	if (P2(1,n2)==0),%%% 2nd arg is Dirac (don't use conj) %%%
	  ip(n1,n2) = IPDirac(K,P2(2,n2),P1(n1,:),0);
	else,  %%% 2nd argument is NOT a Dirac %%%
	  p2=P2(2,n2); df=f1-P2(3,n2); K2=P2(4,n2);
	  if (P2(1,n2)==K+1), %%% 2nd arg is a complex exponential %%%
	    s2=sqrt(2); argR2=zeros(1,N); indxs = indxs1;
	  else,
	    s2=pow2(P2(1,n2)); 
	    argR2=((mod(nd-p2+N-1,N)+1-w)/s2).^2;
	    indxs2 = find(argR2(indxs1)<maxexp); indxs = indxs1(indxs2);
	  end;
	  if(indxs),
	    ip(n1,n2)=exp(-pi.*(argR1(indxs)+argR2(indxs)))*...
		exp(i*2.0*pi*df.*argI(indxs));
	    ip(n1,n2)=ip(n1,n2)*K1*K2*sqrt(2)/sqrt(s1*s2);
	  end;
	end;  %~~ end test for Dirac (2nd arg)
      end;  %~~ end for n2=1:N2 loop
    end;  %~~ end test for ALL2 
  end; %~~ end test for Dirac (1st arg)
end; %~~ end for n1=1:N1 loop

%% end of file: IP.m
\end{verbatim}   

\subsection{Inner Product with Dirac Atoms}
\begin{verbatim}
function ipd = IPDirac(K, p1, P2, CONJ)
%
% Matlab code for computing inner product of Dirac impulse(s) located
% at points specified in p1, with an atom having parameters specified
% in P2.
%
% Inputs
%   K    the signal size is 2^(K+1).
%   p1   point(s) specifying the center(s) of the Dirac impulse(s).
%   P2   parameter vector specifying the atom.
%   CONJ a boolean variable to indicate whether the Dirac is meant to
%        be first or second argument of inner product.  CONJ==1 if
%        Dirac is first argument to inner product because, in that
%        case, we take the conjugate of the atom (which is the 2nd
%        argument to the inner product in this case).
% Outputs
%   ipd  a 1 x length(p1) vector of inner products of the atom with
%        Dirac impulses having centers specified by p1.

N = pow2(K+1); w = pow2(K);
f2 = P2(3); K2 = P2(4);
flps = floor(p1);               % ensure integer valued centers
imps = mod(w+flps-1,N)+1;       % center(s) of Dirac impulse(s)

if (CONJ), konj=-1; else, konj=1; end; 

if (P2(1)==K+1),  % 2nd arg is a complex exponential
  ipd = K2*exp(konj*i*2.0*pi*f2*(imps-w));
elseif (P2(1)>0), % 2nd arg is neither Dirac nor complex exponential,
  s2 = pow2(P2(1)); p2 = P2(2); 
  argR2=((mod(imps-p2+N-1,N)+1-w)/s2).^2; 
  ipd = zeros(1,length(p1));
  indxs = find(argR2 < 11.4731); 
  ipd(indxs) = (K2*2^(.25)/sqrt(s2)).*...
       exp(-pi.*argR2(indxs)).*exp(konj*i*2.0*pi*f2*(imps(indxs)-w));
else,             % 2nd arg is also a Dirac
  ipd = (flps==floor(P2(2)));
end;

%% end of file: IPDirac.m
\end{verbatim}   


