\section{Sound Examples}
This section presents some notes on the music, used in the
experiments of section~\ref{sec:practical}.

\subsection{\emph{Metamorphosis} (1988), by Philip
Glass\protect\footnotemark} 
\footnotetext{This brief presentation is based on that of Kerman and
Tomlinson~\cite{Kerman:2000}} 
\begin{define}{\bf Philip Glass (b.~1937) }
%studied at the University of Chicago and with
%Persichetti, Bergsma, and N. Boulanger and
may be the most wide-ranging of the minimalist composers.  He plays
keyboards in a crossover group, the Philip Glass Players, which has
appeared in concert halls and art museums, at rock clubs and jazz
festivals.  He has written music for some remarkable art films; he
collaborates with the innovative multimedia artist Robert Wilson.  The
easy-to-play piano piece that this paper examines makes a much quieter
statement than the famous Glass pieces such as his operas and ensemble
numbers written for his group.  Nevertheless, \emph{Metamorphosis} is
typical of the composer's work in its constructive principles and
overall effect.

\emph{Metamorphsis} consists of five very similar movements of
approximately the same length.  In \emph{Metamorphosis 1}, three brief
elements alternate, without transitions of any kind to link them.  The
first element, {\bf a}, announces four chords in a slow dotted rhythm;
the second, {\bf b}, introduces gentle motion, oscillating quietly
between two notes of a simple chord.  This music is utterly
uneventful, except that the bass comes in halfway through the
four-measure span.  The third element, {\bf c}, is a fragment of
melody -- repetitive, plaintive, built over harmonies similar to those
of {\bf a}, and with a fluid accompaniment continuing from {\bf b}.
The dynamic never rises above mezzo forte.  

This simple plan for \emph{Metamorphosis 1} can be represented
\newcommand{\Aa}{\mathbf{a}}
\newcommand{\Bb}{\mathbf{b}}
\newcommand{\Cc}{\mathbf{c}}
\[ ||: \Aa \; \Bb \; \Aa \; \Bb \; \Cc \; \Bb \; \Cc \; \Bb \; \Cc \; \Bb :|| \; \Aa' \; \Aa' \; \Aa' \; \Bb \; \Bb \; \Bb 
\mbox{ (repeated three times)}\]
where $\Aa'$ is an extension of the four-chord series to five -- a
telling change in these inert surroundings.  Notice that although the
ending of the chord series features rich and rather melancholy chords,
the way the three elements are juxtaposed somehow leaches all the
expressivity out of them.  One waits fascinated for that inevitable
bass entry in b.
\end{define}