%
\section{Summary and Conclusions}
Basic DSP theory was reviewed with a focus on 
\emph{translation invariance} -- translation invariant
operators of $\LG$, and translation invariant
subspaces of $\LG$. 
When such great significance is attached to translation invariance, 
a deeper understanding of exponential functions, and their 
unrivaled status in classical DSP, is possible.  
In particular, exponentials are the \emph{characters} 
of abelian group indexing sets, such as $\Z/N$, over which
classical DSP is performed.  
Each character of an abelian group represents a one-dimensional
translation invariant subspace, and the characters are eigenvectors
of translations, therefore, of convolutions. 
%(which explains why expansions in the character basis -- \eg
%classical Fourier transform -- diagonalize the convolution
%product).  

We described the group algebra $\CG$,
the algebra isomorphism $\LG \simeq \CG$, and why it is
useful for manipulations involving (generalized)
translations and convolutions of the space of signals. 
We saw that, for an abelian group $A$, translations of $\LA$
represent simple linear shifts in space or time, while for a nonabelian
group $G$, translations of $\LG$ are more general than
simple spatial or temporal shifts.  This leads to
more interesting translation invariant subspaces.

Motivating many studies in the area of noncommutative
harmonic analysis (including this one) is a simple
but important fact about the generalized translations that
result when a signal is indexed by a nonabelian group.
As we have seen, such operations offer more complex and
interesting signal transformations.  Equally important,
however, is the fact that each transformation can be written
as a left-multiplication. Thus, the increase in signal transform
complexity resulting from a nonabelian indexing scheme
comes at no increase in computational complexity. 


% if have a single appendix:
%\appendix[Proof of the Zonklar Equations]
% or
%\appendix  % for no appendix heading
% do not use \section anymore after \appendix, only \section*
% is possibly needed

% use appendices with more than one appendix
% then use \section to start each appendix
% you must declare a \section before using any
% \subsection or using \label (\appendices by itself
% starts a section numbered zero.)
%
% Use this command to get the appendices' numbers in "A", "B" instead of the
% default capitalized Roman numerals ("I", "II", etc.).
% However, the capital letter form may result in awkward subsection numbers
% (such as "A-A"). Capitalized Roman numerals are the default.
%\useRomanappendicesfalse
%
%\appendices
%\section{Proof of the First Zonklar Equation}
%Appendix one text goes here.

% you can choose not to have a title for an appendix
% if you want by leaving the argument blank
%\section{}
%Appendix two text goes here.

% use section* for acknowledgement
\section*{Acknowledgment}
% optional entry into table of contents (if used)
%\addcontentsline{toc}{section}{Acknowledgment}
The author would like to thank Textron Systems and the U.S.~Navy
for supporting this research.  This work also owes a great deal to Myoung
An and Richard Tolimieri, whose prior contributions to this field are
responsible for anything of value in the present work. The author thanks them
for many helpful conversations and suggestions, but takes full responsibility for any
errors that appear above.
