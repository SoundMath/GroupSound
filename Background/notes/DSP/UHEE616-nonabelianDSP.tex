
%=======================================================================
\section{Nonabelian Group DSP}
%=======================================================================
This section presents some basic principles of
DSP, but relies on a more general mathematical formalism
than that commonly found in textbooks on the subject.\footnote{A few notable
  exceptions are~\cite{An:2003, Tolimieri:2003, Tolimieri:1998, Chirikjian:2002}.}
%\cite{{An:2003},{Chirikjian:2001},{Tolimieri:1998},{Tolimieri:1997}}}.
%the book by Chirikjian and Kyatkin~  %and the books by Tolimieri and An

%------------------------------------------------------------------------
\subsection{The Group of Characters: Main Theorems}
%------------------------------------------------------------------------
A \emph{character} of $G$ is a group homomorphism of $G$
into $\C^{\times}$, where $\C^{\times} = \C \setminus \{0\}$.
In other words, the mapping $\varrho: G \to \C^{\times}$ is 
a character of $G$ if it satisfies 
$\varrho(xy) = \varrho(x)\varrho(y), \, x, y \in G$.
%There is always at least one character, the 
%\emph{trivial character}, which is 1 for all $y\in G$. 
Let $G^*$ denote the set of all characters of $G$.

By the identification~(\ref{eq:iso}) between $\LG$ and
$\CG$, a character $\varrho \in G^*$ can be viewed as
a formal sum,
\begin{equation}
\varrho = \sum_{x\in G}\varrho(x)x.
\end{equation}
Therefore, $G^* \subset \CG$. 
Expressing the characters as formal sums 
leads to simple proofs of important DSP results.
%%%
%%% T-1
%%%
\begin{theorem}\label{thm:char-action}     
If $\varrho$ is a character of $G$, then
\begin{equation}
y\varrho = \varrho y = \varrho(y^{-1})\varrho, \quad y\in G.
\end{equation}
\end{theorem}
\begin{proof}
%%%
%%% P-1
%%%
By a change of variables,
\[
\varrho y = \sum_{x\in G}\varrho(x)xy = \sum_{x\in G}\varrho(xy^{-1})x, \quad y\in G.
\]
By homomorphism property, $\varrho(xy^{-1}) =
\varrho(x)\varrho(y^{-1})$.  Therefore, 
\[
\varrho y %=\sum_{x\in G}\varrho(xy^{-1})x 
= \sum_{x\in G}\varrho(x)\varrho(y^{-1})x
= \varrho(y^{-1})\varrho, \quad y\in G.
\]
A similar change of variables argument shows
\[
y\varrho %= \sum_{x\in G}\varrho(x)yx
= \sum_{x\in G}\varrho(y^{-1}x)x
= \varrho(y^{-1})\varrho, \quad y\in G.
\]
%As above, write $\varrho$ as a formal sum and change
%variables.
%% \qed
\end{proof}
%\begin{proof}By a change of variables,\begin{equation}
%\varrho y = \sum_{x\in G}\varrho(x)xy = \sum_{x\in G}\varrho(xy^{-1})x, \quad y\in G.
%\end{equation}By homomorphism property, $\varrho(xy^{-1}) =
%\varrho(x)\varrho(y^{-1})$.  Therefore, $\varrho y = \varrho(y^{-1})\varrho$, for all $\quad y\in G$.
%The same change of variables argument works for $y\varrho$.\qed\end{proof}
%%%
%%% T-2
%%%
\begin{theorem}\label{thm:eigenvector}             
%If $\varrho\in G^*$ is a nontrivial character of $G$, then
For $\varrho\in G^*$,
\begin{equation}
\frac{1}{|G|} \sum_{x\in G} \varrho(x) = 
\begin{cases}  1, & \varrho(x)=1, \forall x\in G,\\
0, & \text{otherwise.}
\end{cases}
\end{equation}
\end{theorem}
where $|G|$ is the order of $G$.
%%%
%%% P-2
%%%
\begin{proof}
%%Let $y\in G$ be such that $\varrho(y) \neq 0$, and $\varrho(y) \neq 1$.  (If no
%%such $y$ exists, the result is obvious.)
By a change of variables,
\begin{equation}
\varrho(y) \sum_{x\in G} \varrho(x) = \sum_{x\in G}
\varrho(yx) = \sum_{x\in G} \varrho(x), \quad y\in G.
\end{equation}
Thus, either $\varrho(x)=1, \forall x\in G$, 
or $\sum \varrho(x)=0$.
\end{proof}
%\qed 

Theorem~\ref{thm:char-action} shows that every character is
an eigenvector of left-multiplication by elements of the
group $G$, so we call them $\lt{L}(G)$-eigenvectors.
Therefore, by linearity, the characters are eigenvectors
of left-multiplication by $f\in \CG$ (convolution by
$f\in \LG$).  This is re-stated more formally as the following
formula for the $G$-\emph{spectral components} of $f$:
\begin{corollary}\label{cor:FT}
If $\varrho\in G^*$ and $f\in \CG$, then
\begin{equation}\label{eq:FT}
f\varrho = \varrho f = \hat{f}(\varrho)\varrho,
\end{equation}
where $\hat{f}(\varrho) = \sum_{y\in G} f(y) \varrho(y^{-1})$.
%\end{cor}
\end{corollary}
\begin{proof}
By Theorem~\ref{thm:char-action},
\begin{equation}
f\varrho = \sum_{y\in G} f(y) y\varrho =  \sum_{y\in G} f(y)\varrho(y^{-1}) \varrho
\end{equation}
Similarly for $\varrho f$, mutatis mutandis.
%\qed
\end{proof}

The functions which make up the standard Fourier basis
% -- the exponential functions -- 
are eigenvectors of standard convolution.  
As seen in the foregoing proofs, % of~\ref{thm:eigenvector}, 
this is merely a consequence of the fact that the
exponential functions are %satisfy properties which allow usto call them 
characters.  \emph{The notion of a character basis
generalizes the Fourier basis to include bases which
can diagonalize any linear combination of 
left group multiplications.}
%%% LEFT OFF with notes on p. 132 %%%
\begin{corollary}\label{cor:idemp}
If $\lambda, \tau \in G^*$, then
\begin{equation}
\lambda \tau = 
\begin{cases}  |G|\lambda, & \tau=\lambda,\\
0, & \tau\neq\lambda.
\end{cases}
\end{equation}
\end{corollary}
\begin{proof}
Suppose $\tau = \lambda$; then,
%\begin{eqnarray*}
\[\lambda \tau %= \lambda\lambda &=& \sum_{x\in G} \tau(x)x\tau\\
= \sum_{x\in G} \lambda(x)\lambda(x^{-1})\lambda
= \sum_{x\in G} \lambda(1)\lambda = |G|\lambda
\]%\end{eqnarray*}
Suppose $\tau \neq \lambda$. By definition,
\begin{equation}
\hat{\lambda}(\tau) = \sum_{x\in G}\lambda(x)\tau(x^{-1}) =
\sum_{y\in G}\lambda(y^{-1})\tau(y) = \hat{\tau}(\lambda)
\end{equation}
By~(\ref{eq:FT}), 
$\hat{\lambda}(\tau)\tau = \lambda\tau = \tau\lambda =\hat{\tau}(\lambda)\lambda $.  
Since $\hat{\lambda}(\tau) = \hat{\tau}(\lambda)$ and $\tau
\neq \lambda$, it must be the case that
$\hat{\tau} = 0$ and $\lambda\tau = 0$.
%\qed
\end{proof}
Corollary~\ref{cor:idemp} can be expressed in the language
of \emph{idempotent theory}.  A nonzero element $e \in \CG$
is called an \emph{idempotent} if $e^2 = e$. Two idempotents
$e_1$ and $e_2$ are called orthogonal if
$e_1e_2 = e_2e_1 = 0$.
Corollary~\ref{cor:idemp} says that 
\[
\left\{\frac{1}{|G|}\rho : \rho \in G^* \right\}
\]
is a set of pairwise orthogonal idempotents.

