%------------------------------------------------------------------------
\ismasubsec{Translation-Invariant Subspaces}
%------------------------------------------------------------------------
A subspace $\vs{V}$ of the space $\CG$ is called a
\emph{left ideal} if 
\begin{equation}
u\vs{V} = \{uf : f \in \vs{V}\} \subset \vs{V}, \quad u \in G. 
\end{equation}
A left ideal of $\CG$ corresponds to a subspace of $\LG$
invariant under all left translations.  

If $\vs{V}$ is a
left ideal, then, by linearity, 
$g\vs{V} \subset \vs{V}$ for all $g \in \CG$.
The set $\CG g$, defined by 
$\{fg : f \in \CG\}$, is a left ideal of $\CG$,
called \emph{the left ideal generated by} $g$ in $\CG$. 
%$\CG g = \CG$ if and only if $g$ is an invertible element in $\CG$. 
A left ideal $\vs{V}$ of $\CG$ is called \emph{irreducible}
if the only left ideals of $\CG$ contained in $\vs{V}$ are
$\{0\}$ and $\vs{V}$. The sum of two distinct, irreducible
left ideals is always a direct sum. 
% (\cite{An:2003}, p.~129).

For \emph{abelian} group $A$, the group algebra $\C A$ of
signals is decomposed into a direct sum of irreducible ideals.  
Since multiplication of $\C A$ by elements of $G$
corresponds to translation, ideals represent
translation-invariant subspaces.  Furthermore, in the 
abelian case, such translation-invariant subspaces are
one-dimensional.   

Similarly, for \emph{nonabelian} group $G$, the group algebra
$\CG$ is decomposed into a direct sum of left ideals
and, again, the ideals are translation-invariant
subspaces.  However, some of them must now be multi-dimensional,
and herein lies the potential advantage of using nonabelian
groups for indexing the data. The left translations
are more general and represent a broader class of 
transformations. Therefore, projections of data into the
resulting left ideals can reveal more complicated partitions
and structures as compared with the Fourier components in
the abelian group case. 

