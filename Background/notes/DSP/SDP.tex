% This is the master sdp.tex file -- the most
% up-to-date -- the one on which all others are based.

% UHEE616-sdp.tex integrated on 2005.03.16

%----------------------------------------------------------------------
\subsection{Semidirect Product Groups}
%----------------------------------------------------------------------
To determine whether a particular group is useful for a DSP
application, we must specify exactly how this group
represents the data.
The group representation may reduce computational
complexity, or it may simply make it easier to state,
understand, or model a given signal processing task.

This section describes %procedures for specifying and studying 
a simple class of nonabelian groups that have
proven useful in applications -- 
\emph{abelian by abelian semidirect products}. 
%These are
%perhaps the simplest extension of abelian groups
%% case, these are
%%groups of the form $G = A \sdp B$, where $A$ and $B$
%%are abelian groups. Not surprisingly, 
%and DSP over such groups closely resembles that over abelian
%groups.  However, the resulting processing tools can have
%vastly different characteristics. 

%%% LEFT OFF HERE --> resume from top of page 109
%%% LEFT OFF with notes on p. 132 %%%

%-----------------------------------------------------------------------
%\ismasubsec{Action Group}
%-----------------------------------------------------------------------
Let $G$ be a finite group of order $N$, $K$ a subgroup of $G$,
and $H$ a normal subgroup of $G$. If $G = HK$ and $H \cap
K = \{1\}$, then we say that $G$ is the 
%\emph{internal semidirect product}
\emph{semidirect product} $G = H \sdp K$. 
It can be shown that $G = H \sdp K$ if and only if every $x \in
G$ has a unique representation of the form $x = yz, \; y\in H,
z\in K$.

Denote by $Aut(H)$ the set of all \emph{automorphisms} of
$H$. The mapping $\Psi:K\rightarrow Aut(H)$ defined by  
\begin{equation}\label{eq:homo}
\Psi_z(x) = zxz^{-1}, \quad z\in K, x\in H
\end{equation}
is a group homomorphism. 
Define the binary composition in $G$
%relative to the representation $G= H\sdp K$
in terms of $\Psi$ as follows:
\begin{equation}\label{eq:PsiProd}
x_1x_2 = (y_1z_1)(y_2z_2)= y_1\Psi_{z_1}(y_2)z_1z_2,
\end{equation}
\[
y_1, y_2 \in H,\; z_1, z_2 \in K. 
\]

If $K$ is a normal subgroup of $G$,
then $y^{-1}Ky = K$ for all $y\in G$, 
%(by definition of \emph{normal subgroup}) in which case, %$[H, K] = \{1\}$, 
and $G$ is simply the cartesian product $H\times K$ 
with component-wise multiplication. 
What is new in the semidirect product is
the possibility that $K$ acts nontrivially on $H$. 
For this reason, $K$ is sometimes called the ``action group.''

%-----------------------------------------------------------------------
\subsubsection{Simplest Nonabelian Example}
%-----------------------------------------------------------------------
If the mapping $\Psi$ given in~(\ref{eq:homo}) is defined over
$K=U(N)$, then $\Psi$ is a group isomorphism.
Under this identification, we can form the semidirect
product $G = H\sdp K$, with $H = C_N(x)$ and $K$ a
subgroup of $U(N)$.  Throughout this section, $G$ will
denote such a semidirect product group.

The elements $u\in K$ are integers. However, following~\cite{An:2003} we
denote by $k_u$ the element $u\in K$, as this avoids confusion that can arise 
on occasion. This notation is especially useful when $K$ 
is a cyclic group with generator $u$.  If  we denote elements of $K$ by $k_u^j$, instead of 
by $u^j$, it is easier to distinguish them from elements of the abelian group $C_N(x)$.

Suppose the action group $K$ is a cyclic group of order $J = |K|$ with generator
$u$. We identify each element of $K$ with an index, and denote 
the set of elements by $K = \{k_u^j: 0\leq j < J\}$.
Thus, to each $k_v \in K$, there corresponds a $j\in \Z$ 
such that $k_u^j=k_v$.
We use $x^n k_v$ and $x^n k_u^j$ to denote  
typical points of $G=C_N(x)\sdp K$.

Given two points in $G$, say $z = x^m k_u$
and $y=x^n k_v$, define multiplication %on $G$ 
according
to~(\ref{eq:PsiProd}) as follows:
\begin{equation}\label{eq:prod}
  zy = (x^m k_u)(x^n k_v) = x^{m+u n} k_u k_v,
\end{equation}
where %In~(\ref{eq:prod}), 
$m+ u n$ is taken modulo $N$.
Since $k_v=k_u^j$ for some $j\in \Z$, then 
$k_u k_v = k_u^{1+j}$, and $zy = x^{m+u n} k_u^{j+1}$.

Let $z = x^m k_v$ and suppose $k_w$ is the inverse 
of $k_v$ in $K$.  Then the inverse of $z$ must be
$z^{-1}=x^{N-wm} k_w$, since this satisfies
%\begin{equation}\label{eq:inv}z^{-1}=x^{N-wm} k_w.\end{equation}
$\inverse{z}z \equiv 1$.

Suppose $K \subset U(N)$ has order $|K|=J$, 
and consider the semidirect product group
with elements
\begin{equation}
  G   = \{x^n k_u^j : 0 \leq n < N, 0 \leq j < J\}.\label{eq:sdp}
\end{equation}
%\ismasubsec{Translations on Semidirect Product Groups}
For $f\in \CG$, %the formal sum can be expressed in the following ways:
\begin{equation}
  f = \sum_{y\in G} f(y)y= \sum_{n,j} f(x^n k_u^j)x^n k_u^j,
\end{equation}
%where $0\leq n <N$ and $0\leq j < J$.

As above, translations of $\CG$ are defined as
left multiplication by elements of $G$.  
For semidirect product~(\ref{eq:sdp}) 
there is a simple dichotomy of translation types that arise
from left-multiplication by elements of $G$.  %Translations of the
First, the familiar ``abelian translates'' 
are obtained upon left-multiplication by powers of $x$
(Fig.~\ref{fig:cyclicshift}).  
By change of variables, 
\begin{equation}\label{eq:1}
x^mf = \sum_{n,j} f(x^{n-m} k_u^j)x^n k_u^j,
\end{equation}
which is simply a right shift of $f$ by $m$ units.
Similarly, left-multiplication by powers of
$x^{-1}$ effects left shift of $f$. 
(Recall, $x^{-1} \equiv x^{N-1}$
and $x^{-m} \equiv x^{N-m}$.)  

Of the second type are the ``nonabelian translates,'' 
obtained upon left-multiplication by $k_v \in K$.
\begin{equation}
  k_vf %&=& \sum_{n,j} f(x^n k_u^j)k_v x^n k_u^j\nonumber\\
  = \sum_{n,j} f(k_v^{-1}x^n k_u^j)x^n k_u^j.
\end{equation}
Suppose $k_w = k_u^\ell$ is the inverse of $k_v$ in $K$.  Then,
\begin{equation}
  k_vf %  &=& \sum_{n,j} f(k_w x^{n}k_u^j)x^n k_u^j\nonumber\\
  = \sum_{n,j} f(x^{wn}k_u^{\ell + j})x^n k_u^j\label{eq:nonabtrans}
\end{equation}
%As usual, summation is over $0\leq n <N$ and $0\leq j < J$.
From equation~(\ref{eq:nonabtrans}) it is clear that $k_vf$ 
results in a more complex transformation than that of 
$x^m f$ as given by~(\ref{eq:1}).

%\newcommand{\zmv}{\ensuremath{x^m k_v}}
%\newcommand{\zmvInv}{\ensuremath{x^{N-wm} k_w}}

For the general element $z = \zmv \in G$ with 
inverse $z^{-1}=\zmvInv$ %(equation~(\ref{eq:inv})), 
we derive rules for generalized translations.
% with respect to $z$ and $z^{-1}$.
\begin{equation*}
  zf = \sum_{y\in G} f(z^{-1}y)y= \sum_{n,j} f(x^{N-w(m-n)} k_w k_u^j)x^n k_u^j
%  &=& \sum_{y\in G} f(y)zy = \sum_{y\in G} f(z^{-1}y)y\nonumber \\  
%  &=& \sum_{n,j} f(\zmvInv \, x^n k_u^j) x^n k_u^j\nonumber\\
%  &=& \sum_{n,j} f(x^{N-w(m-n)} k_w k_u^j)x^n k_u^j
\end{equation*}
\begin{equation*}
  z^{-1}f = \sum_{y\in G} f(zy)y= \sum_{n,j} f(x^{m+vn} k_vk_u^j)x^n k_u^j
%  &=& \sum_{y\in G} f(y)z^{-1}y = \sum_{y\in G} f(zy)y\nonumber \\  
%  &=& \sum_{n,j} f(\zmv x^n k_u^j)x^n k_u^j\nonumber\\
%  &=& \sum_{n,j} f(x^{m+vn} k_vk_u^j)x^n k_u^j
\end{equation*}

