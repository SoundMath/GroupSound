
%======================================================================
\section{Translation Invariance}
%======================================================================
%-----------------------------------------------------------------------
\subsection{Generalized Translation and Convolution}
%-----------------------------------------------------------------------
%\ismasubsubsec{General definition of translation}
For $y\in G$, the mapping $\lt{T}(y)$ of $\LG$ defined by 
\begin{equation}\label{eq:trans}
(\lt{T}(y)f)(x) = f(y^{-1}x), \quad x \in G,
\end{equation}
is a linear operator of $\LG$ called 
\emph{left translation by} $y$.

%---
%\ismasubsubsec{General definition of convolution}
%---
The mapping $\lt{C}(f)$ of $\LG$ defined by 
\begin{equation}
\lt{C}(f) = \sum_{y\in G} f(y) \lt{T}(y), \quad f\in \LG,
\end{equation}
is a linear operator of $\LG$ called 
\emph{left convolution by} $f$.  By definition, for $x\in G$,
\begin{equation}\label{eq:conv}
(\lt{C}(f)g)(x) = \sum_{y\in G} f(y) g(y^{-1}x), \quad g \in \LG.
\end{equation}
%The collection of all left convolutions of $\LG$ is
%$\vs{C}(G) = \left\{\lt{C}(f) : f \in G \right\}$.

For $f, g \in \LG$, the composition
$f * g  = \lt{C}(f)g$
is called the \emph{convolution product}.
The vector space $\LG$ paired with the convolution product
is an algebra, the \emph{convolution algebra over} $G$.

To gain some familiarity with the general 
definitions of translation %~(\ref{eq:trans}) 
and convolution, %~(\ref{eq:conv}), 
it helps to verify 
that these definitions agree with what we expect 
when $G$ is a familiar abelian group. 
%---
\begin{example}
%---
If $G=\Z/N$, then~(\ref{eq:trans}) becomes
%translation of $\LG$ by $y\in G$ is defined by 
\begin{equation}
(\lt{T}(y)f)(x) = f(x-y),  \qquad x \in G,
\end{equation}
and~(\ref{eq:conv}) becomes
%convolution of $\LG$ by $g\in \LG$ is defined by
%\begin{equation}
%\lt{C}(g)f = \sum_{y \in G}g(y)\lt{T}(y)f, \qquad f \in \LG,
%\end{equation}
%which, at $x \in G$, is
\begin{equation}
(\lt{C}(g)f)(x) =\sum_{y \in G} g(y)f(x-y).
\end{equation}
\end{example}
\begin{example}
%Consider the translation and convolution operations for the special case
If $G = \Z/M \times \Z/N$, then translation of $\LG$ by
$y\in G$ is given by 
\[
(\T(y)f)(x) = f(x_1-y_1, x_2-y_2), \qquad x \in G.
\]
while convolution of $\LG$ by $g\in \LG$ is given by
\[
\lt{C}(g)f = \sum_{y \in G}g(y)\T(y)f, \qquad f \in \LG.
\]
Evaluated at a point $x = (x_1,x_2) \in G$,
\[
(\lt{C}(g)f)(x) =\sum_{y_1 = 0}^{M-1}\sum_{y_2=0}^{N-1}g(y_1,y_2)f(x_1-y_1,x_2-y_2).
\]
\end{example}
