
%-----------------------------------------------------------------------
\subsection{The Group Algebra $\CG$}
%-----------------------------------------------------------------------
\label{sec:groupalgebra}
The \emph{group algebra} $\CG$ is the space of all formal sums
\begin{equation}
f = \sum_{x\in G} f(x) x, \quad f(x) \in \C,
\end{equation}
with the following operations for $f, g \in \CG$:
\begin{equation}
f+g = \sum_{x\in G} (f(x) + g(x))x,
\end{equation}
\begin{equation}
\alpha f = \sum_{x\in G} (\alpha f(x)) x, 
           \quad \alpha \in \C,
\end{equation}
\begin{equation}
fg = \sum_{x\in G}\left(\sum_{y\in G} f(y)g(y^{-1}x)\right)x. 
\end{equation}

The mapping $\lt{L}(g)$ of $\CG$ defined by 
$\lt{L}(g)f = gf$
is a linear operator on the space $\CG$ called 
\emph{left multiplication by} $g$.  
Since $y\in G$ can be identified with the formal
sum $e_y \in \CG$ consisting of a single nonzero term,
\begin{equation}\label{eq:leftmult}
yf = \lt{L}(e_y)f = \sum_{x\in G}f(y^{-1}x) x.
\end{equation}
In relation to translation of $\LG$, (\ref{eq:leftmult}) is the
$\CG$ analog. Fig.~\ref{fig:cyclicshift} illustrates.

The mapping $\Theta: \LG \to \CG$ defined by
\begin{equation}\label{eq:iso}
\Theta(f) = \sum_{x\in G} f(x) x, \quad f\in \LG,
\end{equation}
is an algebra isomorphism of the convolution algebra $\LG$
onto the group algebra $\CG$.  
Thus we can identify
$\Theta(f)$ with $f$, using  context to decide whether
$f$ refers to the function in $\LG$ or the formal sum in
$\CG$.  

An important aspect of the foregoing isomorphism is the
correspondence between the translations of the spaces.
Translation of $\LG$ by $y\in G$ %$\T(y)$ 
corresponds to left multiplication of $\CG$ by $y\in G$.
%$\lt{L}(y)$ 
Convolution of $\LG$ by $f\in \LG$ corresponds to
left multiplication of $\CG$ by $f\in \CG$. 
%We state 
%these relations symbolically as follows:
%\begin{center}
%\begin{tabular}{ccc}
%  $\LG$ & $\simeq$ & $\CG$ \\
%  $\lt{T}(y)$ & $\leftrightarrow$ & $\lt{L}(y)$\\
%  $\lt{C}(f)$ & $\leftrightarrow$ & $\lt{L}(f)$
%\end{tabular}
%\end{center}

\begin{figure}
\centerline{\framebox{
	\includegraphics[width=\columnwidth]{t_cyclicshift}}}
  \caption{An impulse $f\in \CA$ and a few abelian group translates, $x^2f, x^9f,
      x^{-7}f$.}
  \label{fig:cyclicshift}
\end{figure}

%\begin{figure}
%%  \centerline{\epsfig{figure=figures/t_cyclicshift,width=70mm, height=50mm}}
%  \centering
%\includegraphics[width=80mm, height=70mm]{t_cyclicshift}
%  \caption{An impulse $f\in \CA$ and a few abelian group translates, $x^2f, x^9f,
%      x^{-7}f$.}
%  \label{fig:cyclicshift}
%\end{figure}

%% %------------------------------------------------------------------------
%% \subsection{Ideals: Translation-Invariant Subspaces}
%% %------------------------------------------------------------------------
%% A subspace $\vs{V}$ of the space $\CG$ is called a
%% \emph{left ideal} if 
%% \begin{equation}
%% u\vs{V} = \{uf : f \in \vs{V}\} \subset \vs{V}, \quad u \in G. 
%% \end{equation}
%% A left ideal of $\CG$ corresponds to a subspace of $\LG$
%% invariant under all left translations.  

%% If $\vs{V}$ is a left ideal, then, by linearity, 
%% $g\vs{V} \subset \vs{V}$ for all $g \in \CG$.
%% The set $\CG g$, defined by 
%% $\{fg : f \in \CG\}$, is a left ideal of $\CG$,
%% called \emph{the left ideal generated by} $g$ in $\CG$. 
%% %$\CG g = \CG$ if and only if $g$ is an invertible element in $\CG$. 
%% A left ideal $\vs{V}$ of $\CG$ is called \emph{irreducible}
%% if the only left ideals of $\CG$ contained in $\vs{V}$ are
%% $\{0\}$ and $\vs{V}$. The sum of two distinct, irreducible
%% left ideals is always a direct sum. 
%% % (\cite{An:2003}, p.~129).

%% For \emph{abelian} group $A$, the group algebra $\C A$ of
%% signals is decomposed into a direct sum of irreducible ideals.  
%% Since multiplication of $\C A$ by elements of $A$
%% corresponds to translation, ideals represent
%% translation-invariant subspaces.  Furthermore, in the 
%% abelian case, such translation-invariant subspaces are
%% one-dimensional.   

%% Similarly, for \emph{nonabelian} group $G$, the group algebra
%% $\CG$ is decomposed into a direct sum of left ideals. Here,
%% again, the ideals are translation-invariant 
%% subspaces.  However, some of these subspaces must now be
%% multi-dimensional, and herein lies the potential advantage
%% of using nonabelian groups for indexing the data. The left
%% translations are more general and represent a broader class of 
%% transformations. Therefore, projections of data into the
%% resulting left ideals can reveal more complicated partitions
%% and structures in the data as compared with the Fourier
%% components in the abelian group case. 
