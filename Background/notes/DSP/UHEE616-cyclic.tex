%============================================================
\section{Finite Groups}
%============================================================
\label{sec:hafg}
%\input{hafg/hafg}
%% 'hafg-prelim.tex'
This section summarizes the notations,
definitions, and important facts needed below.
The presentation style is terse since the goal of this
section is to distill from the more general literature only
those results that are most relevant to our application.
%%IF INCLUDING APPENDIX, INCLUDE NEXT SENTENCE:
%%More details are provided in Section~\ref{appsec:abstr-algebr}.  
The books~\cite{An:2003} and~\cite{Tolimieri:1998} treat similar
material in a more thorough and rigorous manner.
Throughout, $\C$ denotes complex numbers, 
$G$ an arbitrary finite (nonabelian) group, %of order $N$, 
and $\LG$ the collection of complex valued functions of $G$.

%---
\subsection{Cyclic Groups}
%---
A group $C$ is called a \emph{cyclic group} if there exists
$x\in C$ such that every $y\in C$ has the form $y=x^n$ for
some integer $n$.  In this case, we call $x$ a
\emph{generator} of $C$. 
Cyclic groups are frequently constructed as special
subgroups of arbitrary groups.  

%Throughout the following discussion, 
If $G$ is an arbitrary finite group, 
and $x\in G$, then the set of powers of $x$,
\begin{equation}
  \label{eq:cyclic-group}
gp_G(x) = \{x^n : n\in \Z\},
\end{equation}
is a cyclic subgroup of $G$ called the 
\emph{group generated by} $x$ \emph{in} $G$.
If the underlying group is understood,
(\ref{eq:cyclic-group}) may be denoted $gp(x)$.
%\end{definition}

It will be convenient to have notation for a cyclic group of
order $N$ without reference to a particular underlying group.
Let the set of formal symbols
\begin{equation}\label{eq:cyclicGroup}
C_N(x) = \{ x^n : 0 \leq n < N\}
\end{equation} 
denote the cyclic group of order $N$ with
generator $x$, and define binary composition by
\begin{equation}\label{eq:binarycomp}
x^m x^n = x^{m+n}, \quad 0\leq m, n < N,
\end{equation}
where $m+n$ is addition modulo $N$.  Then $C_N(x)$ is a
cyclic group of order $N$ having generator $x$. The identity
element of $C_N(x)$ is $x^0 = 1$, and the inverse of $x^n$
in $C_N(x)$ is $x^{N-n}$.

%Most of the groups discussed above have elements
%in the set $\Z/N$, and we wrote the binary composition on
%the group as addition modulo $N$.   
To say that a group is \emph{abelian} is to specify that the binary
composition of the group is commutative, in which case the
symbol $+$ is usually used to represent this operation.
For nonabelian groups, we write the (non-commutative) binary
composition as multiplication.  Since our work involves
both abelian and nonabelian groups, it is notationally
cleaner to write the binary operations of an arbitrary group --
abelian or otherwise -- as multiplication. The
following examples illustrate that additive groups, such as $\Z/N$ with
addition modulo $N$, have simple multiplicative representations.
%\footnote{Appendix
%  Section~\ref{appsec:abelianDSP} discusses elementary
%  abelian groups, such as $\Z/N$ and $\Z/M \times \Z/N$, in
%  greater detail.}  
%~~~
\begin{example}
%~~~
\label{ex:ZN}
Let $\Z/N = \{0, 1, \ldots, N-1\}$,
and let addition modulo $N$ be the binary composition on $\Z/N$.
This group is isomorphic to the cyclic group $C_N(x)$, 
\[
\Z/N = \{ n : 0 \leq n < N \} \simeq \{ x^n : 0 \leq n < N\}
= C_N(x),
\]
and it is by this identification that the binary composition
of $\Z/N$ can be written as multiplication. 
More precisely, by uniquely identifying each element $m \in \Z/N$
with the corresponding element $x^m \in C_N(x)$, the binary composition
$m+n$ is replaced with that of~(\ref{eq:binarycomp}).  
\end{example}

%~~~
\begin{example}
%~~~
\label{ex:ZMZN}
Consider the direct product group
\begin{equation}\label{eq:ZM}
\Z/M \times \Z/N = \{(m, n) : 0 \leq m < M, \, 0 \leq n < N \},
\end{equation}
each element of which might represent a 2-dimensional spatial
coordinate. More generally, identify~(\ref{eq:ZM}) by
isomorphism with the group
\begin{equation}
  \label{eq:star}
C_{M}(x) \times C_{N}(y) = \{x^m y^n : 0 \leq m < M, \, 0 \leq n < N \},  
\end{equation}
and define binary composition as follows:
\[
(x^m y^n) (x^j y^k) = x^{m+j}y^{n+k}, \; 0\leq m, j < M,
\; 0 \leq n, k < N,
\]
where $m+j$ is addition modulo $M$ and $n+k$ is addition modulo $N$.
\end{example}

%~~~
\begin{example}
%~~~
For an integer $L \in \Z/N$, denote by $gp_N(x^L)$ the
subgroup generated by $x^L$ in $C_N(x)$. If $L$ divides $N$,
then   
\[
gp_N(x^L) = \{x^{mL} : 0\leq m < M\}, \quad LM = N,
\]
and $gp_N(x^L)$ is a cyclic group of order $M$.
\end{example}

%---
\subsection{Group of Units}
%---
Multiplication modulo $N$ is a ring product on the group of
integers $\Z/N$. An element $m\in \Z/N$ is called a {\it unit} if
there exists an $n\in \Z/N$ such that $mn = 1$.  The set
$U(N)$ of all units in $\Z/N$ is a group with respect to
multiplication modulo $N$, and is called the 
\emph{group of units}.%unit group} of $\Z/N$. 
The group of units can be described %characterized 
as the set of all integers $0<m<N$ such that $m$ and $N$ are
relatively prime.  
\begin{example}
For $N=8$, 
%\begin{equation}\label{eq:unitGroup}
$U(8) = \{1, 3, 5, 7\}$.
%\end{equation}
\end{example}

%---
\subsection{Quotient Groups}
%---
%Suppose the image $f$ is a real-valued function on the abelian group
In image processing applications the set used to index the data 
is an important factor influencing performance of the
resulting algorithms.
Typically image data are indexed by elements
of direct products of cyclic groups, such as~(\ref{eq:star}).
%\[C_N(x) \times C_N(y) = \{x^m y^n : 0 \leq m,n < N \}.\]
Implicit in our present treatment of such direct product groups 
are some standard identifications, such as $x^1y^0 = (x,1) = x$, and $x^0y^1 =
(1,y) = y$.  There is no ambiguity in this representation, though
it may take some getting used to. The unaccustomed reader is well-advised to 
consult~\cite{An:2003} for reassurance.

Let $A=C_N(x) \times C_N(y)$ and suppose $B$ is the subgroup
of $A$ with elements in
\[
 gp_N(x^L) \times gp_N(y^L) = 
\{x^{pL} y^{qL} : 0 \leq p,q < M \},
\]
where $LM = N$.  The group $B$ is a direct product of cyclic groups of order
$M$. The \emph{quotient group} $A/B$ is given by
\[
A/B = \{x^j y^k B : 0 \leq j,k < L \}.
\]
Each member of $A/B$ is a direct product of cyclic subgroups
of $A$, called a \emph{$B$-coset of $A$}.  More specifically, the
member $x^j y^k B \in A/B$ is called the 
\emph{$B$-coset of $A$ with representative
$x^j y^k$}.  The elements within a particular coset are 
called \emph{equivalent modulo $B$}.
A complete set of \emph{$B$-coset representatives in $A$}  is 
$H = \{x^j y^k : 0 \leq j,k < L \} = C_L(x) \times C_L(y)$.
Thus, $H$ is a direct product of cyclic groups of order $L$.
Furthermore, any element $a\in A$ can be uniquely written as
\[
a = hb, \qquad h\in H,\; b\in B,
\]
where $h$ specifies that $a$ belongs to the coset $hB$, and $b$ identifies $a$
within that coset. 
We give concrete examples of $B$-cosets for a few special cases.  
\begin{example}
For $N=8$, $M=2$, $L=4$,
\begin{equation} \label{eq:I-9}
A = C_8(x) \times C_8(y) = \{x^m y^n : 0 \leq m,n < 8 \},
\end{equation}
and
\[
B = gp_8(x^4) \times gp_8(y^4) = \{x^{p4} y^{q4} : 0 \leq p,q < 2 \}.
\]
In the following table, the numbers denote exponents $mn$ on the
elements $x^my^n \in A$ in (\ref{eq:I-9}). 
\[
\begin{matrix}  
\begin{array}{cc} & n\\
                m & \end{array} & 0 & 1 & 2 & 3 &  & 4 & 5 & 6 & 7\\
0 & \mathbf{00} & 01 & 02 & 03 & | & \mathbf{04} & 05 & 06 & 07\\
1 & 10 & 11 & 12 & 13 & | & 14 & 15 & 16 & 17\\
2 & 20 & 21 & 22 & 23 & | & 24 & 25 & 26 & 27\\
3 & 30 & 31 & 32 & 33 & | & 34 & 35 & 36 & 37\\
\hline
4 & \mathbf{40} & 41 & 42 & 43 & | & \mathbf{44} & 45 & 46 & 47\\
5 & 50 & 51 & 52 & 53 & | & 54 & 55 & 56 & 57\\
6 & 60 & 61 & 62 & 63 & | & 64 & 65 & 66 & 67\\
7 & 70 & 71 & 72 & 73 & | & 74 & 75 & 76 & 77
\end{matrix}\]
For illustrative purposes, the exponents belonging to the
$B$-coset with representative $x^0y^0$ are set in bold font;
that is, 
\[
x^0y^0B = B = \begin{bmatrix} 00 & 04 \\ 40 & 44 \end{bmatrix}.
\]
The $B$-coset with representative $x^1y^0$ is 
\[
x^1y^0B = xB = \begin{bmatrix}10 & 14 \\ 50 & 54 \end{bmatrix}.
\]
A few more examples are the $B$-cosets
\[
yB = \begin{bmatrix} 01 & 05 \\ 41 & 45 \end{bmatrix}, \quad
xyB = \begin{bmatrix} 11 & 15 \\ 51 & 55 \end{bmatrix}
\]
\[
x^2B = \begin{bmatrix} 20 & 24 \\ 60 & 64 \end{bmatrix}, \quad
x^2yB = \begin{bmatrix} 21 & 25 \\ 61 & 65 \end{bmatrix}
\]
\end{example}
which have $B$-coset representatives $x^0y^1$,
$x^1y^1$, $x^2y^0$, and $x^2y^1$, respectively.
